%% Generated by Sphinx.
\def\sphinxdocclass{report}
\documentclass[a4paper,10pt,english]{sphinxmanual}
\ifdefined\pdfpxdimen
   \let\sphinxpxdimen\pdfpxdimen\else\newdimen\sphinxpxdimen
\fi \sphinxpxdimen=.75bp\relax

\PassOptionsToPackage{warn}{textcomp}
\usepackage[utf8]{inputenc}
\ifdefined\DeclareUnicodeCharacter
% support both utf8 and utf8x syntaxes
\edef\sphinxdqmaybe{\ifdefined\DeclareUnicodeCharacterAsOptional\string"\fi}
  \DeclareUnicodeCharacter{\sphinxdqmaybe00A0}{\nobreakspace}
  \DeclareUnicodeCharacter{\sphinxdqmaybe2500}{\sphinxunichar{2500}}
  \DeclareUnicodeCharacter{\sphinxdqmaybe2502}{\sphinxunichar{2502}}
  \DeclareUnicodeCharacter{\sphinxdqmaybe2514}{\sphinxunichar{2514}}
  \DeclareUnicodeCharacter{\sphinxdqmaybe251C}{\sphinxunichar{251C}}
  \DeclareUnicodeCharacter{\sphinxdqmaybe2572}{\textbackslash}
\fi
\usepackage{cmap}
\usepackage[T1]{fontenc}
\usepackage{amsmath,amssymb,amstext}
\usepackage{babel}
\usepackage{amsmath,amsfonts,amssymb,amsthm}
\usepackage{fncychap}
\usepackage{sphinx}
\sphinxsetup{hmargin={0.7in,0.7in}, vmargin={1in,1in},         verbatimwithframe=true,         TitleColor={rgb}{0,0,0},         HeaderFamily=\rmfamily\bfseries,         InnerLinkColor={rgb}{0,0,1},         OuterLinkColor={rgb}{0,0,1}}
\fvset{fontsize=\small}
\usepackage{geometry}

% Include hyperref last.
\usepackage{hyperref}
% Fix anchor placement for figures with captions.
\usepackage{hypcap}% it must be loaded after hyperref.
% Set up styles of URL: it should be placed after hyperref.
\urlstyle{same}

\addto\captionsenglish{\renewcommand{\figurename}{Fig.}}
\addto\captionsenglish{\renewcommand{\tablename}{Table}}
\addto\captionsenglish{\renewcommand{\literalblockname}{Listing}}

\addto\captionsenglish{\renewcommand{\literalblockcontinuedname}{continued from previous page}}
\addto\captionsenglish{\renewcommand{\literalblockcontinuesname}{continues on next page}}
\addto\captionsenglish{\renewcommand{\sphinxnonalphabeticalgroupname}{Non-alphabetical}}
\addto\captionsenglish{\renewcommand{\sphinxsymbolsname}{Symbols}}
\addto\captionsenglish{\renewcommand{\sphinxnumbersname}{Numbers}}

\addto\extrasenglish{\def\pageautorefname{page}}

\setcounter{tocdepth}{0}


        %%%%%%%%%%%%%%%%%%%% Meher %%%%%%%%%%%%%%%%%%
        %%%add number to subsubsection 2=subsection, 3=subsubsection
        %%% below subsubsection is not good idea.
        \setcounter{secnumdepth}{3}
        %
        %%%% Table of content upto 2=subsection, 3=subsubsection
        \setcounter{tocdepth}{2}

        \usepackage{amsmath,amsfonts,amssymb,amsthm}
        \usepackage{graphicx}

        %%% reduce spaces for Table of contents, figures and tables
        %%% it is used "\addtocontents{toc}{\vskip -1.2cm}" etc. in the document
        \usepackage[notlot,nottoc,notlof]{}

        \usepackage{color}
        \usepackage{transparent}
        \usepackage{eso-pic}
        \usepackage{lipsum}

        \usepackage{footnotebackref} %%link at the footnote to go to the place of footnote in the text

        %% spacing between line
        \usepackage{setspace}
        %%%%\onehalfspacing
        %%%%\doublespacing
        \singlespacing


        %%%%%%%%%%% datetime
        \usepackage{datetime}

        \newdateformat{MonthYearFormat}{%
            \monthname[\THEMONTH], \THEYEAR}


        %% RO, LE will not work for 'oneside' layout.
        %% Change oneside to twoside in document class
        \usepackage{fancyhdr}
        \pagestyle{fancy}
        \fancyhf{}

        %%% Alternating Header for oneside
        \fancyhead[L]{\ifthenelse{\isodd{\value{page}}}{ \small \nouppercase{\leftmark} }{}}
        \fancyhead[R]{\ifthenelse{\isodd{\value{page}}}{}{ \small \nouppercase{\rightmark} }}

        %%% Alternating Header for two side
        %\fancyhead[RO]{\small \nouppercase{\rightmark}}
        %\fancyhead[LE]{\small \nouppercase{\leftmark}}

        %% for oneside: change footer at right side. If you want to use Left and right then use same as header defined above.
        \fancyfoot[R]{\ifthenelse{\isodd{\value{page}}}{{\tiny Jeroen Bouwman} }{\href{http://xxx.html}{\tiny xxx}}}

        %%% Alternating Footer for two side
        %\fancyfoot[RO, RE]{\scriptsize Meher Krishna Patel (mekrip@gmail.com)}

        %%% page number
        \fancyfoot[CO, CE]{\thepage}

        \renewcommand{\headrulewidth}{0.5pt}
        \renewcommand{\footrulewidth}{0.5pt}

        \RequirePackage{tocbibind} %%% comment this to remove page number for following
        \addto\captionsenglish{\renewcommand{\contentsname}{Table of contents}}
        \addto\captionsenglish{\renewcommand{\listfigurename}{List of figures}}
        \addto\captionsenglish{\renewcommand{\listtablename}{List of tables}}
        % \addto\captionsenglish{\renewcommand{\chaptername}{Chapter}}


        %%reduce spacing for itemize
        \usepackage{enumitem}
        \setlist{nosep}

        %%%%%%%%%%% Quote Styles at the top of chapter
        \usepackage{epigraph}
        \setlength{\epigraphwidth}{0.8\columnwidth}
        \newcommand{\chapterquote}[2]{\epigraphhead[60]{\epigraph{\textit{#1}}{\textbf {\textit{--#2}}}}}
        %%%%%%%%%%% Quote for all places except Chapter
        \newcommand{\sectionquote}[2]{{\quote{\textit{``#1''}}{\textbf {\textit{--#2}}}}}
    

\title{CASCADe Documentation}
\date{Feb 06, 2019}
\release{0.9 beta}
\author{Jeroen Bouwman}
\newcommand{\sphinxlogo}{\sphinxincludegraphics{Exoplanets-A_Ecusson_Noir_Alpha.png}\par}
\renewcommand{\releasename}{ }
\makeindex
\begin{document}

\pagestyle{empty}

        \pagenumbering{Roman} %%% to avoid page 1 conflict with actual page 1

        \begin{titlepage}
            \centering

            \vspace*{40mm} %%% * is used to give space from top
            \textbf{\Huge {CASCADe Documentation}}

            \vspace{0mm}
            \begin{figure}[!h]
                \centering
                \includegraphics[scale=0.3]{Exoplanets-A_Ecusson_Noir_Alpha.png}
            \end{figure}

            \vspace{0mm}
            \Large \textbf{{Jeroen Bouwman}}

            \small Created on : January, 2019

            \vspace*{0mm}
            \small  Last updated : \MonthYearFormat\today


            %% \vfill adds at the bottom
            \vfill
            \small \textit{More documents are freely available at }{\href{http://xxx.html}{xxx}}
        \end{titlepage}

        \clearpage
        \pagenumbering{roman}
        \tableofcontents
        \listoffigures
        \listoftables
        \clearpage
        \pagenumbering{arabic}

        
\pagestyle{plain}
 
\pagestyle{normal}
\phantomsection\label{\detokenize{index::doc}}




At present several thousand transiting exoplanet systems have been discovered.
For relatively few systems, however, a spectro-photometric characterization of
the planetary atmospheres could be performed due to the tiny photometric signatures
of the atmospheres and the large systematic noise introduced by the used instruments
or the earth atmosphere. Several methods have been developed to deal with instrument
and atmospheric noise. These methods include high precision calibration and modeling
of the instruments, modeling of the noise using methods like principle component
analysis or Gaussian processes and the simultaneous observations of many reference
stars. Though significant progress has been made, most of these methods have drawbacks
as they either have to make too many assumptions or do not fully utilize all
information available in the data to negate the noise terms.

The \DUrole{blue}{CASCADe} project implements a novel “data driven” method, pioneered by
Schoelkopf et al (2016) utilizing the causal connections within a data set,
and uses this to calibrate the spectral timeseries data of single transiting
systems. The current code has been tested successfully to spectroscopic data
obtained with the Spitzer and HST observatories.


\chapter{CASCADe documentation}
\label{\detokenize{index:cascade-documentation}}

\section{Installing CASCADe}
\label{\detokenize{install:installing-cascade}}\label{\detokenize{install::doc}}

\subsection{Using Anaconda}
\label{\detokenize{install:using-anaconda}}
The easiest way is to create an anaconda environment
and to install and run CASCADe wihtin this environment.
In case anaconda is not yet installed on the local system, start with
downloading the installer, which can be found at:
\begin{quote}

\sphinxurl{https://www.anaconda.com/download/}
\end{quote}

Once installed update conda to the latest version:

\sphinxcode{\sphinxupquote{{}`bash
conda update conda
{}`}}

and then create and environment for CASCADe:

\sphinxcode{\sphinxupquote{{}`bash
conda create -{-}name cascade
{}`}}

The conda cascade environment can then be activated with the following command:

\sphinxcode{\sphinxupquote{{}`bash
source activate cascade
{}`}}

One can now install all necessary packages for CASCADe within this environment either with conda install or
pip install. The CASCADe package itself can be downloaded from gitlab.


\subsection{Getting CASCADe from gitlab}
\label{\detokenize{install:getting-cascade-from-gitlab}}
The CASCADe code can be found at gitlab and can be downloaded with git. Note that the gitlab repository
is private and to download the code, one needs to create a gitlab account after which the account can be
added to the list of users which have acces to the repository.

The CASCADe repository path is, depending if the HTTPS or SSH protocol is used:
\begin{itemize}
\item {} 
HTTPS: \sphinxtitleref{https://gitlab.com/jbouwman/CASCADe}

\item {} 
SSH: {}`{}` \sphinxhref{mailto:git@gitlab.com}{git@gitlab.com}:jbouwman/CASCADe {}`{}`

\end{itemize}

To get started, open a terminal window in the directory
you wish to clone the repository files into, and run one
of the following commands:

Clone via HTTPS:

\sphinxcode{\sphinxupquote{{}`bash
git clone https://gitlab.com/jbouwman/CASCADe
{}`}}

Clone via SSH:

\sphinxcode{\sphinxupquote{{}`bash
git clone git@gitlab.com:jbouwman/CASCADe
{}`}}

Both commands will download a copy of the files in a
folder named after the project’s name. You can then navigate to the directory and start working
on it locally. In case one is using Anaconda make sure the environment is activated before installing CASCADe.


\subsection{Updating the CASCADe code}
\label{\detokenize{install:updating-the-cascade-code}}
The insalled CASCADe code can be kept up to date by regularly checking for updates. The latest version can be
pulled from gitlab with the command:

\sphinxcode{\sphinxupquote{{}`bash
git pull master
{}`}}


\section{Using Cascade}
\label{\detokenize{howto:using-cascade}}\label{\detokenize{howto::doc}}
to run the code, first load all needed modules:

\fvset{hllines={, ,}}%
\begin{sphinxVerbatim}[commandchars=\\\{\}]
\PYG{k+kn}{import} \PYG{n+nn}{cascade}
\end{sphinxVerbatim}

Then, create transit spectroscopy object

\fvset{hllines={, ,}}%
\begin{sphinxVerbatim}[commandchars=\\\{\}]
\PYG{n}{tso} \PYG{o}{=} \PYG{n}{cascade}\PYG{o}{.}\PYG{n}{TSO}\PYG{o}{.}\PYG{n}{TSOSuite}\PYG{p}{(}\PYG{p}{)}
\end{sphinxVerbatim}

To reset all previous divined or initialized parameters

\fvset{hllines={, ,}}%
\begin{sphinxVerbatim}[commandchars=\\\{\}]
\PYG{n}{tso}\PYG{o}{.}\PYG{n}{execute}\PYG{p}{(}\PYG{l+s+s2}{\PYGZdq{}}\PYG{l+s+s2}{reset}\PYG{l+s+s2}{\PYGZdq{}}\PYG{p}{)}
\end{sphinxVerbatim}

Initialize the TSO object using ini files which define the data, model parameters and behavior of the causal pixel model implemented in CASCADe.

\fvset{hllines={, ,}}%
\begin{sphinxVerbatim}[commandchars=\\\{\}]
\PYG{n}{path} \PYG{o}{=} \PYG{n}{cascade}\PYG{o}{.}\PYG{n}{initialize}\PYG{o}{.}\PYG{n}{default\PYGZus{}initialization\PYGZus{}path}
\PYG{n}{tso} \PYG{o}{=} \PYG{n}{cascade}\PYG{o}{.}\PYG{n}{TSO}\PYG{o}{.}\PYG{n}{TSOSuite}\PYG{p}{(}\PYG{l+s+s2}{\PYGZdq{}}\PYG{l+s+s2}{initialize}\PYG{l+s+s2}{\PYGZdq{}}\PYG{p}{,} \PYG{l+s+s2}{\PYGZdq{}}\PYG{l+s+s2}{cascade\PYGZus{}cpm.ini}\PYG{l+s+s2}{\PYGZdq{}}\PYG{p}{,}
                           \PYG{l+s+s2}{\PYGZdq{}}\PYG{l+s+s2}{cascade\PYGZus{}object.ini}\PYG{l+s+s2}{\PYGZdq{}}\PYG{p}{,}
                           \PYG{l+s+s2}{\PYGZdq{}}\PYG{l+s+s2}{cascade\PYGZus{}data\PYGZus{}spectral\PYGZus{}images.ini}\PYG{l+s+s2}{\PYGZdq{}}\PYG{p}{,} \PYG{n}{path}\PYG{o}{=}\PYG{n}{path}\PYG{p}{)}
\end{sphinxVerbatim}

Load the observational data

\fvset{hllines={, ,}}%
\begin{sphinxVerbatim}[commandchars=\\\{\}]
\PYG{n}{tso}\PYG{o}{.}\PYG{n}{execute}\PYG{p}{(}\PYG{l+s+s2}{\PYGZdq{}}\PYG{l+s+s2}{load\PYGZus{}data}\PYG{l+s+s2}{\PYGZdq{}}\PYG{p}{)}
\end{sphinxVerbatim}

Subtract the background

\fvset{hllines={, ,}}%
\begin{sphinxVerbatim}[commandchars=\\\{\}]
\PYG{n}{tso}\PYG{o}{.}\PYG{n}{execute}\PYG{p}{(}\PYG{l+s+s2}{\PYGZdq{}}\PYG{l+s+s2}{subtract\PYGZus{}background}\PYG{l+s+s2}{\PYGZdq{}}\PYG{p}{)}
\end{sphinxVerbatim}

Sigma clip data

\fvset{hllines={, ,}}%
\begin{sphinxVerbatim}[commandchars=\\\{\}]
\PYG{n}{tso}\PYG{o}{.}\PYG{n}{execute}\PYG{p}{(}\PYG{l+s+s2}{\PYGZdq{}}\PYG{l+s+s2}{sigma\PYGZus{}clip\PYGZus{}data}\PYG{l+s+s2}{\PYGZdq{}}\PYG{p}{)}
\end{sphinxVerbatim}

Determine the position of source from the spectroscopic data set

\fvset{hllines={, ,}}%
\begin{sphinxVerbatim}[commandchars=\\\{\}]
\PYG{n}{tso}\PYG{o}{.}\PYG{n}{execute}\PYG{p}{(}\PYG{l+s+s2}{\PYGZdq{}}\PYG{l+s+s2}{determine\PYGZus{}source\PYGZus{}position}\PYG{l+s+s2}{\PYGZdq{}}\PYG{p}{)}
\end{sphinxVerbatim}

Set the extraction area within which the signal of the exoplanet will be determined

\fvset{hllines={, ,}}%
\begin{sphinxVerbatim}[commandchars=\\\{\}]
\PYG{n}{tso}\PYG{o}{.}\PYG{n}{execute}\PYG{p}{(}\PYG{l+s+s2}{\PYGZdq{}}\PYG{l+s+s2}{set\PYGZus{}extraction\PYGZus{}mask}\PYG{l+s+s2}{\PYGZdq{}}\PYG{p}{)}
\end{sphinxVerbatim}

Extract the spectrum of the Star + planet in an optimal way

\fvset{hllines={, ,}}%
\begin{sphinxVerbatim}[commandchars=\\\{\}]
\PYG{n}{tso}\PYG{o}{.}\PYG{n}{execute}\PYG{p}{(}\PYG{l+s+s2}{\PYGZdq{}}\PYG{l+s+s2}{optimal\PYGZus{}extraction}\PYG{l+s+s2}{\PYGZdq{}}\PYG{p}{)}
\end{sphinxVerbatim}

Setup the matrix of regressors used to model the noise

\fvset{hllines={, ,}}%
\begin{sphinxVerbatim}[commandchars=\\\{\}]
\PYG{n}{tso}\PYG{o}{.}\PYG{n}{execute}\PYG{p}{(}\PYG{l+s+s2}{\PYGZdq{}}\PYG{l+s+s2}{select\PYGZus{}regressors}\PYG{l+s+s2}{\PYGZdq{}}\PYG{p}{)}
\end{sphinxVerbatim}

Define the eclipse model

\fvset{hllines={, ,}}%
\begin{sphinxVerbatim}[commandchars=\\\{\}]
\PYG{n}{tso}\PYG{o}{.}\PYG{n}{execute}\PYG{p}{(}\PYG{l+s+s2}{\PYGZdq{}}\PYG{l+s+s2}{define\PYGZus{}eclipse\PYGZus{}model}\PYG{l+s+s2}{\PYGZdq{}}\PYG{p}{)}
\end{sphinxVerbatim}

Derive the calibrated time series and fit for the planetary signal

\fvset{hllines={, ,}}%
\begin{sphinxVerbatim}[commandchars=\\\{\}]
\PYG{n}{tso}\PYG{o}{.}\PYG{n}{execute}\PYG{p}{(}\PYG{l+s+s2}{\PYGZdq{}}\PYG{l+s+s2}{calibrate\PYGZus{}timeseries}\PYG{l+s+s2}{\PYGZdq{}}\PYG{p}{)}
\end{sphinxVerbatim}

Extract the planetary signal

\fvset{hllines={, ,}}%
\begin{sphinxVerbatim}[commandchars=\\\{\}]
\PYG{n}{tso}\PYG{o}{.}\PYG{n}{execute}\PYG{p}{(}\PYG{l+s+s2}{\PYGZdq{}}\PYG{l+s+s2}{extract\PYGZus{}spectrum}\PYG{l+s+s2}{\PYGZdq{}}\PYG{p}{)}
\end{sphinxVerbatim}

Correct the extracted planetary signal for non uniform subtraction of average eclipse/transit signal

\fvset{hllines={, ,}}%
\begin{sphinxVerbatim}[commandchars=\\\{\}]
\PYG{n}{tso}\PYG{o}{.}\PYG{n}{execute}\PYG{p}{(}\PYG{l+s+s2}{\PYGZdq{}}\PYG{l+s+s2}{correct\PYGZus{}extracted\PYGZus{}spectrum}\PYG{l+s+s2}{\PYGZdq{}}\PYG{p}{)}
\end{sphinxVerbatim}

Save the planetary signal

\fvset{hllines={, ,}}%
\begin{sphinxVerbatim}[commandchars=\\\{\}]
\PYG{n}{tso}\PYG{o}{.}\PYG{n}{execute}\PYG{p}{(}\PYG{l+s+s2}{\PYGZdq{}}\PYG{l+s+s2}{save\PYGZus{}results}\PYG{l+s+s2}{\PYGZdq{}}\PYG{p}{)}
\end{sphinxVerbatim}

Plot results (planetary spectrum, residual etc.)

\fvset{hllines={, ,}}%
\begin{sphinxVerbatim}[commandchars=\\\{\}]
\PYG{n}{tso}\PYG{o}{.}\PYG{n}{execute}\PYG{p}{(}\PYG{l+s+s2}{\PYGZdq{}}\PYG{l+s+s2}{plot\PYGZus{}results}\PYG{l+s+s2}{\PYGZdq{}}\PYG{p}{)}
\end{sphinxVerbatim}


\section{CASCADe API}
\label{\detokenize{cascade:cascade-api}}\label{\detokenize{cascade::doc}}

\subsection{The cascade.TSO module}
\label{\detokenize{cascade.TSO:module-cascade.TSO.TSO}}\label{\detokenize{cascade.TSO:the-cascade-tso-module}}\label{\detokenize{cascade.TSO::doc}}\index{cascade.TSO.TSO (module)@\spxentry{cascade.TSO.TSO}\spxextra{module}}
The TSO module is the main module of the CASCADe package.
The classes defined in this module define the time series object and
all routines acting upon the TSO instance to extract the spectrum of the
transiting exoplanet.
\index{TSOSuite (class in cascade.TSO.TSO)@\spxentry{TSOSuite}\spxextra{class in cascade.TSO.TSO}}

\begin{fulllineitems}
\phantomsection\label{\detokenize{cascade.TSO:cascade.TSO.TSO.TSOSuite}}\pysiglinewithargsret{\sphinxbfcode{\sphinxupquote{class }}\sphinxbfcode{\sphinxupquote{TSOSuite}}}{\emph{*init\_files}, \emph{path=None}}{}
Bases: \sphinxhref{https://docs.python.org/3/library/functions.html\#object}{\sphinxcode{\sphinxupquote{object}}}

Transit Spectroscopy Object Suite class.

This is the main class containing the light curve data of and transiting
exoplanet and all functionality to calibrate and analyse the light curves
and to extractthe spectrum of the transiting exoplanet.
\begin{quote}\begin{description}
\item[{Parameters}] \leavevmode
\sphinxstyleliteralstrong{\sphinxupquote{init\_files}} (\sphinxtitleref{list} of \sphinxtitleref{str}) \textendash{} List containing all the initialization files needed to run the
CASCADe code.

\item[{Raises}] \leavevmode
\sphinxhref{https://docs.python.org/3/library/exceptions.html\#ValueError}{\sphinxcode{\sphinxupquote{ValueError}}} \textendash{} Raised when commands not recognized as valid

\end{description}\end{quote}

\begin{sphinxadmonition}{note}{Examples}

To make instance of TSOSuite class

\fvset{hllines={, ,}}%
\begin{sphinxVerbatim}[commandchars=\\\{\}]
\PYG{g+gp}{\PYGZgt{}\PYGZgt{}\PYGZgt{} }\PYG{n}{tso} \PYG{o}{=} \PYG{n}{cascade}\PYG{o}{.}\PYG{n}{TSO}\PYG{o}{.}\PYG{n}{TSOSuite}\PYG{p}{(}\PYG{p}{)}
\end{sphinxVerbatim}
\end{sphinxadmonition}
\index{execute() (TSOSuite method)@\spxentry{execute()}\spxextra{TSOSuite method}}

\begin{fulllineitems}
\phantomsection\label{\detokenize{cascade.TSO:cascade.TSO.TSO.TSOSuite.execute}}\pysiglinewithargsret{\sphinxbfcode{\sphinxupquote{execute}}}{\emph{command}, \emph{*init\_files}, \emph{path=None}}{}
Check if command is valid and excecute if True
\begin{quote}\begin{description}
\item[{Parameters}] \leavevmode\begin{itemize}
\item {} 
\sphinxstyleliteralstrong{\sphinxupquote{command}} (\sphinxtitleref{str}) \textendash{} Command to be excecuted. If valid the method corresponding
to the command will be excecuted

\item {} 
\sphinxstyleliteralstrong{\sphinxupquote{*init\_files}} (\sphinxtitleref{tuple} of \sphinxtitleref{str}) \textendash{} Single or multiple file names of the .ini files containing the
parameters defining the observation and calibration settings.

\item {} 
\sphinxstyleliteralstrong{\sphinxupquote{path}} (\sphinxtitleref{str}) \textendash{} (optional) Filepath to the .ini files, standard value in None

\end{itemize}

\item[{Raises}] \leavevmode
\sphinxhref{https://docs.python.org/3/library/exceptions.html\#ValueError}{\sphinxcode{\sphinxupquote{ValueError}}} \textendash{} error is raised if command is not valid

\end{description}\end{quote}

\begin{sphinxadmonition}{note}{Examples}

\fvset{hllines={, ,}}%
\begin{sphinxVerbatim}[commandchars=\\\{\}]
\PYG{g+gp}{\PYGZgt{}\PYGZgt{}\PYGZgt{} }\PYG{n}{tso}\PYG{o}{.}\PYG{n}{execute}\PYG{p}{(}\PYG{l+s+s1}{\PYGZsq{}}\PYG{l+s+s1}{reset}\PYG{l+s+s1}{\PYGZsq{}}\PYG{p}{)}
\end{sphinxVerbatim}
\end{sphinxadmonition}

\end{fulllineitems}

\index{initialize\_TSO() (TSOSuite method)@\spxentry{initialize\_TSO()}\spxextra{TSOSuite method}}

\begin{fulllineitems}
\phantomsection\label{\detokenize{cascade.TSO:cascade.TSO.TSO.TSOSuite.initialize_TSO}}\pysiglinewithargsret{\sphinxbfcode{\sphinxupquote{initialize\_TSO}}}{\emph{*init\_files}, \emph{path=None}}{}
Initializes the TSO object by reading in a single or
multiple .ini files
\begin{quote}\begin{description}
\item[{Parameters}] \leavevmode\begin{itemize}
\item {} 
\sphinxstyleliteralstrong{\sphinxupquote{*init\_files}} (\sphinxtitleref{tuple} of \sphinxtitleref{str}) \textendash{} Single or multiple file names of the .ini files containing the
parameters defining the observation and calibration settings.

\item {} 
\sphinxstyleliteralstrong{\sphinxupquote{path}} (\sphinxtitleref{str}) \textendash{} (optional) Filepath to the .ini files, standard value in None

\end{itemize}

\item[{Variables}] \leavevmode
\sphinxstyleliteralstrong{\sphinxupquote{cascade\_parameters}} \textendash{} cascade.initialize.initialize.configurator

\item[{Raises}] \leavevmode
\sphinxhref{https://docs.python.org/3/library/exceptions.html\#FileNotFoundError}{\sphinxcode{\sphinxupquote{FileNotFoundError}}} \textendash{} Raises error if .ini file is not found

\end{description}\end{quote}

\begin{sphinxadmonition}{note}{Examples}

\fvset{hllines={, ,}}%
\begin{sphinxVerbatim}[commandchars=\\\{\}]
\PYG{g+gp}{\PYGZgt{}\PYGZgt{}\PYGZgt{} }\PYG{n}{tso}\PYG{o}{.}\PYG{n}{execute}\PYG{p}{(}\PYG{l+s+s2}{\PYGZdq{}}\PYG{l+s+s2}{initialize}\PYG{l+s+s2}{\PYGZdq{}}\PYG{p}{,} \PYG{n}{init\PYGZus{}flle\PYGZus{}name}\PYG{p}{)}
\end{sphinxVerbatim}
\end{sphinxadmonition}

\end{fulllineitems}

\index{reset\_TSO() (TSOSuite method)@\spxentry{reset\_TSO()}\spxextra{TSOSuite method}}

\begin{fulllineitems}
\phantomsection\label{\detokenize{cascade.TSO:cascade.TSO.TSO.TSOSuite.reset_TSO}}\pysiglinewithargsret{\sphinxbfcode{\sphinxupquote{reset\_TSO}}}{}{}
Reset initialization of TSO object by removing all loaded parameters.

\begin{sphinxadmonition}{note}{Examples}

\fvset{hllines={, ,}}%
\begin{sphinxVerbatim}[commandchars=\\\{\}]
\PYG{g+gp}{\PYGZgt{}\PYGZgt{}\PYGZgt{} }\PYG{n}{tso}\PYG{o}{.}\PYG{n}{execute}\PYG{p}{(}\PYG{l+s+s2}{\PYGZdq{}}\PYG{l+s+s2}{reset}\PYG{l+s+s2}{\PYGZdq{}}\PYG{p}{)}
\end{sphinxVerbatim}
\end{sphinxadmonition}

\end{fulllineitems}

\index{load\_data() (TSOSuite method)@\spxentry{load\_data()}\spxextra{TSOSuite method}}

\begin{fulllineitems}
\phantomsection\label{\detokenize{cascade.TSO:cascade.TSO.TSO.TSOSuite.load_data}}\pysiglinewithargsret{\sphinxbfcode{\sphinxupquote{load\_data}}}{}{}
Load the transit time series observations from file, for the
object, observatory, instrument and file location specified in the
loaded initialization files
\begin{quote}\begin{description}
\item[{Variables}] \leavevmode
\sphinxstyleliteralstrong{\sphinxupquote{observation}} (\sphinxtitleref{cascade.instruments.instruments.Observation}) \textendash{} Instance of Observation class containing all observational data

\end{description}\end{quote}

\begin{sphinxadmonition}{note}{Examples}

To load the observed data into the tso object:

\fvset{hllines={, ,}}%
\begin{sphinxVerbatim}[commandchars=\\\{\}]
\PYG{g+gp}{\PYGZgt{}\PYGZgt{}\PYGZgt{} }\PYG{n}{tso}\PYG{o}{.}\PYG{n}{execute}\PYG{p}{(}\PYG{l+s+s2}{\PYGZdq{}}\PYG{l+s+s2}{load\PYGZus{}data}\PYG{l+s+s2}{\PYGZdq{}}\PYG{p}{)}
\end{sphinxVerbatim}
\end{sphinxadmonition}

\end{fulllineitems}

\index{subtract\_background() (TSOSuite method)@\spxentry{subtract\_background()}\spxextra{TSOSuite method}}

\begin{fulllineitems}
\phantomsection\label{\detokenize{cascade.TSO:cascade.TSO.TSO.TSOSuite.subtract_background}}\pysiglinewithargsret{\sphinxbfcode{\sphinxupquote{subtract\_background}}}{}{}
Subtract median background determined from data or background model
from the science observations.
\begin{quote}\begin{description}
\item[{Variables}] \leavevmode
\sphinxstyleliteralstrong{\sphinxupquote{isBackgroundSubtracted}} (\sphinxtitleref{bool}) \textendash{} \sphinxtitleref{True} if background is subtracted

\item[{Raises}] \leavevmode
\sphinxhref{https://docs.python.org/3/library/exceptions.html\#AttributeError}{\sphinxcode{\sphinxupquote{AttributeError}}} \textendash{} In case no background data is defined

\end{description}\end{quote}

\begin{sphinxadmonition}{note}{Examples}

To subtract the background from the spectral images:

\fvset{hllines={, ,}}%
\begin{sphinxVerbatim}[commandchars=\\\{\}]
\PYG{g+gp}{\PYGZgt{}\PYGZgt{}\PYGZgt{} }\PYG{n}{tso}\PYG{o}{.}\PYG{n}{execute}\PYG{p}{(}\PYG{l+s+s2}{\PYGZdq{}}\PYG{l+s+s2}{subtract\PYGZus{}background}\PYG{l+s+s2}{\PYGZdq{}}\PYG{p}{)}
\end{sphinxVerbatim}
\end{sphinxadmonition}

\end{fulllineitems}

\index{sigma\_clip\_data\_cosmic() (TSOSuite static method)@\spxentry{sigma\_clip\_data\_cosmic()}\spxextra{TSOSuite static method}}

\begin{fulllineitems}
\phantomsection\label{\detokenize{cascade.TSO:cascade.TSO.TSO.TSOSuite.sigma_clip_data_cosmic}}\pysiglinewithargsret{\sphinxbfcode{\sphinxupquote{static }}\sphinxbfcode{\sphinxupquote{sigma\_clip\_data\_cosmic}}}{\emph{data}, \emph{sigma}}{}
Sigma clip of time series data cube allong the time axis.
\begin{quote}\begin{description}
\item[{Parameters}] \leavevmode\begin{itemize}
\item {} 
\sphinxstyleliteralstrong{\sphinxupquote{data}} (\sphinxtitleref{ndarray}) \textendash{} Input data to be cliped, last axis of data to be assumed the time

\item {} 
\sphinxstyleliteralstrong{\sphinxupquote{sigma}} (\sphinxtitleref{float}) \textendash{} Sigma value of sigmaclip

\end{itemize}

\item[{Returns}] \leavevmode
\sphinxstylestrong{sigma\_clip\_mask} (\sphinxtitleref{ndarray}) \textendash{} Updated mask for input data with bad data points flagged \sphinxtitleref{(=1)}

\end{description}\end{quote}

\end{fulllineitems}

\index{sigma\_clip\_data() (TSOSuite method)@\spxentry{sigma\_clip\_data()}\spxextra{TSOSuite method}}

\begin{fulllineitems}
\phantomsection\label{\detokenize{cascade.TSO:cascade.TSO.TSO.TSOSuite.sigma_clip_data}}\pysiglinewithargsret{\sphinxbfcode{\sphinxupquote{sigma\_clip\_data}}}{}{}
Perform sigma clip on science data to flag bad data.
\begin{quote}\begin{description}
\item[{Variables}] \leavevmode
\sphinxstyleliteralstrong{\sphinxupquote{isSigmaCliped}} (\sphinxtitleref{bool}) \textendash{} Set to \sphinxtitleref{True} if bad data has been masked using sigma clip

\item[{Raises}] \leavevmode
\sphinxhref{https://docs.python.org/3/library/exceptions.html\#AttributeError}{\sphinxcode{\sphinxupquote{AttributeError}}} \textendash{} Error is raised if sigma value, the filter length or
the convolution kernel are not defined.

\end{description}\end{quote}

\begin{sphinxadmonition}{note}{Examples}

To sigma clip the observation data stored in an instance of a TSO
object, run the following example:

\fvset{hllines={, ,}}%
\begin{sphinxVerbatim}[commandchars=\\\{\}]
\PYG{g+gp}{\PYGZgt{}\PYGZgt{}\PYGZgt{} }\PYG{n}{tso}\PYG{o}{.}\PYG{n}{execute}\PYG{p}{(}\PYG{l+s+s2}{\PYGZdq{}}\PYG{l+s+s2}{sigma\PYGZus{}clip\PYGZus{}data}\PYG{l+s+s2}{\PYGZdq{}}\PYG{p}{)}
\end{sphinxVerbatim}
\end{sphinxadmonition}

\end{fulllineitems}

\index{create\_cleaned\_dataset() (TSOSuite method)@\spxentry{create\_cleaned\_dataset()}\spxextra{TSOSuite method}}

\begin{fulllineitems}
\phantomsection\label{\detokenize{cascade.TSO:cascade.TSO.TSO.TSOSuite.create_cleaned_dataset}}\pysiglinewithargsret{\sphinxbfcode{\sphinxupquote{create\_cleaned\_dataset}}}{}{}
Create a cleaned dataset to be used in regresion analysis.
\begin{quote}\begin{description}
\item[{Variables}] \leavevmode
\sphinxstyleliteralstrong{\sphinxupquote{cleaned\_data}} (\sphinxtitleref{masked quantity}) \textendash{} A cleaned version of the spctral timeseries data of the transiting
exoplanet system

\item[{Raises}] \leavevmode
\sphinxstyleemphasis{AttributeError, AssertionError} \textendash{} An error is raised if the data set to be cleaned has not been
background subtracted or no sigma clip has been run first to
identify those pixels to be cleaned.

\end{description}\end{quote}

\begin{sphinxadmonition}{note}{Examples}

To create a cleaned version of the spectral data stored in an instance
of a TSO object run:

\fvset{hllines={, ,}}%
\begin{sphinxVerbatim}[commandchars=\\\{\}]
\PYG{g+gp}{\PYGZgt{}\PYGZgt{}\PYGZgt{} }\PYG{n}{tso}\PYG{o}{.}\PYG{n}{execute}\PYG{p}{(}\PYG{l+s+s2}{\PYGZdq{}}\PYG{l+s+s2}{create\PYGZus{}cleaned\PYGZus{}dataset}\PYG{l+s+s2}{\PYGZdq{}}\PYG{p}{)}
\end{sphinxVerbatim}
\end{sphinxadmonition}

\end{fulllineitems}

\index{define\_eclipse\_model() (TSOSuite method)@\spxentry{define\_eclipse\_model()}\spxextra{TSOSuite method}}

\begin{fulllineitems}
\phantomsection\label{\detokenize{cascade.TSO:cascade.TSO.TSO.TSOSuite.define_eclipse_model}}\pysiglinewithargsret{\sphinxbfcode{\sphinxupquote{define\_eclipse\_model}}}{}{}
This function defines the light curve model used to analize the
transit or eclipse. We define both the actual trasit/eclipse signal
as wel as an calibration signal.
\begin{quote}\begin{description}
\item[{Variables}] \leavevmode\begin{itemize}
\item {} 
\sphinxstyleliteralstrong{\sphinxupquote{light\_curve}} (\sphinxtitleref{ndarray}) \textendash{} The lightcurve model

\item {} 
\sphinxstyleliteralstrong{\sphinxupquote{transit\_timing}} (\sphinxtitleref{list}) \textendash{} list with start time and end time of transit

\item {} 
\sphinxstyleliteralstrong{\sphinxupquote{light\_curve\_interpolated}} (\sphinxtitleref{list} of \sphinxtitleref{ndarray}) \textendash{} to the time grid of the observations interpolated lightcurve model

\item {} 
\sphinxstyleliteralstrong{\sphinxupquote{calibration\_signal}} (\sphinxtitleref{list} of \sphinxtitleref{ndarray}) \textendash{} lightcurve model of the calibration signal

\item {} 
\sphinxstyleliteralstrong{\sphinxupquote{transittype}} (\sphinxtitleref{str}) \textendash{} Currently either ‘eclipse’ or ‘transit’

\end{itemize}

\item[{Raises}] \leavevmode
\sphinxhref{https://docs.python.org/3/library/exceptions.html\#AttributeError}{\sphinxcode{\sphinxupquote{AttributeError}}} \textendash{} Raises error if observations not properly defined.

\end{description}\end{quote}

\begin{sphinxadmonition}{note}{Notes}

The only lightcurve models presently incorporated in the CASCADe code
are the ones provide by batman package.
\end{sphinxadmonition}

\begin{sphinxadmonition}{note}{Examples}

To define the lightcurve model appropriate for the observations
loaded into the instance of a TSO obkect, excecute the
following command:

\fvset{hllines={, ,}}%
\begin{sphinxVerbatim}[commandchars=\\\{\}]
\PYG{g+gp}{\PYGZgt{}\PYGZgt{}\PYGZgt{} }\PYG{n}{tso}\PYG{o}{.}\PYG{n}{execute}\PYG{p}{(}\PYG{l+s+s2}{\PYGZdq{}}\PYG{l+s+s2}{define\PYGZus{}eclipse\PYGZus{}model}\PYG{l+s+s2}{\PYGZdq{}}\PYG{p}{)}
\end{sphinxVerbatim}
\end{sphinxadmonition}

\end{fulllineitems}

\index{determine\_source\_position() (TSOSuite method)@\spxentry{determine\_source\_position()}\spxextra{TSOSuite method}}

\begin{fulllineitems}
\phantomsection\label{\detokenize{cascade.TSO:cascade.TSO.TSO.TSOSuite.determine_source_position}}\pysiglinewithargsret{\sphinxbfcode{\sphinxupquote{determine\_source\_position}}}{}{}
This function determines the position of the source in the slit
over time and the spectral trace.

We check if trace and position are already set, if not, determine them
from the data by deriving the “center of light” of the dispersed
light.
\begin{quote}\begin{description}
\item[{Variables}] \leavevmode\begin{itemize}
\item {} 
\sphinxstyleliteralstrong{\sphinxupquote{spectral\_trace}} (\sphinxtitleref{ndarray}) \textendash{} The trace of the dispersed light on the detector normalized
to its median position. In case the data are extracted spectra,
the trace is zero.

\item {} 
\sphinxstyleliteralstrong{\sphinxupquote{position}} (\sphinxtitleref{ndarray}) \textendash{} Postion of the source on the detector in the cross dispersion
directon as a function of time, normalized to the
median position.

\item {} 
\sphinxstyleliteralstrong{\sphinxupquote{median\_position}} (\sphinxtitleref{float}) \textendash{} median source position.

\end{itemize}

\item[{Raises}] \leavevmode
\sphinxstyleemphasis{AttributeError, AssertionError} \textendash{} Raises error if input observational data or type of data is
not properly difined. Also raises arror if data is not sigma cliped

\end{description}\end{quote}

\begin{sphinxadmonition}{note}{Examples}

To determine the position of the source in the cross dispersion
direction from the in the tso object loaded data set, excecute the
following command:

\fvset{hllines={, ,}}%
\begin{sphinxVerbatim}[commandchars=\\\{\}]
\PYG{g+gp}{\PYGZgt{}\PYGZgt{}\PYGZgt{} }\PYG{n}{tso}\PYG{o}{.}\PYG{n}{execute}\PYG{p}{(}\PYG{l+s+s2}{\PYGZdq{}}\PYG{l+s+s2}{determine\PYGZus{}source\PYGZus{}position}\PYG{l+s+s2}{\PYGZdq{}}\PYG{p}{)}
\end{sphinxVerbatim}
\end{sphinxadmonition}

\end{fulllineitems}

\index{set\_extraction\_mask() (TSOSuite method)@\spxentry{set\_extraction\_mask()}\spxextra{TSOSuite method}}

\begin{fulllineitems}
\phantomsection\label{\detokenize{cascade.TSO:cascade.TSO.TSO.TSOSuite.set_extraction_mask}}\pysiglinewithargsret{\sphinxbfcode{\sphinxupquote{set\_extraction\_mask}}}{}{}
Set mask which defines the area of interest within which
a transit signal will be determined. The mask is set along the
spectral trace with a pixel width of nExtractionWidth
\begin{quote}\begin{description}
\item[{Variables}] \leavevmode\begin{itemize}
\item {} 
\sphinxstyleliteralstrong{\sphinxupquote{nExtractionWidth}} (\sphinxtitleref{int}) \textendash{} The width of the extraction aperture , cetered on the
spectral trace of the source. In case of 1d spectral data, a
width of 1 will be assumed.

\item {} 
\sphinxstyleliteralstrong{\sphinxupquote{extraction\_mask}} (\sphinxtitleref{ndarray}) \textendash{} In case data are Spectra : 1D mask
In case data are Spectral images or cubes: 2D mask

\end{itemize}

\item[{Raises}] \leavevmode
\sphinxhref{https://docs.python.org/3/library/exceptions.html\#AttributeError}{\sphinxcode{\sphinxupquote{AttributeError}}} \textendash{} Raises error if the width of the mask or the source position
and spectral trace are not defined.

\end{description}\end{quote}

\begin{sphinxadmonition}{note}{Examples}

To set the extraction mask, which will define the sub set of the data
from which the planetary spectrum will be determined, sexcecute the
following command:

\fvset{hllines={, ,}}%
\begin{sphinxVerbatim}[commandchars=\\\{\}]
\PYG{g+gp}{\PYGZgt{}\PYGZgt{}\PYGZgt{} }\PYG{n}{tso}\PYG{o}{.}\PYG{n}{execute}\PYG{p}{(}\PYG{l+s+s2}{\PYGZdq{}}\PYG{l+s+s2}{set\PYGZus{}extraction\PYGZus{}mask}\PYG{l+s+s2}{\PYGZdq{}}\PYG{p}{)}
\end{sphinxVerbatim}
\end{sphinxadmonition}

\end{fulllineitems}

\index{\_create\_edge\_mask() (TSOSuite method)@\spxentry{\_create\_edge\_mask()}\spxextra{TSOSuite method}}

\begin{fulllineitems}
\phantomsection\label{\detokenize{cascade.TSO:cascade.TSO.TSO.TSOSuite._create_edge_mask}}\pysiglinewithargsret{\sphinxbfcode{\sphinxupquote{\_create\_edge\_mask}}}{\emph{kernel}, \emph{roi\_mask\_cube}}{}
Helper function for the optimal extraction task. This function
creates an edge mask to mask all pixels for which the convolution
kernel extends beyond the region of interest.
\begin{quote}\begin{description}
\item[{Parameters}] \leavevmode\begin{itemize}
\item {} 
\sphinxstyleliteralstrong{\sphinxupquote{kernel}} (\sphinxtitleref{array\_like}) \textendash{} Convolution kernel specific for a given instrument and observing
mode, used in tasks such as replacing bad pixels and
spectral extraction.

\item {} 
\sphinxstyleliteralstrong{\sphinxupquote{roi\_mask}} (\sphinxtitleref{ndarray}) \textendash{} Mask defining the region of interest from which the speectra will
be extracted.

\end{itemize}

\item[{Returns}] \leavevmode
\sphinxstylestrong{edge\_mask} (\sphinxstyleemphasis{‘array\_like’}) \textendash{} The edge mask based on the input kernel and roi\_mask

\end{description}\end{quote}

\end{fulllineitems}

\index{\_create\_extraction\_profile() (TSOSuite method)@\spxentry{\_create\_extraction\_profile()}\spxextra{TSOSuite method}}

\begin{fulllineitems}
\phantomsection\label{\detokenize{cascade.TSO:cascade.TSO.TSO.TSOSuite._create_extraction_profile}}\pysiglinewithargsret{\sphinxbfcode{\sphinxupquote{\_create\_extraction\_profile}}}{\emph{cleaned\_data\_with\_roi\_mask}, \emph{extracted\_spectra}, \emph{kernel}, \emph{mask\_for\_extraction}}{}
Helper function for the optimal extraction task.
This function creates the normilzed source profile used for optimal
extraction. The cleaned data is convolved with an appropriate kernel
to smooth the profile and to increase the SNR. On the edges, where the
kernel extends over the boundary, non convolved data is used to
prevent edge effects.
\begin{quote}\begin{description}
\item[{Parameters}] \leavevmode\begin{itemize}
\item {} 
\sphinxstyleliteralstrong{\sphinxupquote{cleaned\_data\_with\_roi\_mask}} (\sphinxtitleref{numpy.ma.core.MaskedArray}) \textendash{} The cleaned input data from which the extraction profile is
derived. The mask attached to this data defines the region of
interest around the target source.

\item {} 
\sphinxstyleliteralstrong{\sphinxupquote{extracted\_spectra}} (\sphinxtitleref{numpy.ma.core.MaskedArray}) \textendash{} Best guess for the spectrum of the source from which the extraction
profile is determined.

\item {} 
\sphinxstyleliteralstrong{\sphinxupquote{kernel}} (\sphinxtitleref{ndarray}) \textendash{} Convolution kernel used to create smoothed spectral images

\item {} 
\sphinxstyleliteralstrong{\sphinxupquote{mask\_for\_extraction}} (\sphinxtitleref{ndarray}) \textendash{} Mask containing all pixels which are flagged as bad in the not
cleaned data, i.e. all pixels which value have been replaced
after cleaning.

\end{itemize}

\item[{Returns}] \leavevmode
\sphinxstylestrong{extraction\_profile} (\sphinxstyleemphasis{‘ndarray’}) \textendash{} The extraction profile used for optimal spectral extraction.

\end{description}\end{quote}

\end{fulllineitems}

\index{\_create\_3dKernel() (TSOSuite method)@\spxentry{\_create\_3dKernel()}\spxextra{TSOSuite method}}

\begin{fulllineitems}
\phantomsection\label{\detokenize{cascade.TSO:cascade.TSO.TSO.TSOSuite._create_3dKernel}}\pysiglinewithargsret{\sphinxbfcode{\sphinxupquote{\_create\_3dKernel}}}{\emph{sigma\_time}}{}
Helper function for the optimal extraction taks. This function
creates a 3d kernel from a 2d instrument specific kernel
to include the time dimention thus enabling convolution in both the
spatial and wavelength direction as well as along the time axis.
\begin{quote}\begin{description}
\item[{Parameters}] \leavevmode
\sphinxstyleliteralstrong{\sphinxupquote{sigma\_time}} (\sphinxtitleref{float}) \textendash{} 

\item[{Returns}] \leavevmode
\sphinxstylestrong{3dKernel} (\sphinxtitleref{ndarray})

\item[{Raises}] \leavevmode
\sphinxhref{https://docs.python.org/3/library/exceptions.html\#AttributeError}{\sphinxcode{\sphinxupquote{AttributeError}}} \textendash{} An error is raised if no instrument specific kernel is defined.

\end{description}\end{quote}

\end{fulllineitems}

\index{optimal\_extraction() (TSOSuite method)@\spxentry{optimal\_extraction()}\spxextra{TSOSuite method}}

\begin{fulllineitems}
\phantomsection\label{\detokenize{cascade.TSO:cascade.TSO.TSO.TSOSuite.optimal_extraction}}\pysiglinewithargsret{\sphinxbfcode{\sphinxupquote{optimal\_extraction}}}{}{}
Optimally extract spectrum using procedure of Horne 1986
%
\begin{footnote}[1]\sphinxAtStartFootnote
Horne 1986, PASP 98, 609
%
\end{footnote} The extraction consists of two iterations: The first one to
determine the extraction profile, the second itereation to
extract the spectrum of the target source.

\begin{sphinxadmonition}{note}{Notes}

We use a convolution with a kernel elongated along the spectral trace
rather than a polynomial fit along the trace as in the original paper
by Horne 1986.
\end{sphinxadmonition}
\begin{quote}\begin{description}
\item[{Variables}] \leavevmode
\sphinxstyleliteralstrong{\sphinxupquote{dataset\_optimal\_extracted}} (\sphinxtitleref{cascade.data\_model.SpectralDataTimeSeries}) \textendash{} Time series if optimally extracted 1d spectra.

\item[{Raises}] \leavevmode
\sphinxstyleemphasis{AttributeError, AssertionError} \textendash{} An error is raised if the data and cleaned data sets are not
defined, the source position is not determined or of the
parameters for the optimal extraction task are not set in the
initialization files.

\end{description}\end{quote}

\begin{sphinxadmonition}{note}{References}
\end{sphinxadmonition}

\begin{sphinxadmonition}{note}{Examples}

To optimally extract a spectrum of the target star which data is stored
in an TSO instance, excecute the following command:

\fvset{hllines={, ,}}%
\begin{sphinxVerbatim}[commandchars=\\\{\}]
\PYG{g+gp}{\PYGZgt{}\PYGZgt{}\PYGZgt{} }\PYG{n}{tso}\PYG{o}{.}\PYG{n}{execute}\PYG{p}{(}\PYG{l+s+s2}{\PYGZdq{}}\PYG{l+s+s2}{optimal\PYGZus{}extraction}\PYG{l+s+s2}{\PYGZdq{}}\PYG{p}{)}
\end{sphinxVerbatim}
\end{sphinxadmonition}

\end{fulllineitems}

\index{select\_regressors() (TSOSuite method)@\spxentry{select\_regressors()}\spxextra{TSOSuite method}}

\begin{fulllineitems}
\phantomsection\label{\detokenize{cascade.TSO:cascade.TSO.TSO.TSOSuite.select_regressors}}\pysiglinewithargsret{\sphinxbfcode{\sphinxupquote{select\_regressors}}}{}{}
Select pixels which will be used as regressors.
\begin{quote}\begin{description}
\item[{Variables}] \leavevmode
\sphinxstyleliteralstrong{\sphinxupquote{regressor\_list}} (\sphinxtitleref{list} of \sphinxtitleref{int}) \textendash{} 
List of regressors, using the following list index:
\begin{itemize}
\item {} 
first index: {[}\# nod{]}

\item {} 
second index: {[}\# valid pixel in extraction mask{]}

\item {} 
third index: {[}0=pixel coord; 1=list of regressors{]}

\item {} 
forth index: {[}0=coordinate wave direction;
1=coordinate spatial direction{]}

\end{itemize}


\end{description}\end{quote}

\begin{sphinxadmonition}{note}{Examples}

To setup the list of regressors for each data point on which the
exoplanet spectum will be based, execute the following command:

\fvset{hllines={, ,}}%
\begin{sphinxVerbatim}[commandchars=\\\{\}]
\PYG{g+gp}{\PYGZgt{}\PYGZgt{}\PYGZgt{} }\PYG{n}{tso}\PYG{o}{.}\PYG{n}{execute}\PYG{p}{(}\PYG{l+s+s2}{\PYGZdq{}}\PYG{l+s+s2}{select\PYGZus{}regressors}\PYG{l+s+s2}{\PYGZdq{}}\PYG{p}{)}
\end{sphinxVerbatim}
\end{sphinxadmonition}

\end{fulllineitems}

\index{get\_design\_matrix() (TSOSuite static method)@\spxentry{get\_design\_matrix()}\spxextra{TSOSuite static method}}

\begin{fulllineitems}
\phantomsection\label{\detokenize{cascade.TSO:cascade.TSO.TSO.TSOSuite.get_design_matrix}}\pysiglinewithargsret{\sphinxbfcode{\sphinxupquote{static }}\sphinxbfcode{\sphinxupquote{get\_design\_matrix}}}{\emph{cleaned\_data\_in}, \emph{original\_mask\_in}, \emph{regressor\_selection}, \emph{nrebin}, \emph{clip=False}, \emph{clip\_pctl\_time=0.0}, \emph{clip\_pctl\_regressors=0.0}}{}
Return the design matrix based on the data set itself
\begin{quote}\begin{description}
\item[{Parameters}] \leavevmode\begin{itemize}
\item {} 
\sphinxstyleliteralstrong{\sphinxupquote{cleaned\_data\_in}} (\sphinxtitleref{masked quantity}) \textendash{} time series data with bad pixels corrected

\item {} 
\sphinxstyleliteralstrong{\sphinxupquote{original\_mask\_in}} (\sphinxtitleref{ndarray}) \textendash{} data mask before cleaning

\item {} 
\sphinxstyleliteralstrong{\sphinxupquote{regressor\_selection}} (\sphinxtitleref{list} of \sphinxtitleref{int}) \textendash{} List of index values of the data used as regressors

\item {} 
\sphinxstyleliteralstrong{\sphinxupquote{nrebin}} (\sphinxtitleref{int}) \textendash{} Rebinning value for regressions {[}LEAVE at 1!!{]}

\item {} 
\sphinxstyleliteralstrong{\sphinxupquote{clip}} (\sphinxtitleref{bool}) \textendash{} If ‘True{}` bad regressors will be clipped out of selection

\item {} 
\sphinxstyleliteralstrong{\sphinxupquote{clip\_pctl\_time}} (\sphinxtitleref{float}) \textendash{} Percentile of ‘worst’ regressors to be cut out in the
time direction.

\item {} 
\sphinxstyleliteralstrong{\sphinxupquote{clip\_pctl\_regressors}} (\sphinxtitleref{float}) \textendash{} Percentile of ‘worst’ regressors to be cut out in the
wavelegth direction.

\end{itemize}

\item[{Returns}] \leavevmode
\sphinxstyleemphasis{design\_matrix} \textendash{} The design matrix used in the causal pixel regression model

\end{description}\end{quote}

\end{fulllineitems}

\index{reshape\_data() (TSOSuite method)@\spxentry{reshape\_data()}\spxextra{TSOSuite method}}

\begin{fulllineitems}
\phantomsection\label{\detokenize{cascade.TSO:cascade.TSO.TSO.TSOSuite.reshape_data}}\pysiglinewithargsret{\sphinxbfcode{\sphinxupquote{reshape\_data}}}{\emph{data\_in}}{}
Reshape the time series data to a uniform dimentional shape
\begin{quote}\begin{description}
\item[{Parameters}] \leavevmode
\sphinxstyleliteralstrong{\sphinxupquote{data\_in}} (\sphinxtitleref{ndarray}) \textendash{} 

\item[{Returns}] \leavevmode
\sphinxstylestrong{data\_out} (\sphinxtitleref{ndarray})

\end{description}\end{quote}

\end{fulllineitems}

\index{return\_all\_design\_matrices() (TSOSuite method)@\spxentry{return\_all\_design\_matrices()}\spxextra{TSOSuite method}}

\begin{fulllineitems}
\phantomsection\label{\detokenize{cascade.TSO:cascade.TSO.TSO.TSOSuite.return_all_design_matrices}}\pysiglinewithargsret{\sphinxbfcode{\sphinxupquote{return\_all\_design\_matrices}}}{\emph{clip=False}, \emph{clip\_pctl\_time=0.0}, \emph{clip\_pctl\_regressors=0.0}}{}
Setup the regression matrix based on the sub set of the data slected
to be used as calibrators.
\begin{quote}\begin{description}
\item[{Parameters}] \leavevmode\begin{itemize}
\item {} 
\sphinxstyleliteralstrong{\sphinxupquote{clip}} (\sphinxtitleref{bool}) \textendash{} default False

\item {} 
\sphinxstyleliteralstrong{\sphinxupquote{clip\_pctl\_time}} (\sphinxtitleref{float}) \textendash{} Default \sphinxtitleref{0.00}

\item {} 
\sphinxstyleliteralstrong{\sphinxupquote{clip\_pctl\_regressors}} (\sphinxtitleref{float}) \textendash{} Default \sphinxtitleref{0.00}

\end{itemize}

\item[{Variables}] \leavevmode
\sphinxstyleliteralstrong{\sphinxupquote{design\_matrix}} (\sphinxtitleref{list’ of {}`ndarray}) \textendash{} 
list with design matrici with the following index convention:
\begin{itemize}
\item {} 
first index: {[}\# nods{]}

\item {} 
second index : {[}\# of valid pixels within extraction mask{]}

\item {} 
third index : {[}0{]}

\end{itemize}


\item[{Raises}] \leavevmode
\sphinxhref{https://docs.python.org/3/library/exceptions.html\#AttributeError}{\sphinxcode{\sphinxupquote{AttributeError}}}

\end{description}\end{quote}

\end{fulllineitems}

\index{calibrate\_timeseries() (TSOSuite method)@\spxentry{calibrate\_timeseries()}\spxextra{TSOSuite method}}

\begin{fulllineitems}
\phantomsection\label{\detokenize{cascade.TSO:cascade.TSO.TSO.TSOSuite.calibrate_timeseries}}\pysiglinewithargsret{\sphinxbfcode{\sphinxupquote{calibrate\_timeseries}}}{}{}
This is the main function which runs the regression model to
calibrate the input spectral light curve data and to extract the
planetary signal as function of wavelength.
\begin{quote}\begin{description}
\item[{Variables}] \leavevmode
\sphinxstyleliteralstrong{\sphinxupquote{calibration\_results}} (\sphinxtitleref{SimpleNamespace}) \textendash{} The calibration\_results attribute contains all calibrated data
and auxilary data.

\item[{Raises}] \leavevmode
\sphinxhref{https://docs.python.org/3/library/exceptions.html\#AttributeError}{\sphinxcode{\sphinxupquote{AttributeError}}} \textendash{} an Error is raised if the nessecary steps to be able to run this
task have not been executed properly or if the parameters for
the regression model have not been set in the initialization files.

\end{description}\end{quote}

\begin{sphinxadmonition}{note}{Examples}

To create a calibrated spectral time series and derive the
planetary signal execute the following command:

\fvset{hllines={, ,}}%
\begin{sphinxVerbatim}[commandchars=\\\{\}]
\PYG{g+gp}{\PYGZgt{}\PYGZgt{}\PYGZgt{} }\PYG{n}{tso}\PYG{o}{.}\PYG{n}{execute}\PYG{p}{(}\PYG{l+s+s2}{\PYGZdq{}}\PYG{l+s+s2}{calibrate\PYGZus{}timeseries}\PYG{l+s+s2}{\PYGZdq{}}\PYG{p}{)}
\end{sphinxVerbatim}
\end{sphinxadmonition}

\end{fulllineitems}

\index{extract\_spectrum() (TSOSuite method)@\spxentry{extract\_spectrum()}\spxextra{TSOSuite method}}

\begin{fulllineitems}
\phantomsection\label{\detokenize{cascade.TSO:cascade.TSO.TSO.TSOSuite.extract_spectrum}}\pysiglinewithargsret{\sphinxbfcode{\sphinxupquote{extract\_spectrum}}}{}{}
Extract the planetary spectrum from the calibrated light curve data
\begin{quote}\begin{description}
\item[{Variables}] \leavevmode
\sphinxstyleliteralstrong{\sphinxupquote{exoplanet\_spectrum}} (\sphinxtitleref{SimpleNamespace}) \textendash{} 

\item[{Raises}] \leavevmode
\sphinxhref{https://docs.python.org/3/library/exceptions.html\#AttributeError}{\sphinxcode{\sphinxupquote{AttributeError}}}

\end{description}\end{quote}

\begin{sphinxadmonition}{note}{Examples}

To extract the exoplanet spectum from the calibrated signal,
execute the following command:

\fvset{hllines={, ,}}%
\begin{sphinxVerbatim}[commandchars=\\\{\}]
\PYG{g+gp}{\PYGZgt{}\PYGZgt{}\PYGZgt{} }\PYG{n}{tso}\PYG{o}{.}\PYG{n}{execute}\PYG{p}{(}\PYG{l+s+s2}{\PYGZdq{}}\PYG{l+s+s2}{extract\PYGZus{}spectrum}\PYG{l+s+s2}{\PYGZdq{}}\PYG{p}{)}
\end{sphinxVerbatim}
\end{sphinxadmonition}

\end{fulllineitems}

\index{correct\_extracted\_spectrum() (TSOSuite method)@\spxentry{correct\_extracted\_spectrum()}\spxextra{TSOSuite method}}

\begin{fulllineitems}
\phantomsection\label{\detokenize{cascade.TSO:cascade.TSO.TSO.TSOSuite.correct_extracted_spectrum}}\pysiglinewithargsret{\sphinxbfcode{\sphinxupquote{correct\_extracted\_spectrum}}}{}{}
Make correction for non-uniform subtraction of transit signal due to
differences in the relative weighting of the regressors
\begin{quote}\begin{description}
\item[{Raises}] \leavevmode
\sphinxhref{https://docs.python.org/3/library/exceptions.html\#AttributeError}{\sphinxcode{\sphinxupquote{AttributeError}}}

\end{description}\end{quote}

\begin{sphinxadmonition}{note}{Examples}

To correct the extracted planetary signal for non-uniform
subtraction of an averige transit depth, execute the following command:

\fvset{hllines={, ,}}%
\begin{sphinxVerbatim}[commandchars=\\\{\}]
\PYG{g+gp}{\PYGZgt{}\PYGZgt{}\PYGZgt{} }\PYG{n}{tso}\PYG{o}{.}\PYG{n}{execute}\PYG{p}{(}\PYG{l+s+s2}{\PYGZdq{}}\PYG{l+s+s2}{correct\PYGZus{}extracted\PYGZus{}spectrum}\PYG{l+s+s2}{\PYGZdq{}}\PYG{p}{)}
\end{sphinxVerbatim}
\end{sphinxadmonition}

\end{fulllineitems}

\index{save\_results() (TSOSuite method)@\spxentry{save\_results()}\spxextra{TSOSuite method}}

\begin{fulllineitems}
\phantomsection\label{\detokenize{cascade.TSO:cascade.TSO.TSO.TSOSuite.save_results}}\pysiglinewithargsret{\sphinxbfcode{\sphinxupquote{save\_results}}}{}{}
Save results
\begin{quote}\begin{description}
\item[{Raises}] \leavevmode
\sphinxhref{https://docs.python.org/3/library/exceptions.html\#AttributeError}{\sphinxcode{\sphinxupquote{AttributeError}}}

\end{description}\end{quote}

\begin{sphinxadmonition}{note}{Examples}

To save the calibrated spectrum, execute the following command:

\fvset{hllines={, ,}}%
\begin{sphinxVerbatim}[commandchars=\\\{\}]
\PYG{g+gp}{\PYGZgt{}\PYGZgt{}\PYGZgt{} }\PYG{n}{tso}\PYG{o}{.}\PYG{n}{execute}\PYG{p}{(}\PYG{l+s+s2}{\PYGZdq{}}\PYG{l+s+s2}{save\PYGZus{}results}\PYG{l+s+s2}{\PYGZdq{}}\PYG{p}{)}
\end{sphinxVerbatim}
\end{sphinxadmonition}

\end{fulllineitems}

\index{plot\_results() (TSOSuite method)@\spxentry{plot\_results()}\spxextra{TSOSuite method}}

\begin{fulllineitems}
\phantomsection\label{\detokenize{cascade.TSO:cascade.TSO.TSO.TSOSuite.plot_results}}\pysiglinewithargsret{\sphinxbfcode{\sphinxupquote{plot\_results}}}{}{}
Plot the extracted planetary spectrum and scaled signal on the
detector.
\begin{quote}\begin{description}
\item[{Raises}] \leavevmode
\sphinxhref{https://docs.python.org/3/library/exceptions.html\#AttributeError}{\sphinxcode{\sphinxupquote{AttributeError}}}

\end{description}\end{quote}

\begin{sphinxadmonition}{note}{Examples}

To plot the planetary signal and other diadnostic plots, execute
the following command:

\fvset{hllines={, ,}}%
\begin{sphinxVerbatim}[commandchars=\\\{\}]
\PYG{g+gp}{\PYGZgt{}\PYGZgt{}\PYGZgt{} }\PYG{n}{tso}\PYG{o}{.}\PYG{n}{execute}\PYG{p}{(}\PYG{l+s+s2}{\PYGZdq{}}\PYG{l+s+s2}{plot\PYGZus{}results}\PYG{l+s+s2}{\PYGZdq{}}\PYG{p}{)}
\end{sphinxVerbatim}
\end{sphinxadmonition}

\end{fulllineitems}


\end{fulllineitems}



\subsection{The cascade.cpm\_model module}
\label{\detokenize{cascade.cpm_model:module-cascade.cpm_model.cpm_model}}\label{\detokenize{cascade.cpm_model:the-cascade-cpm-model-module}}\label{\detokenize{cascade.cpm_model::doc}}\index{cascade.cpm\_model.cpm\_model (module)@\spxentry{cascade.cpm\_model.cpm\_model}\spxextra{module}}
The cpm\_model module defines the solver and other functionality for the
regression model used in causal pixel model.
\index{solve\_linear\_equation() (in module cascade.cpm\_model.cpm\_model)@\spxentry{solve\_linear\_equation()}\spxextra{in module cascade.cpm\_model.cpm\_model}}

\begin{fulllineitems}
\phantomsection\label{\detokenize{cascade.cpm_model:cascade.cpm_model.cpm_model.solve_linear_equation}}\pysiglinewithargsret{\sphinxbfcode{\sphinxupquote{solve\_linear\_equation}}}{\emph{design\_matrix}, \emph{data}, \emph{weights=None}, \emph{cv\_method='gcv'}, \emph{reg\_par=\{'lam0': 1e-06}, \emph{'lam1': 100.0}, \emph{'nlam': 60\}}, \emph{feature\_scaling='norm'}, \emph{degrees\_of\_freedom=None}}{}
Solve linear system using SVD with TIKHONOV regularization
\begin{quote}\begin{description}
\item[{Parameters}] \leavevmode\begin{itemize}
\item {} 
\sphinxstyleliteralstrong{\sphinxupquote{design\_matrx}} (\sphinxtitleref{ndarray} with ‘ndim=2’) \textendash{} Design matrix

\item {} 
\sphinxstyleliteralstrong{\sphinxupquote{data}} (\sphinxtitleref{ndarray}) \textendash{} Data

\item {} 
\sphinxstyleliteralstrong{\sphinxupquote{weights}} (\sphinxtitleref{ndarray}) \textendash{} Weights used in the linear least square minimization

\item {} 
\sphinxstyleliteralstrong{\sphinxupquote{cv\_method}} ((\sphinxtitleref{‘gvc’\textbar{}’b95’\textbar{}’B100’})) \textendash{} 
Method used to find optimal regularization parameter which can be:
\begin{itemize}
\item {} 
’gvc’ :  Generizalize Cross Validation {[}RECOMMENDED!!!{]},

\item {} 
’b95’ :  normalized cumulatative periodogram using 95\% limit,

\item {} 
’B100’:  normalized cumulatative periodogram

\end{itemize}


\item {} 
\sphinxstyleliteralstrong{\sphinxupquote{reg\_par}} (\sphinxtitleref{dict}) \textendash{} 
Parameter describing search grid to find optimal regularization
parameter lambda:
\begin{itemize}
\item {} 
’lam0’ : minimum lambda

\item {} 
’lam1’ : maximum lambda,

\item {} 
’nlam’ : number of grid points

\end{itemize}


\item {} 
\sphinxstyleliteralstrong{\sphinxupquote{feature\_scaling}} ((\sphinxtitleref{‘norm’\textbar{}None})) \textendash{} if the value is set to ‘norm’ all features are normalized using L2
norm else no featue scaling is applied.

\item {} 
\sphinxstyleliteralstrong{\sphinxupquote{degrees\_of\_freedom}} (\sphinxtitleref{int}) \textendash{} Effective  degrees\_of\_freedom, if set to None the value is calculated
from the dimensions of the imput arrays.

\end{itemize}

\item[{Returns}] \leavevmode

\sphinxstylestrong{fit\_results} (\sphinxtitleref{tuple}) \textendash{} In case the feature\_scaling is set to None, the tuble contains the
following parameters:
\begin{itemize}
\item {} 
(fit\_parameters, err\_fit\_parameters, lam\_reg)

\end{itemize}

else the following results are returned:
\begin{itemize}
\item {} 
(fit\_parameters\_scaled, err\_fit\_parameters\_scaled, lam\_reg,
pc\_matrix, fit\_parameters, err\_fit\_parameters)

\end{itemize}


\end{description}\end{quote}

\begin{sphinxadmonition}{note}{Notes}

This routine solves the linear equation
\begin{equation*}
\begin{split}A x = y\end{split}
\end{equation*}
by finding optimal solution \textasciicircum{}x by minimizing
\begin{equation*}
\begin{split}||y-A*\hat{x}||^2 + \lambda * ||\hat{x}||^2\end{split}
\end{equation*}
For details on the implementation see %
\begin{footnote}[1]\sphinxAtStartFootnote
PHD thesis by Diana Maria SIMA, “Regularization techniques in
Model Fitting and Parameter estimation”, KU Leuven 2006
%
\end{footnote}, %
\begin{footnote}[2]\sphinxAtStartFootnote
Hogg et al 2010, “Data analysis recipies: Fitting a model to data”
%
\end{footnote}, %
\begin{footnote}[3]\sphinxAtStartFootnote
Rust \& O’Leaary, “Residual periodograms for choosing regularization
parameters for ill-posed porblems”
%
\end{footnote}, %
\begin{footnote}[4]\sphinxAtStartFootnote
Krakauer et al “Using generalized cross-validationto select
parameters in inversions for regional carbon fluxes”
%
\end{footnote}
\end{sphinxadmonition}

\begin{sphinxadmonition}{note}{References}
\end{sphinxadmonition}

\begin{sphinxadmonition}{note}{Examples}

\fvset{hllines={, ,}}%
\begin{sphinxVerbatim}[commandchars=\\\{\}]
\PYG{g+gp}{\PYGZgt{}\PYGZgt{}\PYGZgt{} }\PYG{k+kn}{import} \PYG{n+nn}{numpy} \PYG{k}{as} \PYG{n+nn}{np}
\PYG{g+gp}{\PYGZgt{}\PYGZgt{}\PYGZgt{} }\PYG{k+kn}{from} \PYG{n+nn}{cascade}\PYG{n+nn}{.}\PYG{n+nn}{cpm\PYGZus{}model} \PYG{k}{import} \PYG{n}{solve\PYGZus{}linear\PYGZus{}equation}
\PYG{g+gp}{\PYGZgt{}\PYGZgt{}\PYGZgt{} }\PYG{n}{A} \PYG{o}{=} \PYG{n}{np}\PYG{o}{.}\PYG{n}{array}\PYG{p}{(}\PYG{p}{[}\PYG{p}{[}\PYG{l+m+mi}{1}\PYG{p}{,} \PYG{l+m+mi}{0}\PYG{p}{,} \PYG{o}{\PYGZhy{}}\PYG{l+m+mi}{1}\PYG{p}{]}\PYG{p}{,} \PYG{p}{[}\PYG{l+m+mi}{0}\PYG{p}{,} \PYG{l+m+mi}{1}\PYG{p}{,} \PYG{l+m+mi}{0}\PYG{p}{]}\PYG{p}{,} \PYG{p}{[}\PYG{l+m+mi}{1}\PYG{p}{,} \PYG{l+m+mi}{0}\PYG{p}{,} \PYG{l+m+mi}{1}\PYG{p}{]}\PYG{p}{,} \PYG{p}{[}\PYG{l+m+mi}{1}\PYG{p}{,} \PYG{l+m+mi}{1}\PYG{p}{,} \PYG{l+m+mi}{0}\PYG{p}{]}\PYG{p}{,} \PYG{p}{[}\PYG{o}{\PYGZhy{}}\PYG{l+m+mi}{1}\PYG{p}{,} \PYG{l+m+mi}{1}\PYG{p}{,} \PYG{l+m+mi}{0}\PYG{p}{]}\PYG{p}{]}\PYG{p}{)}
\PYG{g+gp}{\PYGZgt{}\PYGZgt{}\PYGZgt{} }\PYG{n}{coef} \PYG{o}{=} \PYG{n}{np}\PYG{o}{.}\PYG{n}{array}\PYG{p}{(}\PYG{p}{[}\PYG{l+m+mi}{4}\PYG{p}{,} \PYG{l+m+mi}{2}\PYG{p}{,} \PYG{l+m+mi}{7}\PYG{p}{]}\PYG{p}{)}
\PYG{g+gp}{\PYGZgt{}\PYGZgt{}\PYGZgt{} }\PYG{n}{b} \PYG{o}{=} \PYG{n}{np}\PYG{o}{.}\PYG{n}{dot}\PYG{p}{(}\PYG{n}{A}\PYG{p}{,} \PYG{n}{coef}\PYG{p}{)}
\PYG{g+gp}{\PYGZgt{}\PYGZgt{}\PYGZgt{} }\PYG{n}{b} \PYG{o}{=} \PYG{n}{b} \PYG{o}{+} \PYG{n}{np}\PYG{o}{.}\PYG{n}{random}\PYG{o}{.}\PYG{n}{normal}\PYG{p}{(}\PYG{l+m+mf}{0.0}\PYG{p}{,} \PYG{l+m+mf}{0.01}\PYG{p}{,} \PYG{n}{size}\PYG{o}{=}\PYG{n}{b}\PYG{o}{.}\PYG{n}{size}\PYG{p}{)}
\PYG{g+gp}{\PYGZgt{}\PYGZgt{}\PYGZgt{} }\PYG{n}{results} \PYG{o}{=} \PYG{n}{solve\PYGZus{}linear\PYGZus{}equation}\PYG{p}{(}\PYG{n}{A}\PYG{p}{,} \PYG{n}{b}\PYG{p}{)}
\PYG{g+gp}{\PYGZgt{}\PYGZgt{}\PYGZgt{} }\PYG{n+nb}{print}\PYG{p}{(}\PYG{n}{results}\PYG{p}{)}
\end{sphinxVerbatim}
\end{sphinxadmonition}

\end{fulllineitems}

\index{return\_PCR() (in module cascade.cpm\_model.cpm\_model)@\spxentry{return\_PCR()}\spxextra{in module cascade.cpm\_model.cpm\_model}}

\begin{fulllineitems}
\phantomsection\label{\detokenize{cascade.cpm_model:cascade.cpm_model.cpm_model.return_PCR}}\pysiglinewithargsret{\sphinxbfcode{\sphinxupquote{return\_PCR}}}{\emph{design\_matrix}, \emph{n\_components=None}, \emph{variance\_prior\_scaling=1.0}}{}
Perform principal component regression with marginalization.
To marginalize over the eigen-lightcurves we need to solve
x = (A.T V\textasciicircum{}(-1) A)\textasciicircum{}(-1) * (A.T V\textasciicircum{}(-1) y), where V = C + B.T Lambda B,
with B matrix containing the eigenlightcurves and lambda
the median squared amplitudes of the eigenlightcurves.

\end{fulllineitems}



\subsection{The cascade.data\_model module}
\label{\detokenize{cascade.data_model:module-cascade.data_model.data_model}}\label{\detokenize{cascade.data_model:the-cascade-data-model-module}}\label{\detokenize{cascade.data_model::doc}}\index{cascade.data\_model.data\_model (module)@\spxentry{cascade.data\_model.data\_model}\spxextra{module}}
This module defines the data models for the CASCADe transit spectroscopy code
\index{InstanceDescriptorMixin (class in cascade.data\_model.data\_model)@\spxentry{InstanceDescriptorMixin}\spxextra{class in cascade.data\_model.data\_model}}

\begin{fulllineitems}
\phantomsection\label{\detokenize{cascade.data_model:cascade.data_model.data_model.InstanceDescriptorMixin}}\pysigline{\sphinxbfcode{\sphinxupquote{class }}\sphinxbfcode{\sphinxupquote{InstanceDescriptorMixin}}}
Bases: \sphinxhref{https://docs.python.org/3/library/functions.html\#object}{\sphinxcode{\sphinxupquote{object}}}

Mixin to be able to add descriptor to the instance of the class
and not the class itself

\end{fulllineitems}

\index{UnitDesc (class in cascade.data\_model.data\_model)@\spxentry{UnitDesc}\spxextra{class in cascade.data\_model.data\_model}}

\begin{fulllineitems}
\phantomsection\label{\detokenize{cascade.data_model:cascade.data_model.data_model.UnitDesc}}\pysiglinewithargsret{\sphinxbfcode{\sphinxupquote{class }}\sphinxbfcode{\sphinxupquote{UnitDesc}}}{\emph{keyname}}{}
Bases: \sphinxhref{https://docs.python.org/3/library/functions.html\#object}{\sphinxcode{\sphinxupquote{object}}}

A descriptor for adding auxilary measurements,
setting the property for the unit atribute

\end{fulllineitems}

\index{FlagDesc (class in cascade.data\_model.data\_model)@\spxentry{FlagDesc}\spxextra{class in cascade.data\_model.data\_model}}

\begin{fulllineitems}
\phantomsection\label{\detokenize{cascade.data_model:cascade.data_model.data_model.FlagDesc}}\pysiglinewithargsret{\sphinxbfcode{\sphinxupquote{class }}\sphinxbfcode{\sphinxupquote{FlagDesc}}}{\emph{keyname}}{}
Bases: \sphinxhref{https://docs.python.org/3/library/functions.html\#object}{\sphinxcode{\sphinxupquote{object}}}

A descriptor for adding logical flags

\end{fulllineitems}

\index{AuxilaryInfoDesc (class in cascade.data\_model.data\_model)@\spxentry{AuxilaryInfoDesc}\spxextra{class in cascade.data\_model.data\_model}}

\begin{fulllineitems}
\phantomsection\label{\detokenize{cascade.data_model:cascade.data_model.data_model.AuxilaryInfoDesc}}\pysiglinewithargsret{\sphinxbfcode{\sphinxupquote{class }}\sphinxbfcode{\sphinxupquote{AuxilaryInfoDesc}}}{\emph{keyname}}{}
Bases: \sphinxhref{https://docs.python.org/3/library/functions.html\#object}{\sphinxcode{\sphinxupquote{object}}}

A descriptor for adding Auxilary information to the dataset

\end{fulllineitems}

\index{MeasurementDesc (class in cascade.data\_model.data\_model)@\spxentry{MeasurementDesc}\spxextra{class in cascade.data\_model.data\_model}}

\begin{fulllineitems}
\phantomsection\label{\detokenize{cascade.data_model:cascade.data_model.data_model.MeasurementDesc}}\pysiglinewithargsret{\sphinxbfcode{\sphinxupquote{class }}\sphinxbfcode{\sphinxupquote{MeasurementDesc}}}{\emph{keyname}}{}
Bases: \sphinxhref{https://docs.python.org/3/library/functions.html\#object}{\sphinxcode{\sphinxupquote{object}}}

A descriptor for adding auxilary measurements,
setting the properties for the the measurement and unit

\end{fulllineitems}

\index{SpectralData (class in cascade.data\_model.data\_model)@\spxentry{SpectralData}\spxextra{class in cascade.data\_model.data\_model}}

\begin{fulllineitems}
\phantomsection\label{\detokenize{cascade.data_model:cascade.data_model.data_model.SpectralData}}\pysiglinewithargsret{\sphinxbfcode{\sphinxupquote{class }}\sphinxbfcode{\sphinxupquote{SpectralData}}}{\emph{wavelength=nan}, \emph{wavelength\_unit=None}, \emph{data=nan}, \emph{data\_unit=None}, \emph{uncertainty=nan}, \emph{mask=False}, \emph{**kwargs}}{}
Bases: {\hyperref[\detokenize{cascade.data_model:cascade.data_model.data_model.InstanceDescriptorMixin}]{\sphinxcrossref{\sphinxcode{\sphinxupquote{cascade.data\_model.data\_model.InstanceDescriptorMixin}}}}}

Class defining basic properties of spectral data
In the instance if the SpectralData class
all data are stored internally as numppy arrays. Outputted data are
astropy Quantities unless no units (=None) are specified.
\begin{quote}\begin{description}
\item[{Parameters}] \leavevmode\begin{itemize}
\item {} 
\sphinxstyleliteralstrong{\sphinxupquote{wavelength}} \textendash{} wavelength of data (can be frequencies)

\item {} 
\sphinxstyleliteralstrong{\sphinxupquote{wavelenth\_unit}} \textendash{} The physical unit of the wavelength (uses astropy.units)

\item {} 
\sphinxstyleliteralstrong{\sphinxupquote{data}} \textendash{} spectral data

\item {} 
\sphinxstyleliteralstrong{\sphinxupquote{data\_unit}} \textendash{} the physical unit of the data (uses astropy.units)

\item {} 
\sphinxstyleliteralstrong{\sphinxupquote{uncertainty}} \textendash{} uncertainty on spectral data

\item {} 
\sphinxstyleliteralstrong{\sphinxupquote{mask}} \textendash{} mask defining masked data

\item {} 
\sphinxstyleliteralstrong{\sphinxupquote{**kwargs}} \textendash{} any auxilary data relevant to the spectral data
(like position, detector temperature etc.)
If unit is not explicitly given a unit atribute is added.
Input argument can be instance of astropy quantity.
Auxilary atributes are added to instance of the SpectralData class
and not to the class itself. Only the required input stated above
is always defined for all instances.

\end{itemize}

\end{description}\end{quote}

\begin{sphinxadmonition}{note}{Examples}

To create an instance of a SpectralData object with an
initialization with data using units, run the following code:

\fvset{hllines={, ,}}%
\begin{sphinxVerbatim}[commandchars=\\\{\}]
\PYG{g+gp}{\PYGZgt{}\PYGZgt{}\PYGZgt{} }\PYG{k+kn}{import} \PYG{n+nn}{numpy} \PYG{k}{as} \PYG{n+nn}{np}
\PYG{g+gp}{\PYGZgt{}\PYGZgt{}\PYGZgt{} }\PYG{k+kn}{import} \PYG{n+nn}{astropu}\PYG{n+nn}{.}\PYG{n+nn}{units} \PYG{k}{as} \PYG{n+nn}{u}
\PYG{g+gp}{\PYGZgt{}\PYGZgt{}\PYGZgt{} }\PYG{k+kn}{from} \PYG{n+nn}{cascade}\PYG{n+nn}{.}\PYG{n+nn}{data\PYGZus{}model} \PYG{k}{import} \PYG{n}{SpectralData}
\end{sphinxVerbatim}

\fvset{hllines={, ,}}%
\begin{sphinxVerbatim}[commandchars=\\\{\}]
\PYG{g+gp}{\PYGZgt{}\PYGZgt{}\PYGZgt{} }\PYG{n}{wave} \PYG{o}{=} \PYG{n}{np}\PYG{o}{.}\PYG{n}{array}\PYG{p}{(}\PYG{p}{[}\PYG{l+m+mf}{1.0}\PYG{p}{,} \PYG{l+m+mf}{2.0}\PYG{p}{,} \PYG{l+m+mf}{3.0}\PYG{p}{,} \PYG{l+m+mf}{4.0}\PYG{p}{,} \PYG{l+m+mf}{5.0}\PYG{p}{,} \PYG{l+m+mf}{6.0}\PYG{p}{]}\PYG{p}{)}\PYG{o}{*}\PYG{n}{u}\PYG{o}{.}\PYG{n}{micron}
\PYG{g+gp}{\PYGZgt{}\PYGZgt{}\PYGZgt{} }\PYG{n}{flux} \PYG{o}{=} \PYG{n}{np}\PYG{o}{.}\PYG{n}{array}\PYG{p}{(}\PYG{p}{[}\PYG{l+m+mf}{8.0}\PYG{p}{,} \PYG{l+m+mf}{8.0}\PYG{p}{,} \PYG{l+m+mf}{8.0}\PYG{p}{,} \PYG{l+m+mf}{8.0}\PYG{p}{,} \PYG{l+m+mf}{8.0}\PYG{p}{,} \PYG{l+m+mf}{8.0}\PYG{p}{]}\PYG{p}{)}\PYG{o}{*}\PYG{n}{u}\PYG{o}{.}\PYG{n}{Jy}
\PYG{g+gp}{\PYGZgt{}\PYGZgt{}\PYGZgt{} }\PYG{n}{sd} \PYG{o}{=} \PYG{n}{SpectralData}\PYG{p}{(}\PYG{n}{wavelength}\PYG{o}{=}\PYG{n}{wave}\PYG{p}{,} \PYG{n}{data}\PYG{o}{=}\PYG{n}{flux}\PYG{p}{)}
\end{sphinxVerbatim}

\fvset{hllines={, ,}}%
\begin{sphinxVerbatim}[commandchars=\\\{\}]
\PYG{g+gp}{\PYGZgt{}\PYGZgt{}\PYGZgt{} }\PYG{n+nb}{print}\PYG{p}{(}\PYG{n}{sd}\PYG{o}{.}\PYG{n}{data}\PYG{p}{,} \PYG{n}{sd}\PYG{o}{.}\PYG{n}{wavelength}\PYG{p}{)}
\end{sphinxVerbatim}

To change to convert the units to a different but equivalent unit:

\fvset{hllines={, ,}}%
\begin{sphinxVerbatim}[commandchars=\\\{\}]
\PYG{g+gp}{\PYGZgt{}\PYGZgt{}\PYGZgt{} }\PYG{n}{sd}\PYG{o}{.}\PYG{n}{data\PYGZus{}unit} \PYG{o}{=} \PYG{n}{u}\PYG{o}{.}\PYG{n}{erg}\PYG{o}{/}\PYG{n}{u}\PYG{o}{.}\PYG{n}{s}\PYG{o}{/}\PYG{n}{u}\PYG{o}{.}\PYG{n}{cm}\PYG{o}{*}\PYG{o}{*}\PYG{l+m+mi}{2}\PYG{o}{/}\PYG{n}{u}\PYG{o}{.}\PYG{n}{Hz}
\PYG{g+gp}{\PYGZgt{}\PYGZgt{}\PYGZgt{} }\PYG{n}{sd}\PYG{o}{.}\PYG{n}{wavelength\PYGZus{}unit} \PYG{o}{=} \PYG{n}{u}\PYG{o}{.}\PYG{n}{cm}
\PYG{g+gp}{\PYGZgt{}\PYGZgt{}\PYGZgt{} }\PYG{n+nb}{print}\PYG{p}{(}\PYG{n}{sd}\PYG{o}{.}\PYG{n}{data}\PYG{p}{,} \PYG{n}{sd}\PYG{o}{.}\PYG{n}{wavelength}\PYG{p}{)}
\end{sphinxVerbatim}
\end{sphinxadmonition}
\index{wavelength (SpectralData attribute)@\spxentry{wavelength}\spxextra{SpectralData attribute}}

\begin{fulllineitems}
\phantomsection\label{\detokenize{cascade.data_model:cascade.data_model.data_model.SpectralData.wavelength}}\pysigline{\sphinxbfcode{\sphinxupquote{wavelength}}}
The wavelength atttribute of the SpectralData is defined
through a getter and setter method. This ensures that the
returned wavelength has always a unit associated with it (if the
wavelength\_unit is set) and that the returned  wavelength has the same
dimension and mask as the data attribute.

\end{fulllineitems}

\index{wavelength\_unit (SpectralData attribute)@\spxentry{wavelength\_unit}\spxextra{SpectralData attribute}}

\begin{fulllineitems}
\phantomsection\label{\detokenize{cascade.data_model:cascade.data_model.data_model.SpectralData.wavelength_unit}}\pysigline{\sphinxbfcode{\sphinxupquote{wavelength\_unit}}}
The wavelength\_unit attribute of the SpectralData is defined
through a getter and setter method. This ensures that units can be
updated and the wavelength value will be adjusted accordingly.

\end{fulllineitems}

\index{data (SpectralData attribute)@\spxentry{data}\spxextra{SpectralData attribute}}

\begin{fulllineitems}
\phantomsection\label{\detokenize{cascade.data_model:cascade.data_model.data_model.SpectralData.data}}\pysigline{\sphinxbfcode{\sphinxupquote{data}}}
The data atttribute of the SpectralData is defined through a
getter and setter method. In case data is initialized with a
masked quantity, the data\_unit and mask attributes will be set
automatically.

\end{fulllineitems}

\index{uncertainty (SpectralData attribute)@\spxentry{uncertainty}\spxextra{SpectralData attribute}}

\begin{fulllineitems}
\phantomsection\label{\detokenize{cascade.data_model:cascade.data_model.data_model.SpectralData.uncertainty}}\pysigline{\sphinxbfcode{\sphinxupquote{uncertainty}}}
The uncertainty atttribute of the SpectralData is defined
through a getter and setter method. This ensures that the
returned uncertainty has the same unit associated with it
(if the data\_unit is set) and the same mask as the data attribute.

\end{fulllineitems}

\index{data\_unit (SpectralData attribute)@\spxentry{data\_unit}\spxextra{SpectralData attribute}}

\begin{fulllineitems}
\phantomsection\label{\detokenize{cascade.data_model:cascade.data_model.data_model.SpectralData.data_unit}}\pysigline{\sphinxbfcode{\sphinxupquote{data\_unit}}}
The data\_unit attribute of the SpectralData is defined
through a getter and setter method. This ensures that units can be
updated and the data value will be adjusted accordingly.

\end{fulllineitems}

\index{mask (SpectralData attribute)@\spxentry{mask}\spxextra{SpectralData attribute}}

\begin{fulllineitems}
\phantomsection\label{\detokenize{cascade.data_model:cascade.data_model.data_model.SpectralData.mask}}\pysigline{\sphinxbfcode{\sphinxupquote{mask}}}
The mask atttribute of the SpectralData is defined
through a getter and setter method. This ensures that the
returned mask has the same dimension as the data attribute and
will be set automatically if the input data is a masked array.

\end{fulllineitems}


\end{fulllineitems}

\index{SpectralDataTimeSeries (class in cascade.data\_model.data\_model)@\spxentry{SpectralDataTimeSeries}\spxextra{class in cascade.data\_model.data\_model}}

\begin{fulllineitems}
\phantomsection\label{\detokenize{cascade.data_model:cascade.data_model.data_model.SpectralDataTimeSeries}}\pysiglinewithargsret{\sphinxbfcode{\sphinxupquote{class }}\sphinxbfcode{\sphinxupquote{SpectralDataTimeSeries}}}{\emph{wavelength=nan}, \emph{wavelength\_unit=None}, \emph{data=array({[}{[}nan{]}{]})}, \emph{data\_unit=None}, \emph{uncertainty=array({[}{[}nan{]}{]})}, \emph{mask=False}, \emph{time=nan}, \emph{time\_unit=None}, \emph{**kwargs}}{}
Bases: {\hyperref[\detokenize{cascade.data_model:cascade.data_model.data_model.SpectralData}]{\sphinxcrossref{\sphinxcode{\sphinxupquote{cascade.data\_model.data\_model.SpectralData}}}}}

Class defining timeseries of spectral data. This class inherits from
SpectralData. The data stored within this class has one additional
time dimension
\begin{quote}\begin{description}
\item[{Parameters}] \leavevmode\begin{itemize}
\item {} 
\sphinxstyleliteralstrong{\sphinxupquote{wavelength}} (\sphinxstyleliteralemphasis{\sphinxupquote{'array\_like'}}) \textendash{} wavelength assigned to each data point (can be also be frequencies)

\item {} 
\sphinxstyleliteralstrong{\sphinxupquote{wavelenth\_unit}} (\sphinxstyleliteralemphasis{\sphinxupquote{'astropy.units.core.Unit'}}) \textendash{} The physical unit of the wavelength .

\item {} 
\sphinxstyleliteralstrong{\sphinxupquote{data}} (\sphinxstyleliteralemphasis{\sphinxupquote{'array\_like'}}) \textendash{} The spectral data to be analysed. This can be either spectra (1D),
spectral images (2D) or spectral data cubes (3D).

\item {} 
\sphinxstyleliteralstrong{\sphinxupquote{data\_unit}} (\sphinxstyleliteralemphasis{\sphinxupquote{astropy.units.core.Unit}}) \textendash{} The physical unit of the data.

\item {} 
\sphinxstyleliteralstrong{\sphinxupquote{uncertainty}} \textendash{} The uncertainty associated with the spectral data.

\item {} 
\sphinxstyleliteralstrong{\sphinxupquote{mask}} (\sphinxstyleliteralemphasis{\sphinxupquote{'array\_like'}}) \textendash{} The bad pixel mask flagging all data not to be used.

\item {} 
\sphinxstyleliteralstrong{\sphinxupquote{**kwargs}} \textendash{} any auxilary data relevant to the spectral data
(like position, detector temperature etc.)
If unit is not explicitly given a unit atribute is added.
Input argument can be instance of astropy quantity.
Auxilary atributes are added to instance of the SpectralData class
and not to the class itself. Only the required input stated above
is always defined for all instances.

\item {} 
\sphinxstyleliteralstrong{\sphinxupquote{time}} (\sphinxstyleliteralemphasis{\sphinxupquote{'array\_like'}}) \textendash{} The time of observation assiciated with each data point.

\item {} 
\sphinxstyleliteralstrong{\sphinxupquote{time\_unit}} (\sphinxstyleliteralemphasis{\sphinxupquote{'astropy.units.core.Unit'}}) \textendash{} physical unit of time data

\end{itemize}

\end{description}\end{quote}

\begin{sphinxadmonition}{note}{Examples}

To create an instance of a SpectralDataaTimeSeries object with an
initialization with data using units, run the following code:

\fvset{hllines={, ,}}%
\begin{sphinxVerbatim}[commandchars=\\\{\}]
\PYG{g+gp}{\PYGZgt{}\PYGZgt{}\PYGZgt{} }\PYG{k+kn}{import} \PYG{n+nn}{numpy} \PYG{k}{as} \PYG{n+nn}{np}
\PYG{g+gp}{\PYGZgt{}\PYGZgt{}\PYGZgt{} }\PYG{k+kn}{from} \PYG{n+nn}{cascade}\PYG{n+nn}{.}\PYG{n+nn}{data\PYGZus{}model} \PYG{k}{import} \PYG{n}{SpectralDataTimeSeries}
\end{sphinxVerbatim}

\fvset{hllines={, ,}}%
\begin{sphinxVerbatim}[commandchars=\\\{\}]
\PYG{g+gp}{\PYGZgt{}\PYGZgt{}\PYGZgt{} }\PYG{n}{wave} \PYG{o}{=} \PYG{n}{np}\PYG{o}{.}\PYG{n}{array}\PYG{p}{(}\PYG{p}{[}\PYG{l+m+mf}{1.0}\PYG{p}{,} \PYG{l+m+mf}{2.0}\PYG{p}{,} \PYG{l+m+mf}{3.0}\PYG{p}{,} \PYG{l+m+mf}{4.0}\PYG{p}{,} \PYG{l+m+mf}{5.0}\PYG{p}{,} \PYG{l+m+mf}{6.0}\PYG{p}{]}\PYG{p}{)}\PYG{o}{*}\PYG{n}{u}\PYG{o}{.}\PYG{n}{micron}
\PYG{g+gp}{\PYGZgt{}\PYGZgt{}\PYGZgt{} }\PYG{n}{flux} \PYG{o}{=} \PYG{n}{np}\PYG{o}{.}\PYG{n}{array}\PYG{p}{(}\PYG{p}{[}\PYG{l+m+mf}{8.0}\PYG{p}{,} \PYG{l+m+mf}{8.0}\PYG{p}{,} \PYG{l+m+mf}{8.0}\PYG{p}{,} \PYG{l+m+mf}{8.0}\PYG{p}{,} \PYG{l+m+mf}{8.0}\PYG{p}{,} \PYG{l+m+mf}{8.0}\PYG{p}{]}\PYG{p}{)}\PYG{o}{*}\PYG{n}{u}\PYG{o}{.}\PYG{n}{Jy}
\PYG{g+gp}{\PYGZgt{}\PYGZgt{}\PYGZgt{} }\PYG{n}{time} \PYG{o}{=} \PYG{n}{np}\PYG{o}{.}\PYG{n}{array}\PYG{p}{(}\PYG{p}{[}\PYG{l+m+mf}{240000.0}\PYG{p}{,} \PYG{l+m+mf}{2400001.0}\PYG{p}{,} \PYG{l+m+mf}{2400002.0}\PYG{p}{]}\PYG{p}{)}\PYG{o}{*}\PYG{n}{u}\PYG{o}{.}\PYG{n}{day}
\PYG{g+gp}{\PYGZgt{}\PYGZgt{}\PYGZgt{} }\PYG{n}{flux\PYGZus{}time\PYGZus{}series} \PYG{o}{=} \PYG{n}{np}\PYG{o}{.}\PYG{n}{repeat}\PYG{p}{(}\PYG{n}{flux}\PYG{p}{[}\PYG{p}{:}\PYG{p}{,} \PYG{n}{np}\PYG{o}{.}\PYG{n}{newaxis}\PYG{p}{]}\PYG{p}{,} \PYG{n}{time}\PYG{o}{.}\PYG{n}{shape}\PYG{p}{[}\PYG{l+m+mi}{0}\PYG{p}{]}\PYG{p}{,} \PYG{l+m+mi}{1}\PYG{p}{)}
\PYG{g+gp}{\PYGZgt{}\PYGZgt{}\PYGZgt{} }\PYG{n}{sdt} \PYG{o}{=} \PYG{n}{SpectralDataTimeSeries}\PYG{p}{(}\PYG{n}{wavelength}\PYG{o}{=}\PYG{n}{wave}\PYG{p}{,} \PYG{n}{data}\PYG{o}{=}\PYG{n}{flux\PYGZus{}time\PYGZus{}series}\PYG{p}{,}
\PYG{g+go}{                                 time=time)}
\end{sphinxVerbatim}
\end{sphinxadmonition}
\index{time (SpectralDataTimeSeries attribute)@\spxentry{time}\spxextra{SpectralDataTimeSeries attribute}}

\begin{fulllineitems}
\phantomsection\label{\detokenize{cascade.data_model:cascade.data_model.data_model.SpectralDataTimeSeries.time}}\pysigline{\sphinxbfcode{\sphinxupquote{time}}}
The time atttribute of the SpectralDataTimeSeries is defined
through a getter and setter method. This ensures that the
returned time has always a unit associated with it if the time\_unit is
set and that the returned time has the same dimension and mask as
the data attribute.

\end{fulllineitems}

\index{time\_unit (SpectralDataTimeSeries attribute)@\spxentry{time\_unit}\spxextra{SpectralDataTimeSeries attribute}}

\begin{fulllineitems}
\phantomsection\label{\detokenize{cascade.data_model:cascade.data_model.data_model.SpectralDataTimeSeries.time_unit}}\pysigline{\sphinxbfcode{\sphinxupquote{time\_unit}}}
The time\_unit attribute of the SpectralDataTimeSeries is defined
through a getter and setter method. This ensures that units can be
updated and the time value will be adjusted accordingly.

\end{fulllineitems}


\end{fulllineitems}



\subsection{The cascade.exoplanet\_tools module}
\label{\detokenize{cascade.exoplanet_tools:module-cascade.exoplanet_tools.exoplanet_tools}}\label{\detokenize{cascade.exoplanet_tools:the-cascade-exoplanet-tools-module}}\label{\detokenize{cascade.exoplanet_tools::doc}}\index{cascade.exoplanet\_tools.exoplanet\_tools (module)@\spxentry{cascade.exoplanet\_tools.exoplanet\_tools}\spxextra{module}}
This Module defines the functionality to get catalog data on the targeted
exoplanet and define the model ligth curve for the system.
It also difines some usefull functionality for exoplanet atmosphere analysis.
\index{Kmag (in module cascade.exoplanet\_tools.exoplanet\_tools)@\spxentry{Kmag}\spxextra{in module cascade.exoplanet\_tools.exoplanet\_tools}}

\begin{fulllineitems}
\phantomsection\label{\detokenize{cascade.exoplanet_tools:cascade.exoplanet_tools.exoplanet_tools.Kmag}}\pysigline{\sphinxbfcode{\sphinxupquote{Kmag}}\sphinxbfcode{\sphinxupquote{ = Unit("Kmag")}}}
Definition of generic K band magnitude

\end{fulllineitems}

\index{Vmag (in module cascade.exoplanet\_tools.exoplanet\_tools)@\spxentry{Vmag}\spxextra{in module cascade.exoplanet\_tools.exoplanet\_tools}}

\begin{fulllineitems}
\phantomsection\label{\detokenize{cascade.exoplanet_tools:cascade.exoplanet_tools.exoplanet_tools.Vmag}}\pysigline{\sphinxbfcode{\sphinxupquote{Vmag}}\sphinxbfcode{\sphinxupquote{ = Unit("Vmag")}}}
Definition of a generic Vband magnitude

\end{fulllineitems}

\index{masked\_array\_input() (in module cascade.exoplanet\_tools.exoplanet\_tools)@\spxentry{masked\_array\_input()}\spxextra{in module cascade.exoplanet\_tools.exoplanet\_tools}}

\begin{fulllineitems}
\phantomsection\label{\detokenize{cascade.exoplanet_tools:cascade.exoplanet_tools.exoplanet_tools.masked_array_input}}\pysiglinewithargsret{\sphinxbfcode{\sphinxupquote{masked\_array\_input}}}{\emph{func}}{}
Decorator function to check and handel masked Quantities

If one of the input arguments is wavelength or flux, the array can be
a masked Quantity, masking out only ‘bad’ data. This decorator checks for
masked arrays and upon finding the first masked array, passes the data
and stores the mask to be used to create a masked Quantity after the
function returns.
\begin{quote}\begin{description}
\item[{Parameters}] \leavevmode
\sphinxstyleliteralstrong{\sphinxupquote{func}} (\sphinxstyleliteralemphasis{\sphinxupquote{method}}) \textendash{} Function to be decorated

\end{description}\end{quote}

\end{fulllineitems}

\index{KmagToJy() (in module cascade.exoplanet\_tools.exoplanet\_tools)@\spxentry{KmagToJy()}\spxextra{in module cascade.exoplanet\_tools.exoplanet\_tools}}

\begin{fulllineitems}
\phantomsection\label{\detokenize{cascade.exoplanet_tools:cascade.exoplanet_tools.exoplanet_tools.KmagToJy}}\pysiglinewithargsret{\sphinxbfcode{\sphinxupquote{KmagToJy}}}{\emph{magnitude: Unit("Kmag")}, \emph{system='Johnson'}}{}
Convert Kband Magnitudes to Jy
\begin{quote}\begin{description}
\item[{Parameters}] \leavevmode\begin{itemize}
\item {} 
\sphinxstyleliteralstrong{\sphinxupquote{magnitude}} (\sphinxstyleliteralemphasis{\sphinxupquote{'Kmag'}}) \textendash{} Input K band magnitude to be converted to Jy.

\item {} 
\sphinxstyleliteralstrong{\sphinxupquote{system}} (\sphinxstyleliteralemphasis{\sphinxupquote{'str'}}) \textendash{} optional, either ‘Johnson’ or ‘2MASS’, default is ‘Johnson’

\end{itemize}

\item[{Returns}] \leavevmode
\sphinxstylestrong{flux} (\sphinxstyleemphasis{‘astropy.units.Quantity’, u.Jy}) \textendash{} Flux in Jy, converted from input Kband magnitude

\item[{Raises}] \leavevmode
\sphinxhref{https://docs.python.org/3/library/exceptions.html\#AssertionError}{\sphinxcode{\sphinxupquote{AssertionError}}} \textendash{} raises error if Photometric system not recognized

\end{description}\end{quote}

\end{fulllineitems}

\index{JytoKmag() (in module cascade.exoplanet\_tools.exoplanet\_tools)@\spxentry{JytoKmag()}\spxextra{in module cascade.exoplanet\_tools.exoplanet\_tools}}

\begin{fulllineitems}
\phantomsection\label{\detokenize{cascade.exoplanet_tools:cascade.exoplanet_tools.exoplanet_tools.JytoKmag}}\pysiglinewithargsret{\sphinxbfcode{\sphinxupquote{JytoKmag}}}{\emph{flux: Unit("Jy")}, \emph{system='Johnson'}}{}
Convert flux in Jy to Kband Magnitudes
\begin{quote}\begin{description}
\item[{Parameters}] \leavevmode\begin{itemize}
\item {} 
\sphinxstyleliteralstrong{\sphinxupquote{flux}} (\sphinxstyleliteralemphasis{\sphinxupquote{'astropy.units.Quantity'}}\sphinxstyleliteralemphasis{\sphinxupquote{, }}\sphinxstyleliteralemphasis{\sphinxupquote{'u.Jy}}\sphinxstyleliteralemphasis{\sphinxupquote{ or }}\sphinxstyleliteralemphasis{\sphinxupquote{equivalent'}}) \textendash{} Input Flux to be converted K band magnitude.

\item {} 
\sphinxstyleliteralstrong{\sphinxupquote{system}} (\sphinxstyleliteralemphasis{\sphinxupquote{'str'}}) \textendash{} optional, either ‘Johnson’ or ‘2MASS’, default is ‘Johnson’

\end{itemize}

\item[{Returns}] \leavevmode
\sphinxstylestrong{magnitude} (\sphinxstyleemphasis{‘astropy.units.Quantity’, Kmag}) \textendash{} Magnitude  converted from input fkux value

\item[{Raises}] \leavevmode
\sphinxhref{https://docs.python.org/3/library/exceptions.html\#AssertionError}{\sphinxcode{\sphinxupquote{AssertionError}}} \textendash{} raises error if Photometric system not recognized

\end{description}\end{quote}

\end{fulllineitems}

\index{Planck() (in module cascade.exoplanet\_tools.exoplanet\_tools)@\spxentry{Planck()}\spxextra{in module cascade.exoplanet\_tools.exoplanet\_tools}}

\begin{fulllineitems}
\phantomsection\label{\detokenize{cascade.exoplanet_tools:cascade.exoplanet_tools.exoplanet_tools.Planck}}\pysiglinewithargsret{\sphinxbfcode{\sphinxupquote{Planck}}}{\emph{wavelength: Unit("micron")}, \emph{temperature: Unit("K")}}{}
This function calculates the emisison from a Black Body.
\begin{quote}\begin{description}
\item[{Parameters}] \leavevmode\begin{itemize}
\item {} 
\sphinxstyleliteralstrong{\sphinxupquote{wavelength}} (\sphinxstyleliteralemphasis{\sphinxupquote{'astropy.units.Quantity'}}) \textendash{} Input wavelength in units of microns or equivalent

\item {} 
\sphinxstyleliteralstrong{\sphinxupquote{temperature}} (\sphinxstyleliteralemphasis{\sphinxupquote{'astropy.units.Quantity'}}) \textendash{} Input temperature in units of Kelvin or equivalent

\end{itemize}

\item[{Returns}] \leavevmode
\sphinxstylestrong{blackbody} (\sphinxstyleemphasis{‘astropy.units.Quantity’}) \textendash{} B\_nu in cgs units {[} erg/s/cm2/Hz/sr{]}

\end{description}\end{quote}

\begin{sphinxadmonition}{note}{Examples}

\fvset{hllines={, ,}}%
\begin{sphinxVerbatim}[commandchars=\\\{\}]
\PYG{g+gp}{\PYGZgt{}\PYGZgt{}\PYGZgt{} }\PYG{k+kn}{import} \PYG{n+nn}{cascade}
\PYG{g+gp}{\PYGZgt{}\PYGZgt{}\PYGZgt{} }\PYG{k+kn}{import} \PYG{n+nn}{matplotlib}\PYG{n+nn}{.}\PYG{n+nn}{pyplot} \PYG{k}{as} \PYG{n+nn}{plt}
\PYG{g+gp}{\PYGZgt{}\PYGZgt{}\PYGZgt{} }\PYG{k+kn}{import} \PYG{n+nn}{numpy} \PYG{k}{as} \PYG{n+nn}{np}
\PYG{g+gp}{\PYGZgt{}\PYGZgt{}\PYGZgt{} }\PYG{k+kn}{from} \PYG{n+nn}{astropy}\PYG{n+nn}{.}\PYG{n+nn}{visualization} \PYG{k}{import} \PYG{n}{quantity\PYGZus{}support}
\PYG{g+gp}{\PYGZgt{}\PYGZgt{}\PYGZgt{} }\PYG{k+kn}{import} \PYG{n+nn}{astropy}\PYG{n+nn}{.}\PYG{n+nn}{units} \PYG{k}{as} \PYG{n+nn}{u}
\end{sphinxVerbatim}

\fvset{hllines={, ,}}%
\begin{sphinxVerbatim}[commandchars=\\\{\}]
\PYG{g+gp}{\PYGZgt{}\PYGZgt{}\PYGZgt{} }\PYG{n}{wave} \PYG{o}{=} \PYG{n}{np}\PYG{o}{.}\PYG{n}{arange}\PYG{p}{(}\PYG{l+m+mi}{4}\PYG{p}{,} \PYG{l+m+mi}{15}\PYG{p}{,} \PYG{l+m+mf}{0.05}\PYG{p}{)} \PYG{o}{*} \PYG{n}{u}\PYG{o}{.}\PYG{n}{micron}
\PYG{g+gp}{\PYGZgt{}\PYGZgt{}\PYGZgt{} }\PYG{n}{temp} \PYG{o}{=} \PYG{l+m+mi}{300} \PYG{o}{*} \PYG{n}{u}\PYG{o}{.}\PYG{n}{K}
\PYG{g+gp}{\PYGZgt{}\PYGZgt{}\PYGZgt{} }\PYG{n}{flux} \PYG{o}{=} \PYG{n}{cascade}\PYG{o}{.}\PYG{n}{exoplanet\PYGZus{}tools}\PYG{o}{.}\PYG{n}{Planck}\PYG{p}{(}\PYG{n}{wave}\PYG{p}{,} \PYG{n}{temp}\PYG{p}{)}
\end{sphinxVerbatim}

\fvset{hllines={, ,}}%
\begin{sphinxVerbatim}[commandchars=\\\{\}]
\PYG{g+gp}{\PYGZgt{}\PYGZgt{}\PYGZgt{} }\PYG{k}{with} \PYG{n}{quantity\PYGZus{}support}\PYG{p}{(}\PYG{p}{)}\PYG{p}{:}
\PYG{g+gp}{... }    \PYG{n}{plt}\PYG{o}{.}\PYG{n}{plot}\PYG{p}{(}\PYG{n}{wave}\PYG{p}{,} \PYG{n}{flux}\PYG{p}{)}
\PYG{g+gp}{... }    \PYG{n}{plt}\PYG{o}{.}\PYG{n}{show}\PYG{p}{(}\PYG{p}{)}
\end{sphinxVerbatim}
\end{sphinxadmonition}

\end{fulllineitems}

\index{SurfaceGravity() (in module cascade.exoplanet\_tools.exoplanet\_tools)@\spxentry{SurfaceGravity()}\spxextra{in module cascade.exoplanet\_tools.exoplanet\_tools}}

\begin{fulllineitems}
\phantomsection\label{\detokenize{cascade.exoplanet_tools:cascade.exoplanet_tools.exoplanet_tools.SurfaceGravity}}\pysiglinewithargsret{\sphinxbfcode{\sphinxupquote{SurfaceGravity}}}{\emph{MassPlanet: Unit("jupiterMass")}, \emph{RadiusPlanet: Unit("jupiterRad")}}{}
Calculates surface gravity of planet
\begin{quote}\begin{description}
\item[{Parameters}] \leavevmode\begin{itemize}
\item {} 
\sphinxstyleliteralstrong{\sphinxupquote{MassPlanet}} \textendash{} Mass of planet in units of Jupiter mass or equivalent

\item {} 
\sphinxstyleliteralstrong{\sphinxupquote{RadiusPlanet}} \textendash{} Radius of planet in units of Jupiter radius or equivalent

\end{itemize}

\item[{Returns}] \leavevmode
\sphinxstyleemphasis{sgrav} \textendash{} Surface gravity in units of  m s-2

\end{description}\end{quote}

\end{fulllineitems}

\index{ScaleHeight() (in module cascade.exoplanet\_tools.exoplanet\_tools)@\spxentry{ScaleHeight()}\spxextra{in module cascade.exoplanet\_tools.exoplanet\_tools}}

\begin{fulllineitems}
\phantomsection\label{\detokenize{cascade.exoplanet_tools:cascade.exoplanet_tools.exoplanet_tools.ScaleHeight}}\pysiglinewithargsret{\sphinxbfcode{\sphinxupquote{ScaleHeight}}}{\emph{MeanMolecularMass: Unit("u")}, \emph{SurfaceGravity: Unit("m / s2")}, \emph{Temperature: Unit("K")}}{}
Calculate the scaleheigth of the planet
\begin{quote}\begin{description}
\item[{Parameters}] \leavevmode\begin{itemize}
\item {} 
\sphinxstyleliteralstrong{\sphinxupquote{MeanMolecularMass}} (\sphinxstyleliteralemphasis{\sphinxupquote{'astropy.units.Quantity'}}) \textendash{} in units of mass of the hydrogen atom or equivalent

\item {} 
\sphinxstyleliteralstrong{\sphinxupquote{SurfaceGravity}} (\sphinxstyleliteralemphasis{\sphinxupquote{'astropy.units.Quantity'}}) \textendash{} in units of m s-2 or equivalent

\item {} 
\sphinxstyleliteralstrong{\sphinxupquote{Temperature}} (\sphinxstyleliteralemphasis{\sphinxupquote{'astropy.units.Quantity'}}) \textendash{} in units of K or equivalent

\end{itemize}

\item[{Returns}] \leavevmode
\sphinxstylestrong{ScaleHeight} (\sphinxstyleemphasis{‘astropy.units.Quantity’}) \textendash{} scaleheigth in unit of km

\end{description}\end{quote}

\end{fulllineitems}

\index{TransitDepth() (in module cascade.exoplanet\_tools.exoplanet\_tools)@\spxentry{TransitDepth()}\spxextra{in module cascade.exoplanet\_tools.exoplanet\_tools}}

\begin{fulllineitems}
\phantomsection\label{\detokenize{cascade.exoplanet_tools:cascade.exoplanet_tools.exoplanet_tools.TransitDepth}}\pysiglinewithargsret{\sphinxbfcode{\sphinxupquote{TransitDepth}}}{\emph{RadiusPlanet: Unit("jupiterRad")}, \emph{RadiusStar: Unit("solRad")}}{}
Calculates the depth of the planetary transit assuming one can
neglect the emision from the night side of the planet.
\begin{quote}\begin{description}
\item[{Parameters}] \leavevmode\begin{itemize}
\item {} 
\sphinxstyleliteralstrong{\sphinxupquote{Planet}} (\sphinxstyleliteralemphasis{\sphinxupquote{Radius}}) \textendash{} Planetary radius in Jovian radii or equivalent

\item {} 
\sphinxstyleliteralstrong{\sphinxupquote{Star}} (\sphinxstyleliteralemphasis{\sphinxupquote{Radius}}) \textendash{} Stellar radius in Solar radii or equivalent

\end{itemize}

\item[{Returns}] \leavevmode
\sphinxstyleemphasis{depth} \textendash{} Relative transit depth (unit less)

\end{description}\end{quote}

\end{fulllineitems}

\index{EquilibriumTemperature() (in module cascade.exoplanet\_tools.exoplanet\_tools)@\spxentry{EquilibriumTemperature()}\spxextra{in module cascade.exoplanet\_tools.exoplanet\_tools}}

\begin{fulllineitems}
\phantomsection\label{\detokenize{cascade.exoplanet_tools:cascade.exoplanet_tools.exoplanet_tools.EquilibriumTemperature}}\pysiglinewithargsret{\sphinxbfcode{\sphinxupquote{EquilibriumTemperature}}}{\emph{StellarTemperature: Unit("K")}, \emph{StellarRadius: Unit("solRad")}, \emph{SemiMajorAxis: Unit("AU")}, \emph{Albedo=0.3}, \emph{epsilon=0.7}}{}
Calculate the Equlibrium Temperature of the Planet
\begin{quote}\begin{description}
\item[{Parameters}] \leavevmode\begin{itemize}
\item {} 
\sphinxstyleliteralstrong{\sphinxupquote{StellarTemperature}} (\sphinxstyleliteralemphasis{\sphinxupquote{'astropy.units.Quantity'}}) \textendash{} Temperature of the central star in units of K or equivalent

\item {} 
\sphinxstyleliteralstrong{\sphinxupquote{StellarRadius}} (\sphinxstyleliteralemphasis{\sphinxupquote{'astropy.units.Quantity'}}) \textendash{} Radius of the central star in units of Solar Radii or equivalent

\item {} 
\sphinxstyleliteralstrong{\sphinxupquote{Albedo}} (\sphinxstyleliteralemphasis{\sphinxupquote{'float'}}) \textendash{} Albedo of the planet.

\item {} 
\sphinxstyleliteralstrong{\sphinxupquote{SemiMajorAxis}} (\sphinxstyleliteralemphasis{\sphinxupquote{'astropy.units.Quantity'}}) \textendash{} The semi-major axis of platetary orbit in units of AU or equivalent

\item {} 
\sphinxstyleliteralstrong{\sphinxupquote{epsilon}} (\sphinxstyleliteralemphasis{\sphinxupquote{'float'}}) \textendash{} Green house effect parameter

\end{itemize}

\item[{Returns}] \leavevmode
\sphinxstylestrong{ET} (\sphinxstyleemphasis{‘astropy.units.Quantity’}) \textendash{} Equlibrium Temperature of the exoplanet

\end{description}\end{quote}

\end{fulllineitems}

\index{convert\_spectrum\_to\_brighness\_temperature() (in module cascade.exoplanet\_tools.exoplanet\_tools)@\spxentry{convert\_spectrum\_to\_brighness\_temperature()}\spxextra{in module cascade.exoplanet\_tools.exoplanet\_tools}}

\begin{fulllineitems}
\phantomsection\label{\detokenize{cascade.exoplanet_tools:cascade.exoplanet_tools.exoplanet_tools.convert_spectrum_to_brighness_temperature}}\pysiglinewithargsret{\sphinxbfcode{\sphinxupquote{convert\_spectrum\_to\_brighness\_temperature}}}{\emph{wavelength: Unit("micron")}, \emph{contrast: Unit("\%")}, \emph{StellarTemperature: Unit("K")}, \emph{StellarRadius: Unit("solRad")}, \emph{RadiusPlanet: Unit("jupiterRad")}, \emph{error: Unit("\%") = None}}{}
Function to convert the secondary eclipse spectrum to brightness
temperature.
\begin{quote}\begin{description}
\item[{Parameters}] \leavevmode\begin{itemize}
\item {} 
\sphinxstyleliteralstrong{\sphinxupquote{wavelength}} \textendash{} Wavelength in u.micron or equivalent unit.

\item {} 
\sphinxstyleliteralstrong{\sphinxupquote{contrast}} \textendash{} Contrast between planet and star in u.percent.

\item {} 
\sphinxstyleliteralstrong{\sphinxupquote{StellarTemperature}} \textendash{} Temperature if the star in u.K or equivalent unit.

\item {} 
\sphinxstyleliteralstrong{\sphinxupquote{StellarRadius}} \textendash{} Radius of the star in u.R\_sun or equivalent unit.

\item {} 
\sphinxstyleliteralstrong{\sphinxupquote{RadiusPlanet}} \textendash{} Radius of the planet in u.R\_jupiter or equivalent unit.

\item {} 
\sphinxstyleliteralstrong{\sphinxupquote{error}} \textendash{} (optional) Error on contrast in u.percent (standart value = None).

\end{itemize}

\item[{Returns}] \leavevmode
\begin{itemize}
\item {} 
\sphinxstyleemphasis{brighness\_temperature} \textendash{} Eclipse spectrum in units of brightness temperature.

\item {} 
\sphinxstyleemphasis{error\_brighness\_temperature} \textendash{} (optional) Error on the spectrum in units of brightness temperature.

\end{itemize}


\end{description}\end{quote}

\end{fulllineitems}

\index{eclipse\_to\_transit() (in module cascade.exoplanet\_tools.exoplanet\_tools)@\spxentry{eclipse\_to\_transit()}\spxextra{in module cascade.exoplanet\_tools.exoplanet\_tools}}

\begin{fulllineitems}
\phantomsection\label{\detokenize{cascade.exoplanet_tools:cascade.exoplanet_tools.exoplanet_tools.eclipse_to_transit}}\pysiglinewithargsret{\sphinxbfcode{\sphinxupquote{eclipse\_to\_transit}}}{\emph{eclipse}}{}
Converts eclipse spectrum to transit spectrum
\begin{quote}\begin{description}
\item[{Parameters}] \leavevmode
\sphinxstyleliteralstrong{\sphinxupquote{eclipse}} \textendash{} Transit depth values to be converted

\item[{Returns}] \leavevmode
\sphinxstyleemphasis{transit} \textendash{} transit depth values derived from input eclipse values

\end{description}\end{quote}

\end{fulllineitems}

\index{transit\_to\_eclipse() (in module cascade.exoplanet\_tools.exoplanet\_tools)@\spxentry{transit\_to\_eclipse()}\spxextra{in module cascade.exoplanet\_tools.exoplanet\_tools}}

\begin{fulllineitems}
\phantomsection\label{\detokenize{cascade.exoplanet_tools:cascade.exoplanet_tools.exoplanet_tools.transit_to_eclipse}}\pysiglinewithargsret{\sphinxbfcode{\sphinxupquote{transit\_to\_eclipse}}}{\emph{transit}}{}
Converts transit spectrum to eclipse spectrum
\begin{quote}\begin{description}
\item[{Parameters}] \leavevmode
\sphinxstyleliteralstrong{\sphinxupquote{transit}} \textendash{} Transit depth values to be converted

\item[{Returns}] \leavevmode
\sphinxstyleemphasis{eclipse} \textendash{} eclipse depth values derived from input transit values

\end{description}\end{quote}

\end{fulllineitems}

\index{combine\_spectra() (in module cascade.exoplanet\_tools.exoplanet\_tools)@\spxentry{combine\_spectra()}\spxextra{in module cascade.exoplanet\_tools.exoplanet\_tools}}

\begin{fulllineitems}
\phantomsection\label{\detokenize{cascade.exoplanet_tools:cascade.exoplanet_tools.exoplanet_tools.combine_spectra}}\pysiglinewithargsret{\sphinxbfcode{\sphinxupquote{combine\_spectra}}}{\emph{identifier\_list={[}{]}}, \emph{path=''}}{}
Convienience function to combine multiple extracted spectra
of the same source by calculating a weighted averige.
\begin{quote}\begin{description}
\item[{Parameters}] \leavevmode\begin{itemize}
\item {} 
\sphinxstyleliteralstrong{\sphinxupquote{identifier\_list}} (\sphinxstyleliteralemphasis{\sphinxupquote{'list' of 'str'}}) \textendash{} List of file identifiers of the individual spectra to be combined

\item {} 
\sphinxstyleliteralstrong{\sphinxupquote{path}} (\sphinxstyleliteralemphasis{\sphinxupquote{'str'}}) \textendash{} path to the fits files

\end{itemize}

\item[{Returns}] \leavevmode
\sphinxstylestrong{combined\_spectrum} (\sphinxstyleemphasis{‘array\_like’}) \textendash{} The combined spectrum based on the spectra specified in the input list

\end{description}\end{quote}

\end{fulllineitems}

\index{get\_calalog() (in module cascade.exoplanet\_tools.exoplanet\_tools)@\spxentry{get\_calalog()}\spxextra{in module cascade.exoplanet\_tools.exoplanet\_tools}}

\begin{fulllineitems}
\phantomsection\label{\detokenize{cascade.exoplanet_tools:cascade.exoplanet_tools.exoplanet_tools.get_calalog}}\pysiglinewithargsret{\sphinxbfcode{\sphinxupquote{get\_calalog}}}{\emph{catalog\_name}, \emph{update=True}}{}
Get exoplanet catalog data
\begin{quote}\begin{description}
\item[{Parameters}] \leavevmode\begin{itemize}
\item {} 
\sphinxstyleliteralstrong{\sphinxupquote{catalog\_name}} (\sphinxstyleliteralemphasis{\sphinxupquote{'str'}}) \textendash{} name of catalog to use

\item {} 
\sphinxstyleliteralstrong{\sphinxupquote{update}} (\sphinxstyleliteralemphasis{\sphinxupquote{'bool'}}) \textendash{} Boolian indicating if local calalog file will be updated

\end{itemize}

\item[{Returns}] \leavevmode
\sphinxstylestrong{files\_downloaded} (\sphinxstyleemphasis{‘list’ of ‘str’}) \textendash{} list of downloaded catalog files

\end{description}\end{quote}

\end{fulllineitems}

\index{parse\_database() (in module cascade.exoplanet\_tools.exoplanet\_tools)@\spxentry{parse\_database()}\spxextra{in module cascade.exoplanet\_tools.exoplanet\_tools}}

\begin{fulllineitems}
\phantomsection\label{\detokenize{cascade.exoplanet_tools:cascade.exoplanet_tools.exoplanet_tools.parse_database}}\pysiglinewithargsret{\sphinxbfcode{\sphinxupquote{parse\_database}}}{\emph{catalog\_name}, \emph{update=True}}{}
Read CSV files containing exoplanet catalog data
\begin{quote}\begin{description}
\item[{Parameters}] \leavevmode\begin{itemize}
\item {} 
\sphinxstyleliteralstrong{\sphinxupquote{catalog\_name}} (\sphinxstyleliteralemphasis{\sphinxupquote{'str'}}) \textendash{} name of catalog to use

\item {} 
\sphinxstyleliteralstrong{\sphinxupquote{update}} (\sphinxstyleliteralemphasis{\sphinxupquote{'bool'}}) \textendash{} Boolian indicating if local calalog file will be updated

\end{itemize}

\item[{Returns}] \leavevmode
\sphinxstylestrong{table\_list} (\sphinxstyleemphasis{‘list’ of ‘astropy.table.Table’}) \textendash{} List containing astropy Tables with the parameters of the exoplanet
systems in the database.

\end{description}\end{quote}

\begin{sphinxadmonition}{note}{Note:}\begin{description}
\item[{The following exoplanet databases can be used:}] \leavevmode
The Transing exoplanet catalog (TEPCAT)
The NASA exoplanet Archive
The Exoplanet Orbit Database

\end{description}
\end{sphinxadmonition}
\begin{quote}\begin{description}
\item[{Raises}] \leavevmode
\sphinxhref{https://docs.python.org/3/library/exceptions.html\#ValueError}{\sphinxcode{\sphinxupquote{ValueError}}} \textendash{} Raises error if the input catalog is nor recognized

\end{description}\end{quote}

\end{fulllineitems}

\index{extract\_exoplanet\_data() (in module cascade.exoplanet\_tools.exoplanet\_tools)@\spxentry{extract\_exoplanet\_data()}\spxextra{in module cascade.exoplanet\_tools.exoplanet\_tools}}

\begin{fulllineitems}
\phantomsection\label{\detokenize{cascade.exoplanet_tools:cascade.exoplanet_tools.exoplanet_tools.extract_exoplanet_data}}\pysiglinewithargsret{\sphinxbfcode{\sphinxupquote{extract\_exoplanet\_data}}}{\emph{data\_list}, \emph{target\_name\_or\_position}, \emph{coord\_unit=None}, \emph{coordinate\_frame='icrs'}, \emph{search\_radius=\textless{}Quantity 5. arcsec\textgreater{}}}{}
Extract the data record for a single target
\begin{quote}\begin{description}
\item[{Parameters}] \leavevmode\begin{itemize}
\item {} 
\sphinxstyleliteralstrong{\sphinxupquote{data\_list}} (\sphinxstyleliteralemphasis{\sphinxupquote{'list' of 'astropy.Table'}}) \textendash{} List containing table with exoplanet data

\item {} 
\sphinxstyleliteralstrong{\sphinxupquote{target\_name\_or\_position}} \textendash{} Name of the target or coordinates of the target for
which the record is extracted

\item {} 
\sphinxstyleliteralstrong{\sphinxupquote{coord\_unit}} \textendash{} Unit of coordinates e.g (u.hourangle, u.deg)

\item {} 
\sphinxstyleliteralstrong{\sphinxupquote{coordinate\_frame}} \textendash{} Frame of coordinate system e.g icrs

\end{itemize}

\item[{Returns}] \leavevmode
\sphinxstylestrong{table\_list} (\sphinxstyleemphasis{‘list’}) \textendash{} List containing data record of the specified planet

\end{description}\end{quote}

\begin{sphinxadmonition}{note}{Examples}

Download the Exoplanet Orbit Database:

\fvset{hllines={, ,}}%
\begin{sphinxVerbatim}[commandchars=\\\{\}]
\PYG{g+gp}{\PYGZgt{}\PYGZgt{}\PYGZgt{} }\PYG{k+kn}{import} \PYG{n+nn}{cascade}
\PYG{g+gp}{\PYGZgt{}\PYGZgt{}\PYGZgt{} }\PYG{n}{ct} \PYG{o}{=} \PYG{n}{cascade}\PYG{o}{.}\PYG{n}{exoplanet\PYGZus{}tools}\PYG{o}{.}\PYG{n}{parse\PYGZus{}database}\PYG{p}{(}\PYG{l+s+s1}{\PYGZsq{}}\PYG{l+s+s1}{EXOPLANETS.ORG}\PYG{l+s+s1}{\PYGZsq{}}\PYG{p}{,}
\PYG{g+go}{                                                update=True)}
\end{sphinxVerbatim}

Extract data record for single system:

\fvset{hllines={, ,}}%
\begin{sphinxVerbatim}[commandchars=\\\{\}]
\PYG{g+gp}{\PYGZgt{}\PYGZgt{}\PYGZgt{} }\PYG{n}{dr} \PYG{o}{=} \PYG{n}{cascade}\PYG{o}{.}\PYG{n}{exoplanet\PYGZus{}tools}\PYG{o}{.}\PYG{n}{extract\PYGZus{}exoplanet\PYGZus{}data}\PYG{p}{(}\PYG{n}{ct}\PYG{p}{,} \PYG{l+s+s1}{\PYGZsq{}}\PYG{l+s+s1}{HD 189733 b}\PYG{l+s+s1}{\PYGZsq{}}\PYG{p}{)}
\PYG{g+gp}{\PYGZgt{}\PYGZgt{}\PYGZgt{} }\PYG{n+nb}{print}\PYG{p}{(}\PYG{n}{dr}\PYG{p}{[}\PYG{l+m+mi}{0}\PYG{p}{]}\PYG{p}{)}
\end{sphinxVerbatim}
\end{sphinxadmonition}

\end{fulllineitems}

\index{lightcuve (class in cascade.exoplanet\_tools.exoplanet\_tools)@\spxentry{lightcuve}\spxextra{class in cascade.exoplanet\_tools.exoplanet\_tools}}

\begin{fulllineitems}
\phantomsection\label{\detokenize{cascade.exoplanet_tools:cascade.exoplanet_tools.exoplanet_tools.lightcuve}}\pysigline{\sphinxbfcode{\sphinxupquote{class }}\sphinxbfcode{\sphinxupquote{lightcuve}}}
Bases: \sphinxhref{https://docs.python.org/3/library/functions.html\#object}{\sphinxcode{\sphinxupquote{object}}}

Class defining lightcurve model used to model the observed
transit/eclipse observations.
Current valid lightcurve models: batman
\begin{quote}\begin{description}
\item[{Variables}] \leavevmode\begin{itemize}
\item {} 
\sphinxstyleliteralstrong{\sphinxupquote{lc}} (\sphinxstyleliteralemphasis{\sphinxupquote{'array\_like'}}) \textendash{} The lightcurve model

\item {} 
\sphinxstyleliteralstrong{\sphinxupquote{par}} (\sphinxstyleliteralemphasis{\sphinxupquote{'ordered\_dict'}}) \textendash{} The lightcurve parameters

\end{itemize}

\end{description}\end{quote}

\begin{sphinxadmonition}{note}{Notes}

Uses factory method to pick model/package used to calculate
the lightcurve model.
\end{sphinxadmonition}
\begin{quote}\begin{description}
\item[{Raises}] \leavevmode
\sphinxhref{https://docs.python.org/3/library/exceptions.html\#ValueError}{\sphinxcode{\sphinxupquote{ValueError}}} \textendash{} Error is raised if no valid lightcurve model is defined

\end{description}\end{quote}

\begin{sphinxadmonition}{note}{Examples}

To test  the generation of a ligthcurve model
first generate standard .ini file and initialize cascade

\fvset{hllines={, ,}}%
\begin{sphinxVerbatim}[commandchars=\\\{\}]
\PYG{g+gp}{\PYGZgt{}\PYGZgt{}\PYGZgt{} }\PYG{k+kn}{import} \PYG{n+nn}{cascade}
\PYG{g+gp}{\PYGZgt{}\PYGZgt{}\PYGZgt{} }\PYG{n}{cascade}\PYG{o}{.}\PYG{n}{initialize}\PYG{o}{.}\PYG{n}{generate\PYGZus{}default\PYGZus{}initialization}\PYG{p}{(}\PYG{p}{)}
\PYG{g+gp}{\PYGZgt{}\PYGZgt{}\PYGZgt{} }\PYG{n}{path} \PYG{o}{=} \PYG{n}{cascade}\PYG{o}{.}\PYG{n}{initialize}\PYG{o}{.}\PYG{n}{default\PYGZus{}initialization\PYGZus{}path}
\PYG{g+gp}{\PYGZgt{}\PYGZgt{}\PYGZgt{} }\PYG{n}{cascade\PYGZus{}param} \PYG{o}{=} \PYG{n}{cascade}\PYG{o}{.}\PYG{n}{initialize}\PYG{o}{.}\PYG{n}{configurator}\PYG{p}{(}\PYG{n}{path}\PYG{o}{+}\PYG{l+s+s2}{\PYGZdq{}}\PYG{l+s+s2}{cascade\PYGZus{}default.ini}\PYG{l+s+s2}{\PYGZdq{}}\PYG{p}{)}
\end{sphinxVerbatim}

Define  the ligthcurve model specified in the .ini file

\fvset{hllines={, ,}}%
\begin{sphinxVerbatim}[commandchars=\\\{\}]
\PYG{g+gp}{\PYGZgt{}\PYGZgt{}\PYGZgt{} }\PYG{n}{lc\PYGZus{}model} \PYG{o}{=} \PYG{n}{cascade}\PYG{o}{.}\PYG{n}{exoplanet\PYGZus{}tools}\PYG{o}{.}\PYG{n}{lightcuve}\PYG{p}{(}\PYG{p}{)}
\PYG{g+gp}{\PYGZgt{}\PYGZgt{}\PYGZgt{} }\PYG{n+nb}{print}\PYG{p}{(}\PYG{n}{lc\PYGZus{}model}\PYG{o}{.}\PYG{n}{valid\PYGZus{}models}\PYG{p}{)}
\PYG{g+gp}{\PYGZgt{}\PYGZgt{}\PYGZgt{} }\PYG{n+nb}{print}\PYG{p}{(}\PYG{n}{lc\PYGZus{}model}\PYG{o}{.}\PYG{n}{par}\PYG{p}{)}
\end{sphinxVerbatim}

Plot the normized lightcurve

\fvset{hllines={, ,}}%
\begin{sphinxVerbatim}[commandchars=\\\{\}]
\PYG{g+gp}{\PYGZgt{}\PYGZgt{}\PYGZgt{} }\PYG{n}{fig}\PYG{p}{,} \PYG{n}{axs} \PYG{o}{=} \PYG{n}{plt}\PYG{o}{.}\PYG{n}{subplots}\PYG{p}{(}\PYG{l+m+mi}{1}\PYG{p}{,} \PYG{l+m+mi}{1}\PYG{p}{,} \PYG{n}{figsize}\PYG{o}{=}\PYG{p}{(}\PYG{l+m+mi}{12}\PYG{p}{,} \PYG{l+m+mi}{10}\PYG{p}{)}\PYG{p}{)}
\PYG{g+gp}{\PYGZgt{}\PYGZgt{}\PYGZgt{} }\PYG{n}{axs}\PYG{o}{.}\PYG{n}{plot}\PYG{p}{(}\PYG{n}{lc\PYGZus{}model}\PYG{o}{.}\PYG{n}{lc}\PYG{p}{[}\PYG{l+m+mi}{0}\PYG{p}{]}\PYG{p}{,} \PYG{n}{lc\PYGZus{}model}\PYG{o}{.}\PYG{n}{lc}\PYG{p}{[}\PYG{l+m+mi}{1}\PYG{p}{]}\PYG{p}{)}
\PYG{g+gp}{\PYGZgt{}\PYGZgt{}\PYGZgt{} }\PYG{n}{axs}\PYG{o}{.}\PYG{n}{set\PYGZus{}ylabel}\PYG{p}{(}\PYG{l+s+sa}{r}\PYG{l+s+s1}{\PYGZsq{}}\PYG{l+s+s1}{Normalized Signal}\PYG{l+s+s1}{\PYGZsq{}}\PYG{p}{)}
\PYG{g+gp}{\PYGZgt{}\PYGZgt{}\PYGZgt{} }\PYG{n}{axs}\PYG{o}{.}\PYG{n}{set\PYGZus{}xlabel}\PYG{p}{(}\PYG{l+s+sa}{r}\PYG{l+s+s1}{\PYGZsq{}}\PYG{l+s+s1}{Phase}\PYG{l+s+s1}{\PYGZsq{}}\PYG{p}{)}
\PYG{g+gp}{\PYGZgt{}\PYGZgt{}\PYGZgt{} }\PYG{n}{axes} \PYG{o}{=} \PYG{n}{plt}\PYG{o}{.}\PYG{n}{gca}\PYG{p}{(}\PYG{p}{)}
\PYG{g+gp}{\PYGZgt{}\PYGZgt{}\PYGZgt{} }\PYG{n}{axes}\PYG{o}{.}\PYG{n}{set\PYGZus{}xlim}\PYG{p}{(}\PYG{p}{[}\PYG{l+m+mi}{0}\PYG{p}{,} \PYG{l+m+mi}{1}\PYG{p}{]}\PYG{p}{)}
\PYG{g+gp}{\PYGZgt{}\PYGZgt{}\PYGZgt{} }\PYG{n}{axes}\PYG{o}{.}\PYG{n}{set\PYGZus{}ylim}\PYG{p}{(}\PYG{p}{[}\PYG{o}{\PYGZhy{}}\PYG{l+m+mf}{1.1}\PYG{p}{,} \PYG{l+m+mf}{0.1}\PYG{p}{]}\PYG{p}{)}
\PYG{g+gp}{\PYGZgt{}\PYGZgt{}\PYGZgt{} }\PYG{n}{plt}\PYG{o}{.}\PYG{n}{show}\PYG{p}{(}\PYG{p}{)}
\end{sphinxVerbatim}
\end{sphinxadmonition}
\index{valid\_models (lightcuve attribute)@\spxentry{valid\_models}\spxextra{lightcuve attribute}}

\begin{fulllineitems}
\phantomsection\label{\detokenize{cascade.exoplanet_tools:cascade.exoplanet_tools.exoplanet_tools.lightcuve.valid_models}}\pysigline{\sphinxbfcode{\sphinxupquote{valid\_models}}\sphinxbfcode{\sphinxupquote{ = \{'batman'\}}}}
\end{fulllineitems}


\end{fulllineitems}

\index{batman\_model (class in cascade.exoplanet\_tools.exoplanet\_tools)@\spxentry{batman\_model}\spxextra{class in cascade.exoplanet\_tools.exoplanet\_tools}}

\begin{fulllineitems}
\phantomsection\label{\detokenize{cascade.exoplanet_tools:cascade.exoplanet_tools.exoplanet_tools.batman_model}}\pysigline{\sphinxbfcode{\sphinxupquote{class }}\sphinxbfcode{\sphinxupquote{batman\_model}}}
Bases: \sphinxhref{https://docs.python.org/3/library/functions.html\#object}{\sphinxcode{\sphinxupquote{object}}}

This class defines the lightcurve model used to analyse the observed
transit/eclipse using the batman package by Laura Kreidberg %
\begin{footnote}[1]\sphinxAtStartFootnote
Kreidberg, L. 2015, PASP 127, 1161
%
\end{footnote}.
\begin{quote}\begin{description}
\item[{Variables}] \leavevmode\begin{itemize}
\item {} 
\sphinxstyleliteralstrong{\sphinxupquote{lc}} (\sphinxstyleliteralemphasis{\sphinxupquote{'array\_like'}}) \textendash{} The values of the lightcurve model

\item {} 
\sphinxstyleliteralstrong{\sphinxupquote{par}} (\sphinxstyleliteralemphasis{\sphinxupquote{'ordered\_dict'}}) \textendash{} The model parameters difining the lightcurve model

\end{itemize}

\end{description}\end{quote}

\begin{sphinxadmonition}{note}{References}
\end{sphinxadmonition}
\index{define\_batman\_model() (batman\_model static method)@\spxentry{define\_batman\_model()}\spxextra{batman\_model static method}}

\begin{fulllineitems}
\phantomsection\label{\detokenize{cascade.exoplanet_tools:cascade.exoplanet_tools.exoplanet_tools.batman_model.define_batman_model}}\pysiglinewithargsret{\sphinxbfcode{\sphinxupquote{static }}\sphinxbfcode{\sphinxupquote{define\_batman\_model}}}{\emph{InputParameter}}{}
This function defines the light curve model used to analize the
transit or eclipse. We use the batman package to calculate the
light curves.We assume here a symmetric transit signal, that the
secondary transit is at phase 0.5 and primary transit at 0.0.
\begin{quote}\begin{description}
\item[{Parameters}] \leavevmode
\sphinxstyleliteralstrong{\sphinxupquote{InputParameter}} (\sphinxstyleliteralemphasis{\sphinxupquote{'dict'}}) \textendash{} Ordered dict containing all needed inut parameter to define model

\item[{Returns}] \leavevmode
\begin{itemize}
\item {} 
\sphinxstylestrong{tmodel} (\sphinxstyleemphasis{‘array\_like’}) \textendash{} Orbital phase of planet used for lightcurve model

\item {} 
\sphinxstylestrong{lcmode} (\sphinxstyleemphasis{‘array\_like’}) \textendash{} Normalized values of the lightcurve model

\end{itemize}


\end{description}\end{quote}

\end{fulllineitems}

\index{ReturnParFromIni() (batman\_model method)@\spxentry{ReturnParFromIni()}\spxextra{batman\_model method}}

\begin{fulllineitems}
\phantomsection\label{\detokenize{cascade.exoplanet_tools:cascade.exoplanet_tools.exoplanet_tools.batman_model.ReturnParFromIni}}\pysiglinewithargsret{\sphinxbfcode{\sphinxupquote{ReturnParFromIni}}}{}{}
Get relevant parameters for lightcurve model from CASCADe
intitialization files
\begin{quote}\begin{description}
\item[{Returns}] \leavevmode
\sphinxstylestrong{par} (\sphinxstyleemphasis{‘ordered\_dict’}) \textendash{} input model parameters for batman lightcurve model

\end{description}\end{quote}

\end{fulllineitems}

\index{ReturnParFromDB() (batman\_model method)@\spxentry{ReturnParFromDB()}\spxextra{batman\_model method}}

\begin{fulllineitems}
\phantomsection\label{\detokenize{cascade.exoplanet_tools:cascade.exoplanet_tools.exoplanet_tools.batman_model.ReturnParFromDB}}\pysiglinewithargsret{\sphinxbfcode{\sphinxupquote{ReturnParFromDB}}}{}{}
Get relevant parameters for lightcurve model from exoplanet database
specified in CASCADe initialization file
\begin{quote}\begin{description}
\item[{Returns}] \leavevmode
\sphinxstylestrong{par} (\sphinxstyleemphasis{‘ordered\_dict’}) \textendash{} input model parameters for batman lightcurve model

\item[{Raises}] \leavevmode
\sphinxhref{https://docs.python.org/3/library/exceptions.html\#ValueError}{\sphinxcode{\sphinxupquote{ValueError}}} \textendash{} Raises error in case the observation type is not recognized.

\end{description}\end{quote}

\end{fulllineitems}


\end{fulllineitems}



\subsection{The cascade.initialize module}
\label{\detokenize{cascade.initialize:module-cascade.initialize.initialize}}\label{\detokenize{cascade.initialize:the-cascade-initialize-module}}\label{\detokenize{cascade.initialize::doc}}\index{cascade.initialize.initialize (module)@\spxentry{cascade.initialize.initialize}\spxextra{module}}
This Module defines the functionality to generate and read .ini files which
are used to initialize CASCADe.

\begin{sphinxadmonition}{note}{Examples}

An example how the initilize module is used:

\fvset{hllines={, ,}}%
\begin{sphinxVerbatim}[commandchars=\\\{\}]
\PYG{g+gp}{\PYGZgt{}\PYGZgt{}\PYGZgt{} }\PYG{k+kn}{import} \PYG{n+nn}{cascade}
\PYG{g+gp}{\PYGZgt{}\PYGZgt{}\PYGZgt{} }\PYG{n}{default\PYGZus{}path} \PYG{o}{=} \PYG{n}{cascade}\PYG{o}{.}\PYG{n}{initialize}\PYG{o}{.}\PYG{n}{default\PYGZus{}initialization\PYGZus{}path}
\PYG{g+gp}{\PYGZgt{}\PYGZgt{}\PYGZgt{} }\PYG{n}{success} \PYG{o}{=} \PYG{n}{cascade}\PYG{o}{.}\PYG{n}{initialize}\PYG{o}{.}\PYG{n}{generate\PYGZus{}default\PYGZus{}initialization}\PYG{p}{(}\PYG{p}{)}
\end{sphinxVerbatim}

\fvset{hllines={, ,}}%
\begin{sphinxVerbatim}[commandchars=\\\{\}]
\PYG{g+gp}{\PYGZgt{}\PYGZgt{}\PYGZgt{} }\PYG{n}{tso} \PYG{o}{=} \PYG{n}{cascade}\PYG{o}{.}\PYG{n}{TSO}\PYG{o}{.}\PYG{n}{TSOSuite}\PYG{p}{(}\PYG{p}{)}
\PYG{g+gp}{\PYGZgt{}\PYGZgt{}\PYGZgt{} }\PYG{n+nb}{print}\PYG{p}{(}\PYG{n}{cascade}\PYG{o}{.}\PYG{n}{initialize}\PYG{o}{.}\PYG{n}{cascade\PYGZus{}configuration}\PYG{o}{.}\PYG{n}{isInitialized}\PYG{p}{)}
\PYG{g+gp}{\PYGZgt{}\PYGZgt{}\PYGZgt{} }\PYG{n+nb}{print}\PYG{p}{(}\PYG{n}{tso}\PYG{o}{.}\PYG{n}{cascade\PYGZus{}parameters}\PYG{o}{.}\PYG{n}{isInitialized}\PYG{p}{)}
\PYG{g+gp}{\PYGZgt{}\PYGZgt{}\PYGZgt{} }\PYG{k}{assert} \PYG{n}{tso}\PYG{o}{.}\PYG{n}{cascade\PYGZus{}parameters} \PYG{o}{==} \PYG{n}{cascade}\PYG{o}{.}\PYG{n}{initialize}\PYG{o}{.}\PYG{n}{cascade\PYGZus{}configuration}
\end{sphinxVerbatim}

\fvset{hllines={, ,}}%
\begin{sphinxVerbatim}[commandchars=\\\{\}]
\PYG{g+gp}{\PYGZgt{}\PYGZgt{}\PYGZgt{} }\PYG{n}{tso}\PYG{o}{.}\PYG{n}{execute}\PYG{p}{(}\PYG{l+s+s1}{\PYGZsq{}}\PYG{l+s+s1}{initialize}\PYG{l+s+s1}{\PYGZsq{}}\PYG{p}{,} \PYG{l+s+s1}{\PYGZsq{}}\PYG{l+s+s1}{cascade\PYGZus{}default.ini}\PYG{l+s+s1}{\PYGZsq{}}\PYG{p}{,} \PYG{n}{path}\PYG{o}{=}\PYG{n}{default\PYGZus{}path}\PYG{p}{)}
\PYG{g+gp}{\PYGZgt{}\PYGZgt{}\PYGZgt{} }\PYG{n+nb}{print}\PYG{p}{(}\PYG{n}{cascade}\PYG{o}{.}\PYG{n}{initialize}\PYG{o}{.}\PYG{n}{cascade\PYGZus{}configuration}\PYG{o}{.}\PYG{n}{isInitialized}\PYG{p}{)}
\PYG{g+gp}{\PYGZgt{}\PYGZgt{}\PYGZgt{} }\PYG{n+nb}{print}\PYG{p}{(}\PYG{n}{tso}\PYG{o}{.}\PYG{n}{cascade\PYGZus{}parameters}\PYG{o}{.}\PYG{n}{isInitialized}\PYG{p}{)}
\end{sphinxVerbatim}

\fvset{hllines={, ,}}%
\begin{sphinxVerbatim}[commandchars=\\\{\}]
\PYG{g+gp}{\PYGZgt{}\PYGZgt{}\PYGZgt{} }\PYG{n}{tso}\PYG{o}{.}\PYG{n}{execute}\PYG{p}{(}\PYG{l+s+s2}{\PYGZdq{}}\PYG{l+s+s2}{reset}\PYG{l+s+s2}{\PYGZdq{}}\PYG{p}{)}
\PYG{g+gp}{\PYGZgt{}\PYGZgt{}\PYGZgt{} }\PYG{n+nb}{print}\PYG{p}{(}\PYG{n}{cascade}\PYG{o}{.}\PYG{n}{initialize}\PYG{o}{.}\PYG{n}{cascade\PYGZus{}configuration}\PYG{o}{.}\PYG{n}{isInitialized}\PYG{p}{)}
\PYG{g+gp}{\PYGZgt{}\PYGZgt{}\PYGZgt{} }\PYG{n+nb}{print}\PYG{p}{(}\PYG{n}{tso}\PYG{o}{.}\PYG{n}{cascade\PYGZus{}parameters}\PYG{o}{.}\PYG{n}{isInitialized}\PYG{p}{)}
\end{sphinxVerbatim}
\end{sphinxadmonition}
\index{default\_initialization\_path (in module cascade.initialize.initialize)@\spxentry{default\_initialization\_path}\spxextra{in module cascade.initialize.initialize}}

\begin{fulllineitems}
\phantomsection\label{\detokenize{cascade.initialize:cascade.initialize.initialize.default_initialization_path}}\pysigline{\sphinxbfcode{\sphinxupquote{default\_initialization\_path}}\sphinxbfcode{\sphinxupquote{ = '/home/bouwman/CASCADeInit/'}}}
Default directory for CASCADe initialization files

\end{fulllineitems}

\index{generate\_default\_initialization() (in module cascade.initialize.initialize)@\spxentry{generate\_default\_initialization()}\spxextra{in module cascade.initialize.initialize}}

\begin{fulllineitems}
\phantomsection\label{\detokenize{cascade.initialize:cascade.initialize.initialize.generate_default_initialization}}\pysiglinewithargsret{\sphinxbfcode{\sphinxupquote{generate\_default\_initialization}}}{}{}
Convenience function to generate an example .ini file for CASCADe
initialization. The file will be saved in the default directory defined by
default\_initialization\_path. Returns True if successfully runned.

\end{fulllineitems}

\index{configurator (class in cascade.initialize.initialize)@\spxentry{configurator}\spxextra{class in cascade.initialize.initialize}}

\begin{fulllineitems}
\phantomsection\label{\detokenize{cascade.initialize:cascade.initialize.initialize.configurator}}\pysiglinewithargsret{\sphinxbfcode{\sphinxupquote{class }}\sphinxbfcode{\sphinxupquote{configurator}}}{\emph{*file\_names}}{}
Bases: \sphinxhref{https://docs.python.org/3/library/functions.html\#object}{\sphinxcode{\sphinxupquote{object}}}

This class defined the configuration singleton which will provide
all parameters needed to run the CASCADe to all modules of the code.
\index{isInitialized (configurator attribute)@\spxentry{isInitialized}\spxextra{configurator attribute}}

\begin{fulllineitems}
\phantomsection\label{\detokenize{cascade.initialize:cascade.initialize.initialize.configurator.isInitialized}}\pysigline{\sphinxbfcode{\sphinxupquote{isInitialized}}\sphinxbfcode{\sphinxupquote{ = False}}}
Will be set to True if initialized

\end{fulllineitems}

\index{reset() (configurator method)@\spxentry{reset()}\spxextra{configurator method}}

\begin{fulllineitems}
\phantomsection\label{\detokenize{cascade.initialize:cascade.initialize.initialize.configurator.reset}}\pysiglinewithargsret{\sphinxbfcode{\sphinxupquote{reset}}}{}{}
If called, this function will remove all initialized parameters.

\end{fulllineitems}


\end{fulllineitems}

\index{cascade\_configuration (in module cascade.initialize.initialize)@\spxentry{cascade\_configuration}\spxextra{in module cascade.initialize.initialize}}

\begin{fulllineitems}
\phantomsection\label{\detokenize{cascade.initialize:cascade.initialize.initialize.cascade_configuration}}\pysigline{\sphinxbfcode{\sphinxupquote{cascade\_configuration}}\sphinxbfcode{\sphinxupquote{ = \textless{}cascade.initialize.initialize.configurator object\textgreater{}}}}
Singleton containing the entite configuration settings for the
CASCADe code to work. This includes object and observation definitions and
causal noise model settings.

\end{fulllineitems}



\subsection{The cascade.instruments module}
\label{\detokenize{cascade.instruments:module-cascade.instruments.instruments}}\label{\detokenize{cascade.instruments:the-cascade-instruments-module}}\label{\detokenize{cascade.instruments::doc}}\index{cascade.instruments.instruments (module)@\spxentry{cascade.instruments.instruments}\spxextra{module}}
Observatory and Instruments specific module of the CASCADe package
\index{Observation (class in cascade.instruments.instruments)@\spxentry{Observation}\spxextra{class in cascade.instruments.instruments}}

\begin{fulllineitems}
\phantomsection\label{\detokenize{cascade.instruments:cascade.instruments.instruments.Observation}}\pysigline{\sphinxbfcode{\sphinxupquote{class }}\sphinxbfcode{\sphinxupquote{Observation}}}
Bases: \sphinxhref{https://docs.python.org/3/library/functions.html\#object}{\sphinxcode{\sphinxupquote{object}}}

This class handles the selection of the correct observatory and
instrument classes and loads the time series data to be analyzed
The observations specific parameters set during the initialization
of the TSO object are used to select the observatory and instrument
through a factory method and to load the specified observations
into the instance of the TSO object.

\begin{sphinxadmonition}{note}{Examples}

The Observation calss is  called during the following command:

\fvset{hllines={, ,}}%
\begin{sphinxVerbatim}[commandchars=\\\{\}]
\PYG{g+gp}{\PYGZgt{}\PYGZgt{}\PYGZgt{} }\PYG{n}{tso}\PYG{o}{.}\PYG{n}{execute}\PYG{p}{(}\PYG{l+s+s2}{\PYGZdq{}}\PYG{l+s+s2}{load\PYGZus{}data}\PYG{l+s+s2}{\PYGZdq{}}\PYG{p}{)}
\end{sphinxVerbatim}
\end{sphinxadmonition}
\index{\_Observation\_\_check\_observation\_type() (Observation method)@\spxentry{\_Observation\_\_check\_observation\_type()}\spxextra{Observation method}}

\begin{fulllineitems}
\phantomsection\label{\detokenize{cascade.instruments:cascade.instruments.instruments.Observation._Observation__check_observation_type}}\pysiglinewithargsret{\sphinxbfcode{\sphinxupquote{\_Observation\_\_check\_observation\_type}}}{}{}
Function to check of the in the .ini specified observation type
valid.
\begin{quote}\begin{description}
\item[{Raises}] \leavevmode
\sphinxhref{https://docs.python.org/3/library/exceptions.html\#ValueError}{\sphinxcode{\sphinxupquote{ValueError}}} \textendash{} Raises error if the specified observation type is not valid or if
the tso instance is not initialized.

\end{description}\end{quote}

\end{fulllineitems}

\index{\_Observation\_\_do\_observations() (Observation method)@\spxentry{\_Observation\_\_do\_observations()}\spxextra{Observation method}}

\begin{fulllineitems}
\phantomsection\label{\detokenize{cascade.instruments:cascade.instruments.instruments.Observation._Observation__do_observations}}\pysiglinewithargsret{\sphinxbfcode{\sphinxupquote{\_Observation\_\_do\_observations}}}{\emph{observatory}}{}
Factory method to load the needed observatory class and methods

\end{fulllineitems}

\index{\_Observation\_\_get\_observatory\_name() (Observation method)@\spxentry{\_Observation\_\_get\_observatory\_name()}\spxextra{Observation method}}

\begin{fulllineitems}
\phantomsection\label{\detokenize{cascade.instruments:cascade.instruments.instruments.Observation._Observation__get_observatory_name}}\pysiglinewithargsret{\sphinxbfcode{\sphinxupquote{\_Observation\_\_get\_observatory\_name}}}{}{}
Function to load the in the .ini files specified observatory name
\begin{quote}\begin{description}
\item[{Returns}] \leavevmode
\sphinxstyleemphasis{ValueError} \textendash{} Returns error if the observatory is not specified or recognized

\end{description}\end{quote}

\end{fulllineitems}

\index{\_Observation\_\_valid\_observation\_type (Observation attribute)@\spxentry{\_Observation\_\_valid\_observation\_type}\spxextra{Observation attribute}}

\begin{fulllineitems}
\phantomsection\label{\detokenize{cascade.instruments:cascade.instruments.instruments.Observation._Observation__valid_observation_type}}\pysigline{\sphinxbfcode{\sphinxupquote{\_Observation\_\_valid\_observation\_type}}}
Set listing the current implemented observation types

\end{fulllineitems}

\index{\_Observation\_\_valid\_observatories (Observation attribute)@\spxentry{\_Observation\_\_valid\_observatories}\spxextra{Observation attribute}}

\begin{fulllineitems}
\phantomsection\label{\detokenize{cascade.instruments:cascade.instruments.instruments.Observation._Observation__valid_observatories}}\pysigline{\sphinxbfcode{\sphinxupquote{\_Observation\_\_valid\_observatories}}}
Dictionary listing the current implemented observatories,
used in factory method to select a observatory specific class

\end{fulllineitems}


\end{fulllineitems}

\index{ObservatoryBase (class in cascade.instruments.instruments)@\spxentry{ObservatoryBase}\spxextra{class in cascade.instruments.instruments}}

\begin{fulllineitems}
\phantomsection\label{\detokenize{cascade.instruments:cascade.instruments.instruments.ObservatoryBase}}\pysigline{\sphinxbfcode{\sphinxupquote{class }}\sphinxbfcode{\sphinxupquote{ObservatoryBase}}}
Bases: \sphinxhref{https://docs.python.org/3/library/functions.html\#object}{\sphinxcode{\sphinxupquote{object}}}

Observatory base class used to define the basic properties an observatory
class should have
\index{name (ObservatoryBase attribute)@\spxentry{name}\spxextra{ObservatoryBase attribute}}

\begin{fulllineitems}
\phantomsection\label{\detokenize{cascade.instruments:cascade.instruments.instruments.ObservatoryBase.name}}\pysigline{\sphinxbfcode{\sphinxupquote{name}}}
Name of the observatory.

\end{fulllineitems}

\index{location (ObservatoryBase attribute)@\spxentry{location}\spxextra{ObservatoryBase attribute}}

\begin{fulllineitems}
\phantomsection\label{\detokenize{cascade.instruments:cascade.instruments.instruments.ObservatoryBase.location}}\pysigline{\sphinxbfcode{\sphinxupquote{location}}}
Location of the observatory

\end{fulllineitems}

\index{NAIF\_ID (ObservatoryBase attribute)@\spxentry{NAIF\_ID}\spxextra{ObservatoryBase attribute}}

\begin{fulllineitems}
\phantomsection\label{\detokenize{cascade.instruments:cascade.instruments.instruments.ObservatoryBase.NAIF_ID}}\pysigline{\sphinxbfcode{\sphinxupquote{NAIF\_ID}}}
NAIF ID of the observatory. With this the location relative to the
sun and the observed target as a function of time can be determined.
Needed to calculate BJD time.

\end{fulllineitems}

\index{observatory\_instruments (ObservatoryBase attribute)@\spxentry{observatory\_instruments}\spxextra{ObservatoryBase attribute}}

\begin{fulllineitems}
\phantomsection\label{\detokenize{cascade.instruments:cascade.instruments.instruments.ObservatoryBase.observatory_instruments}}\pysigline{\sphinxbfcode{\sphinxupquote{observatory\_instruments}}}
The names of the instruments part of the observatory.

\end{fulllineitems}


\end{fulllineitems}

\index{InstrumentBase (class in cascade.instruments.instruments)@\spxentry{InstrumentBase}\spxextra{class in cascade.instruments.instruments}}

\begin{fulllineitems}
\phantomsection\label{\detokenize{cascade.instruments:cascade.instruments.instruments.InstrumentBase}}\pysigline{\sphinxbfcode{\sphinxupquote{class }}\sphinxbfcode{\sphinxupquote{InstrumentBase}}}
Bases: \sphinxhref{https://docs.python.org/3/library/functions.html\#object}{\sphinxcode{\sphinxupquote{object}}}

Instrument base class used to define the basic properties an instrument
class should have
\index{load\_data() (InstrumentBase method)@\spxentry{load\_data()}\spxextra{InstrumentBase method}}

\begin{fulllineitems}
\phantomsection\label{\detokenize{cascade.instruments:cascade.instruments.instruments.InstrumentBase.load_data}}\pysiglinewithargsret{\sphinxbfcode{\sphinxupquote{load\_data}}}{}{}
Method which allows to load data.

\end{fulllineitems}

\index{get\_instrument\_setup() (InstrumentBase method)@\spxentry{get\_instrument\_setup()}\spxextra{InstrumentBase method}}

\begin{fulllineitems}
\phantomsection\label{\detokenize{cascade.instruments:cascade.instruments.instruments.InstrumentBase.get_instrument_setup}}\pysiglinewithargsret{\sphinxbfcode{\sphinxupquote{get\_instrument\_setup}}}{}{}
Method which gets the specific setup of the used instrument.

\end{fulllineitems}

\index{name (InstrumentBase attribute)@\spxentry{name}\spxextra{InstrumentBase attribute}}

\begin{fulllineitems}
\phantomsection\label{\detokenize{cascade.instruments:cascade.instruments.instruments.InstrumentBase.name}}\pysigline{\sphinxbfcode{\sphinxupquote{name}}}
Name of the instrument.

\end{fulllineitems}


\end{fulllineitems}

\index{HST (class in cascade.instruments.instruments)@\spxentry{HST}\spxextra{class in cascade.instruments.instruments}}

\begin{fulllineitems}
\phantomsection\label{\detokenize{cascade.instruments:cascade.instruments.instruments.HST}}\pysigline{\sphinxbfcode{\sphinxupquote{class }}\sphinxbfcode{\sphinxupquote{HST}}}
Bases: {\hyperref[\detokenize{cascade.instruments:cascade.instruments.instruments.ObservatoryBase}]{\sphinxcrossref{\sphinxcode{\sphinxupquote{cascade.instruments.instruments.ObservatoryBase}}}}}

This observatory class defines the instuments and data handling for the
spectropgraphs of the Spitzer Space telescope
\index{name (HST attribute)@\spxentry{name}\spxextra{HST attribute}}

\begin{fulllineitems}
\phantomsection\label{\detokenize{cascade.instruments:cascade.instruments.instruments.HST.name}}\pysigline{\sphinxbfcode{\sphinxupquote{name}}}
Set to ‘HST’

\end{fulllineitems}

\index{location (HST attribute)@\spxentry{location}\spxextra{HST attribute}}

\begin{fulllineitems}
\phantomsection\label{\detokenize{cascade.instruments:cascade.instruments.instruments.HST.location}}\pysigline{\sphinxbfcode{\sphinxupquote{location}}}
Set to ‘SPACE’

\end{fulllineitems}

\index{NAIF\_ID (HST attribute)@\spxentry{NAIF\_ID}\spxextra{HST attribute}}

\begin{fulllineitems}
\phantomsection\label{\detokenize{cascade.instruments:cascade.instruments.instruments.HST.NAIF_ID}}\pysigline{\sphinxbfcode{\sphinxupquote{NAIF\_ID}}}
Set to -48

\end{fulllineitems}

\index{observatory\_instruments (HST attribute)@\spxentry{observatory\_instruments}\spxextra{HST attribute}}

\begin{fulllineitems}
\phantomsection\label{\detokenize{cascade.instruments:cascade.instruments.instruments.HST.observatory_instruments}}\pysigline{\sphinxbfcode{\sphinxupquote{observatory\_instruments}}}
Returns \{‘WFC3’\}

\end{fulllineitems}


\end{fulllineitems}

\index{HSTWFC3 (class in cascade.instruments.instruments)@\spxentry{HSTWFC3}\spxextra{class in cascade.instruments.instruments}}

\begin{fulllineitems}
\phantomsection\label{\detokenize{cascade.instruments:cascade.instruments.instruments.HSTWFC3}}\pysigline{\sphinxbfcode{\sphinxupquote{class }}\sphinxbfcode{\sphinxupquote{HSTWFC3}}}
Bases: {\hyperref[\detokenize{cascade.instruments:cascade.instruments.instruments.InstrumentBase}]{\sphinxcrossref{\sphinxcode{\sphinxupquote{cascade.instruments.instruments.InstrumentBase}}}}}

This instrument class defines the properties of the WFC3 instrument of
the Hubble Space Telescope

For the instrument and observations the following valid options are
available:
\begin{itemize}
\item {} 
detector subarrays : \{‘IRSUB128’, ‘IRSUB256’, ‘IRSUB512’, ‘GRISM128’,
‘GRISM256’, ‘GRISM512’\}

\item {} 
spectroscopic filters : \{‘G141’\}

\item {} 
imaging filters :  \{‘F139M’, ‘F132N’, ‘F167N’\}

\item {} 
data type : \{‘SPECTRUM’, ‘SPECTRAL\_IMAGE’\}

\item {} 
observing strategy : \{‘STARING’\}

\item {} 
data products : \{‘SPC’, ‘flt’, ‘COE’\}

\end{itemize}
\index{name (HSTWFC3 attribute)@\spxentry{name}\spxextra{HSTWFC3 attribute}}

\begin{fulllineitems}
\phantomsection\label{\detokenize{cascade.instruments:cascade.instruments.instruments.HSTWFC3.name}}\pysigline{\sphinxbfcode{\sphinxupquote{name}}}
‘WFC3’
\begin{quote}\begin{description}
\item[{Type}] \leavevmode
Name of the HST instrument

\end{description}\end{quote}

\end{fulllineitems}

\index{load\_data() (HSTWFC3 method)@\spxentry{load\_data()}\spxextra{HSTWFC3 method}}

\begin{fulllineitems}
\phantomsection\label{\detokenize{cascade.instruments:cascade.instruments.instruments.HSTWFC3.load_data}}\pysiglinewithargsret{\sphinxbfcode{\sphinxupquote{load\_data}}}{}{}
This function loads the WFC3 data form disk based on the
parameters defined during the initialization of the TSO object.

\end{fulllineitems}

\index{get\_instrument\_setup() (HSTWFC3 method)@\spxentry{get\_instrument\_setup()}\spxextra{HSTWFC3 method}}

\begin{fulllineitems}
\phantomsection\label{\detokenize{cascade.instruments:cascade.instruments.instruments.HSTWFC3.get_instrument_setup}}\pysiglinewithargsret{\sphinxbfcode{\sphinxupquote{get\_instrument\_setup}}}{}{}
Retrieve all relevant parameters defining the instrument and data setup
\begin{quote}\begin{description}
\item[{Returns}] \leavevmode
\sphinxstylestrong{par} (\sphinxtitleref{collections.OrderedDict}) \textendash{} Dictionary containg all relevant parameters

\item[{Raises}] \leavevmode
\sphinxhref{https://docs.python.org/3/library/exceptions.html\#ValueError}{\sphinxcode{\sphinxupquote{ValueError}}} \textendash{} If obseervationla parameters are not or incorrect defined an
error will be raised

\end{description}\end{quote}

\end{fulllineitems}

\index{get\_spectra() (HSTWFC3 method)@\spxentry{get\_spectra()}\spxextra{HSTWFC3 method}}

\begin{fulllineitems}
\phantomsection\label{\detokenize{cascade.instruments:cascade.instruments.instruments.HSTWFC3.get_spectra}}\pysiglinewithargsret{\sphinxbfcode{\sphinxupquote{get\_spectra}}}{\emph{is\_background=False}}{}
This function combines all functionallity to read fits files
containing the (uncalibrated) spectral timeseries, including
orbital phase and wavelength information
\begin{quote}\begin{description}
\item[{Parameters}] \leavevmode
\sphinxstyleliteralstrong{\sphinxupquote{is\_background}} (\sphinxtitleref{bool}) \textendash{} if \sphinxtitleref{True} the data represents an observaton of the IR background
to be subtracted of the data of the transit spectroscopy target.

\item[{Returns}] \leavevmode
\sphinxstylestrong{SpectralTimeSeries} (\sphinxtitleref{cascade.data\_model.SpectralDataTimeSeries}) \textendash{} Instance of \sphinxtitleref{SpectralDataTimeSeries} containing all spectroscopic
data including uncertainties, time, wavelength and bad pixel mask.

\item[{Raises}] \leavevmode
\sphinxstyleemphasis{AssertionError, KeyError} \textendash{} Raises an error if no data is found or if certain expected
fits keywords are not present in the data files.

\end{description}\end{quote}

\end{fulllineitems}

\index{get\_spectral\_images() (HSTWFC3 method)@\spxentry{get\_spectral\_images()}\spxextra{HSTWFC3 method}}

\begin{fulllineitems}
\phantomsection\label{\detokenize{cascade.instruments:cascade.instruments.instruments.HSTWFC3.get_spectral_images}}\pysiglinewithargsret{\sphinxbfcode{\sphinxupquote{get\_spectral\_images}}}{\emph{is\_background=False}}{}
This function combines all functionallity to read fits files
containing the (uncalibrated) spectral image timeseries, including
orbital phase and wavelength information
\begin{quote}\begin{description}
\item[{Parameters}] \leavevmode
\sphinxstyleliteralstrong{\sphinxupquote{is\_background}} (\sphinxtitleref{bool}) \textendash{} if \sphinxtitleref{True} the data represents an observaton of the IR background
to be subtracted of the data of the transit spectroscopy target.

\item[{Returns}] \leavevmode
\sphinxstylestrong{SpectralTimeSeries} (\sphinxtitleref{cascade.data\_model.SpectralDataTimeSeries}) \textendash{} Instance of \sphinxtitleref{SpectralDataTimeSeries} containing all spectroscopic
data including uncertainties, time, wavelength and bad pixel mask.

\item[{Raises}] \leavevmode
\sphinxstyleemphasis{AssertionError, KeyError} \textendash{} Raises an error if no data is found or if certain expected
fits keywords are not present in the data files.

\end{description}\end{quote}

\end{fulllineitems}

\index{\_define\_convolution\_kernel() (HSTWFC3 method)@\spxentry{\_define\_convolution\_kernel()}\spxextra{HSTWFC3 method}}

\begin{fulllineitems}
\phantomsection\label{\detokenize{cascade.instruments:cascade.instruments.instruments.HSTWFC3._define_convolution_kernel}}\pysiglinewithargsret{\sphinxbfcode{\sphinxupquote{\_define\_convolution\_kernel}}}{}{}
Define the instrument specific convolution kernel which will be used
in the correction procedure of bad pixels

\end{fulllineitems}

\index{\_define\_region\_of\_interest() (HSTWFC3 method)@\spxentry{\_define\_region\_of\_interest()}\spxextra{HSTWFC3 method}}

\begin{fulllineitems}
\phantomsection\label{\detokenize{cascade.instruments:cascade.instruments.instruments.HSTWFC3._define_region_of_interest}}\pysiglinewithargsret{\sphinxbfcode{\sphinxupquote{\_define\_region\_of\_interest}}}{}{}
Defines region on detector which containes the intended target star.

\end{fulllineitems}

\index{\_get\_background\_cal\_data() (HSTWFC3 method)@\spxentry{\_get\_background\_cal\_data()}\spxextra{HSTWFC3 method}}

\begin{fulllineitems}
\phantomsection\label{\detokenize{cascade.instruments:cascade.instruments.instruments.HSTWFC3._get_background_cal_data}}\pysiglinewithargsret{\sphinxbfcode{\sphinxupquote{\_get\_background\_cal\_data}}}{}{}
Get the calibration data from which the background in the science
images can be determined.
\begin{quote}\begin{description}
\item[{Raises}] \leavevmode
\sphinxstyleemphasis{FileNotFoundError, AttributeError} \textendash{} An error is raised if the calibration images are not found or the
background data is not properly defined.

\end{description}\end{quote}

\begin{sphinxadmonition}{note}{Notes}

For further details see:
\begin{quote}

\sphinxurl{http://www.stsci.edu/hst/wfc3/documents/ISRs/WFC3-2015-17.pdf}
\end{quote}
\end{sphinxadmonition}

\end{fulllineitems}

\index{\_fit\_background() (HSTWFC3 method)@\spxentry{\_fit\_background()}\spxextra{HSTWFC3 method}}

\begin{fulllineitems}
\phantomsection\label{\detokenize{cascade.instruments:cascade.instruments.instruments.HSTWFC3._fit_background}}\pysiglinewithargsret{\sphinxbfcode{\sphinxupquote{\_fit\_background}}}{\emph{science\_data\_in}}{}
Determes the background in the HST Grism data using a model for the
background to the spectral timeseries data
\begin{quote}\begin{description}
\item[{Parameters}] \leavevmode
\sphinxstyleliteralstrong{\sphinxupquote{science\_data\_in}} (\sphinxtitleref{masked quantity}) \textendash{} Input data for which the background will be determined

\item[{Returns}] \leavevmode
\sphinxstylestrong{SpectralTimeSeries} (\sphinxtitleref{SpectralDataTimeSeries}) \textendash{} The fitted IR bacgound as a function of time

\end{description}\end{quote}

\begin{sphinxadmonition}{note}{Notes}

All details of the implemented model is described in:

\sphinxurl{http://www.stsci.edu/hst/wfc3/documents/ISRs/WFC3-2015-17.pdf}
\end{sphinxadmonition}

\end{fulllineitems}

\index{\_determine\_relative\_source\_position() (HSTWFC3 method)@\spxentry{\_determine\_relative\_source\_position()}\spxextra{HSTWFC3 method}}

\begin{fulllineitems}
\phantomsection\label{\detokenize{cascade.instruments:cascade.instruments.instruments.HSTWFC3._determine_relative_source_position}}\pysiglinewithargsret{\sphinxbfcode{\sphinxupquote{\_determine\_relative\_source\_position}}}{\emph{spectral\_image\_cube}, \emph{mask}}{}
Determine the shift of the spectra (source) relative to the first
integration. Note that it is important for this to work properly
to have identified bad pixels and to correct the values using an edge
preserving correction, i.e. an correction which takes into account
the dispersion direction and psf size (relative to pixel size)
\begin{quote}\begin{description}
\item[{Parameters}] \leavevmode\begin{itemize}
\item {} 
\sphinxstyleliteralstrong{\sphinxupquote{spectral\_image\_cube}} (\sphinxtitleref{ndarray}) \textendash{} Input spectral image data cube.

\item {} 
\sphinxstyleliteralstrong{\sphinxupquote{mask}} (\sphinxtitleref{ndarray} of \sphinxtitleref{int}) \textendash{} Bad pixel and region of interest mask. Values of 1 indicate
flagged data.

\end{itemize}

\item[{Variables}] \leavevmode
\sphinxstyleliteralstrong{\sphinxupquote{relative\_source\_shift}} \textendash{} relative x and y position as a function of time.

\end{description}\end{quote}

\end{fulllineitems}

\index{\_determine\_source\_position\_from\_cal\_image() (HSTWFC3 method)@\spxentry{\_determine\_source\_position\_from\_cal\_image()}\spxextra{HSTWFC3 method}}

\begin{fulllineitems}
\phantomsection\label{\detokenize{cascade.instruments:cascade.instruments.instruments.HSTWFC3._determine_source_position_from_cal_image}}\pysiglinewithargsret{\sphinxbfcode{\sphinxupquote{\_determine\_source\_position\_from\_cal\_image}}}{\emph{calibration\_image\_cube}, \emph{calibration\_data\_files}}{}
Determines the source position on the detector of the target source in
the calibration image takes prior to the spectroscopic observations.
\begin{quote}\begin{description}
\item[{Parameters}] \leavevmode\begin{itemize}
\item {} 
\sphinxstyleliteralstrong{\sphinxupquote{calibration\_image\_cube}} (\sphinxtitleref{ndarray}) \textendash{} Cube containing all acquisition images of the target.

\item {} 
\sphinxstyleliteralstrong{\sphinxupquote{calibration\_data\_files}} (\sphinxtitleref{list} of \sphinxtitleref{str}) \textendash{} List containing the file names associated with the calibraton data.

\end{itemize}

\item[{Variables}] \leavevmode
\sphinxstyleliteralstrong{\sphinxupquote{calibration\_source\_position}} (\sphinxtitleref{list’ of {}`tuple}) \textendash{} The position of the source in the acquisition images associated
with the HST spectral timeseries observations.

\end{description}\end{quote}

\end{fulllineitems}

\index{\_read\_grism\_configuration\_files() (HSTWFC3 method)@\spxentry{\_read\_grism\_configuration\_files()}\spxextra{HSTWFC3 method}}

\begin{fulllineitems}
\phantomsection\label{\detokenize{cascade.instruments:cascade.instruments.instruments.HSTWFC3._read_grism_configuration_files}}\pysiglinewithargsret{\sphinxbfcode{\sphinxupquote{\_read\_grism\_configuration\_files}}}{}{}
Gets the relevant data from WFC3 configuration files
\begin{quote}\begin{description}
\item[{Variables}] \leavevmode\begin{itemize}
\item {} 
\sphinxstyleliteralstrong{\sphinxupquote{DYDX}} (\sphinxtitleref{list}) \textendash{} The parameters for the spectral trace

\item {} 
\sphinxstyleliteralstrong{\sphinxupquote{DLDP}} (\sphinxstyleliteralemphasis{\sphinxupquote{'list{}`}}) \textendash{} The parameters for the wavelength calibration

\end{itemize}

\item[{Raises}] \leavevmode
\sphinxhref{https://docs.python.org/3/library/exceptions.html\#ValueError}{\sphinxcode{\sphinxupquote{ValueError}}} \textendash{} An error is raised if the parameters associated
with the specified instrument mode can not be found in the
calibration file.

\end{description}\end{quote}

\end{fulllineitems}

\index{\_read\_reference\_pixel\_file() (HSTWFC3 method)@\spxentry{\_read\_reference\_pixel\_file()}\spxextra{HSTWFC3 method}}

\begin{fulllineitems}
\phantomsection\label{\detokenize{cascade.instruments:cascade.instruments.instruments.HSTWFC3._read_reference_pixel_file}}\pysiglinewithargsret{\sphinxbfcode{\sphinxupquote{\_read\_reference\_pixel\_file}}}{}{}
Read the calibration file containig the definition
of the reference pixel appropriate for a given sub array and or filer
\begin{quote}\begin{description}
\item[{Variables}] \leavevmode
\sphinxstyleliteralstrong{\sphinxupquote{reference\_pixels}} (\sphinxtitleref{collections.OrderedDict}) \textendash{} Ordered dict containing the reference pixels to be used in the
wavelength calibration.

\end{description}\end{quote}

\end{fulllineitems}

\index{\_search\_ref\_pixel\_cal\_file() (HSTWFC3 static method)@\spxentry{\_search\_ref\_pixel\_cal\_file()}\spxextra{HSTWFC3 static method}}

\begin{fulllineitems}
\phantomsection\label{\detokenize{cascade.instruments:cascade.instruments.instruments.HSTWFC3._search_ref_pixel_cal_file}}\pysiglinewithargsret{\sphinxbfcode{\sphinxupquote{static }}\sphinxbfcode{\sphinxupquote{\_search\_ref\_pixel\_cal\_file}}}{\emph{ptable}, \emph{inst\_aperture}, \emph{inst\_filter}}{}
Search the reference pixel calibration file for the reference pixel
given the instrument aperture and filter.
\begin{quote}\begin{description}
\item[{Parameters}] \leavevmode\begin{itemize}
\item {} 
\sphinxstyleliteralstrong{\sphinxupquote{ptable}} (\sphinxtitleref{dict}) \textendash{} Calibratrion table with reference positions

\item {} 
\sphinxstyleliteralstrong{\sphinxupquote{inst\_aperture}} (\sphinxtitleref{str}) \textendash{} The instrument aperture

\item {} 
\sphinxstyleliteralstrong{\sphinxupquote{inst\_filter}} (\sphinxtitleref{str}) \textendash{} The instrument filter

\end{itemize}

\item[{Returns}] \leavevmode
\begin{itemize}
\item {} 
\sphinxstylestrong{XREF} (\sphinxtitleref{float}) \textendash{} X reference position for the acquisition image

\item {} 
\sphinxstylestrong{YREF} (\sphinxtitleref{float}) \textendash{} Y reference position for the acquisition image

\end{itemize}


\item[{Raises}] \leavevmode
\sphinxhref{https://docs.python.org/3/library/exceptions.html\#ValueError}{\sphinxcode{\sphinxupquote{ValueError}}} \textendash{} An error is raises if the instrument aperture if filter is not
fount in the calibration table

\end{description}\end{quote}

\begin{sphinxadmonition}{note}{Notes}
\begin{description}
\item[{See also:}] \leavevmode
\sphinxurl{http://www.stsci.edu/hst/observatory/apertures/wfc3.html}

\end{description}
\end{sphinxadmonition}

\end{fulllineitems}

\index{\_get\_subarray\_size() (HSTWFC3 method)@\spxentry{\_get\_subarray\_size()}\spxextra{HSTWFC3 method}}

\begin{fulllineitems}
\phantomsection\label{\detokenize{cascade.instruments:cascade.instruments.instruments.HSTWFC3._get_subarray_size}}\pysiglinewithargsret{\sphinxbfcode{\sphinxupquote{\_get\_subarray\_size}}}{\emph{calibration\_data}, \emph{spectral\_data}}{}
This function determines the size of the used subarray.
\begin{quote}\begin{description}
\item[{Parameters}] \leavevmode\begin{itemize}
\item {} 
\sphinxstyleliteralstrong{\sphinxupquote{calibration\_data}} \textendash{} 

\item {} 
\sphinxstyleliteralstrong{\sphinxupquote{spectral\_data}} \textendash{} 

\end{itemize}

\item[{Variables}] \leavevmode
\sphinxstyleliteralstrong{\sphinxupquote{subarray\_sizes}} \textendash{} 

\item[{Raises}] \leavevmode
\sphinxhref{https://docs.python.org/3/library/exceptions.html\#AttributeError}{\sphinxcode{\sphinxupquote{AttributeError}}}

\end{description}\end{quote}

\end{fulllineitems}

\index{\_get\_wavelength\_calibration() (HSTWFC3 method)@\spxentry{\_get\_wavelength\_calibration()}\spxextra{HSTWFC3 method}}

\begin{fulllineitems}
\phantomsection\label{\detokenize{cascade.instruments:cascade.instruments.instruments.HSTWFC3._get_wavelength_calibration}}\pysiglinewithargsret{\sphinxbfcode{\sphinxupquote{\_get\_wavelength\_calibration}}}{}{}
The functions returns the wavelength calibration
\begin{quote}\begin{description}
\item[{Returns}] \leavevmode
\sphinxstylestrong{wave\_cal} (\sphinxtitleref{ndarray}) \textendash{} Wavelength calibraiton of the observations.

\item[{Raises}] \leavevmode
\sphinxhref{https://docs.python.org/3/library/exceptions.html\#AttributeError}{\sphinxcode{\sphinxupquote{AttributeError}}} \textendash{} An error is raised if the necessary calibration data
is not yet defined.

\end{description}\end{quote}

\end{fulllineitems}

\index{get\_spectral\_trace() (HSTWFC3 method)@\spxentry{get\_spectral\_trace()}\spxextra{HSTWFC3 method}}

\begin{fulllineitems}
\phantomsection\label{\detokenize{cascade.instruments:cascade.instruments.instruments.HSTWFC3.get_spectral_trace}}\pysiglinewithargsret{\sphinxbfcode{\sphinxupquote{get\_spectral\_trace}}}{}{}
Get spectral trace
\begin{quote}\begin{description}
\item[{Returns}] \leavevmode
\sphinxstylestrong{spectral\_trace} (\sphinxtitleref{collections.OrderedDict}) \textendash{} The spectral trace of the dispersed light (both position and
wavelength)

\item[{Raises}] \leavevmode
\sphinxhref{https://docs.python.org/3/library/exceptions.html\#AttributeError}{\sphinxcode{\sphinxupquote{AttributeError}}} \textendash{} An error is raised in the necessary calibration data is
not yet defined.

\end{description}\end{quote}

\end{fulllineitems}

\index{\_WFC3Trace() (HSTWFC3 static method)@\spxentry{\_WFC3Trace()}\spxextra{HSTWFC3 static method}}

\begin{fulllineitems}
\phantomsection\label{\detokenize{cascade.instruments:cascade.instruments.instruments.HSTWFC3._WFC3Trace}}\pysiglinewithargsret{\sphinxbfcode{\sphinxupquote{static }}\sphinxbfcode{\sphinxupquote{\_WFC3Trace}}}{\emph{xc}, \emph{yc}, \emph{DYDX}, \emph{xref=522}, \emph{yref=522}, \emph{xref\_grism=522}, \emph{yref\_grism=522}, \emph{subarray=256}, \emph{subarray\_grism=256}}{}
This function defines the spectral trace for the wfc3 grism modes.
\begin{quote}\begin{description}
\item[{Parameters}] \leavevmode\begin{itemize}
\item {} 
\sphinxstyleliteralstrong{\sphinxupquote{xc}} \textendash{} 

\item {} 
\sphinxstyleliteralstrong{\sphinxupquote{yc}} \textendash{} 

\item {} 
\sphinxstyleliteralstrong{\sphinxupquote{DYDX}} \textendash{} 

\item {} 
\sphinxstyleliteralstrong{\sphinxupquote{xref=522}} \textendash{} 

\item {} 
\sphinxstyleliteralstrong{\sphinxupquote{yref=522}} \textendash{} 

\item {} 
\sphinxstyleliteralstrong{\sphinxupquote{xref\_grism=522}} \textendash{} 

\item {} 
\sphinxstyleliteralstrong{\sphinxupquote{yref\_grism=522}} \textendash{} 

\item {} 
\sphinxstyleliteralstrong{\sphinxupquote{subarray=256}} \textendash{} 

\item {} 
\sphinxstyleliteralstrong{\sphinxupquote{subarray\_grism=256}} \textendash{} 

\end{itemize}

\item[{Returns}] \leavevmode
\sphinxstyleemphasis{trace}

\end{description}\end{quote}

\begin{sphinxadmonition}{note}{Notes}

Details can be found in:
\begin{quote}

\sphinxurl{http://www.stsci.edu/hst/wfc3/documents/ISRs/WFC3-2016-15.pdf}
\end{quote}

and
\begin{quote}

\sphinxurl{http://www.stsci.edu/hst/observatory/apertures/wfc3.html}
\end{quote}
\end{sphinxadmonition}

\end{fulllineitems}

\index{\_WFC3Dispersion() (HSTWFC3 static method)@\spxentry{\_WFC3Dispersion()}\spxextra{HSTWFC3 static method}}

\begin{fulllineitems}
\phantomsection\label{\detokenize{cascade.instruments:cascade.instruments.instruments.HSTWFC3._WFC3Dispersion}}\pysiglinewithargsret{\sphinxbfcode{\sphinxupquote{static }}\sphinxbfcode{\sphinxupquote{\_WFC3Dispersion}}}{\emph{xc}, \emph{yc}, \emph{DYDX}, \emph{DLDP}, \emph{xref=522}, \emph{yref=522}, \emph{xref\_grism=522}, \emph{yref\_grism=522}, \emph{subarray=256}, \emph{subarray\_grism=256}}{}
Convert pixel coordinate to wavelength.
\begin{quote}\begin{description}
\item[{Parameters}] \leavevmode\begin{itemize}
\item {} 
\sphinxstyleliteralstrong{\sphinxupquote{xc}} \textendash{} X coordinate of direct image centroid

\item {} 
\sphinxstyleliteralstrong{\sphinxupquote{yc}} \textendash{} Y coordinate of direct image centroid

\item {} 
\sphinxstyleliteralstrong{\sphinxupquote{xref}} \textendash{} 

\item {} 
\sphinxstyleliteralstrong{\sphinxupquote{yref}} \textendash{} 

\item {} 
\sphinxstyleliteralstrong{\sphinxupquote{xref\_grism}} \textendash{} 

\item {} 
\sphinxstyleliteralstrong{\sphinxupquote{yref\_grism}} \textendash{} 

\item {} 
\sphinxstyleliteralstrong{\sphinxupquote{subarray}} \textendash{} 

\item {} 
\sphinxstyleliteralstrong{\sphinxupquote{subarray\_grism}} \textendash{} 

\end{itemize}

\item[{Returns}] \leavevmode
\sphinxstylestrong{wavelength} (\sphinxstyleemphasis{‘astropy.units.core.Quantity’}) \textendash{} return wavelength mapping of x coordinate in micron

\end{description}\end{quote}

\begin{sphinxadmonition}{note}{Notes}

For details of the method and coefficient adopted see %
\begin{footnote}[1]\sphinxAtStartFootnote
Kuntschner et al. (2009)
%
\end{footnote} and %
\begin{footnote}[2]\sphinxAtStartFootnote
Wilkins et al. (2014)
%
\end{footnote}.
See also:
\begin{quote}

\sphinxurl{http://www.stsci.edu/hst/wfc3/documents/ISRs/WFC3-2016-15.pdf}
\end{quote}

In case the direct image and spectral image are not taken with the
same aperture, the centroid measurement is adjusted according to the
table in:
\begin{quote}

\sphinxurl{http://www.stsci.edu/hst/observatory/apertures/wfc3.html}
\end{quote}
\end{sphinxadmonition}

\begin{sphinxadmonition}{note}{References}
\end{sphinxadmonition}

\end{fulllineitems}


\end{fulllineitems}

\index{Spitzer (class in cascade.instruments.instruments)@\spxentry{Spitzer}\spxextra{class in cascade.instruments.instruments}}

\begin{fulllineitems}
\phantomsection\label{\detokenize{cascade.instruments:cascade.instruments.instruments.Spitzer}}\pysigline{\sphinxbfcode{\sphinxupquote{class }}\sphinxbfcode{\sphinxupquote{Spitzer}}}
Bases: {\hyperref[\detokenize{cascade.instruments:cascade.instruments.instruments.ObservatoryBase}]{\sphinxcrossref{\sphinxcode{\sphinxupquote{cascade.instruments.instruments.ObservatoryBase}}}}}

This observatory class defines the instuments and data handling for the
spectropgraphs of the Spitzer Space telescope
\index{name (Spitzer attribute)@\spxentry{name}\spxextra{Spitzer attribute}}

\begin{fulllineitems}
\phantomsection\label{\detokenize{cascade.instruments:cascade.instruments.instruments.Spitzer.name}}\pysigline{\sphinxbfcode{\sphinxupquote{name}}}
Name of the observatory.

\end{fulllineitems}

\index{location (Spitzer attribute)@\spxentry{location}\spxextra{Spitzer attribute}}

\begin{fulllineitems}
\phantomsection\label{\detokenize{cascade.instruments:cascade.instruments.instruments.Spitzer.location}}\pysigline{\sphinxbfcode{\sphinxupquote{location}}}
Location of the observatory

\end{fulllineitems}

\index{NAIF\_ID (Spitzer attribute)@\spxentry{NAIF\_ID}\spxextra{Spitzer attribute}}

\begin{fulllineitems}
\phantomsection\label{\detokenize{cascade.instruments:cascade.instruments.instruments.Spitzer.NAIF_ID}}\pysigline{\sphinxbfcode{\sphinxupquote{NAIF\_ID}}}
NAIF ID of the observatory. With this the location relative to the
sun and the observed target as a function of time can be determined.
Needed to calculate BJD time.

\end{fulllineitems}

\index{observatory\_instruments (Spitzer attribute)@\spxentry{observatory\_instruments}\spxextra{Spitzer attribute}}

\begin{fulllineitems}
\phantomsection\label{\detokenize{cascade.instruments:cascade.instruments.instruments.Spitzer.observatory_instruments}}\pysigline{\sphinxbfcode{\sphinxupquote{observatory\_instruments}}}
The names of the instruments part of the observatory.

\end{fulllineitems}


\end{fulllineitems}

\index{SpitzerIRS (class in cascade.instruments.instruments)@\spxentry{SpitzerIRS}\spxextra{class in cascade.instruments.instruments}}

\begin{fulllineitems}
\phantomsection\label{\detokenize{cascade.instruments:cascade.instruments.instruments.SpitzerIRS}}\pysigline{\sphinxbfcode{\sphinxupquote{class }}\sphinxbfcode{\sphinxupquote{SpitzerIRS}}}
Bases: {\hyperref[\detokenize{cascade.instruments:cascade.instruments.instruments.InstrumentBase}]{\sphinxcrossref{\sphinxcode{\sphinxupquote{cascade.instruments.instruments.InstrumentBase}}}}}

This instrument class defines the properties of the IRS instrument of
the Spitzer Space Telescope.
For the instrument and observations the following valid options are
available:
\begin{itemize}
\item {} 
detectors :  \{‘SL’, ‘LL’\}

\item {} 
spectral orders : \{‘1’, ‘2’\}

\item {} 
data products : \{‘droop’, ‘COE’\}

\item {} 
observing mode : \{‘STARING’, ‘NODDED’\}

\item {} 
data type : \{‘SPECTRUM’, ‘SPECTRAL\_IMAGE’, ‘SPECTRAL\_DETECTOR\_CUBE’\}

\end{itemize}
\index{name (SpitzerIRS attribute)@\spxentry{name}\spxextra{SpitzerIRS attribute}}

\begin{fulllineitems}
\phantomsection\label{\detokenize{cascade.instruments:cascade.instruments.instruments.SpitzerIRS.name}}\pysigline{\sphinxbfcode{\sphinxupquote{name}}}
Name of the instrument.

\end{fulllineitems}

\index{load\_data() (SpitzerIRS method)@\spxentry{load\_data()}\spxextra{SpitzerIRS method}}

\begin{fulllineitems}
\phantomsection\label{\detokenize{cascade.instruments:cascade.instruments.instruments.SpitzerIRS.load_data}}\pysiglinewithargsret{\sphinxbfcode{\sphinxupquote{load\_data}}}{}{}
Method which allows to load data.

\end{fulllineitems}

\index{get\_instrument\_setup() (SpitzerIRS method)@\spxentry{get\_instrument\_setup()}\spxextra{SpitzerIRS method}}

\begin{fulllineitems}
\phantomsection\label{\detokenize{cascade.instruments:cascade.instruments.instruments.SpitzerIRS.get_instrument_setup}}\pysiglinewithargsret{\sphinxbfcode{\sphinxupquote{get\_instrument\_setup}}}{}{}
Retrieve all relevant parameters defining the instrument and data setup

\end{fulllineitems}

\index{get\_spectra() (SpitzerIRS method)@\spxentry{get\_spectra()}\spxextra{SpitzerIRS method}}

\begin{fulllineitems}
\phantomsection\label{\detokenize{cascade.instruments:cascade.instruments.instruments.SpitzerIRS.get_spectra}}\pysiglinewithargsret{\sphinxbfcode{\sphinxupquote{get\_spectra}}}{\emph{is\_background=False}}{}
This function combines all functionallity to read fits files
containing the (uncalibrated) spectral timeseries, including
orbital phase and wavelength information
\begin{quote}\begin{description}
\item[{Parameters}] \leavevmode
\sphinxstyleliteralstrong{\sphinxupquote{is\_background}} (\sphinxtitleref{bool}) \textendash{} if \sphinxtitleref{True} the data represents an observaton of the IR background
to be subtracted of the data of the transit spectroscopy target.

\item[{Returns}] \leavevmode
\sphinxstylestrong{SpectralTimeSeries} (\sphinxtitleref{cascade.data\_model.SpectralDataTimeSeries}) \textendash{} Instance of \sphinxtitleref{SpectralDataTimeSeries} containing all spectroscopic
data including uncertainties, time, wavelength and bad pixel mask.

\item[{Raises}] \leavevmode
\sphinxstyleemphasis{AssertionError, KeyError} \textendash{} Raises an error if no data is found or if certain expected
fits keywords are not present in the data files.

\end{description}\end{quote}

\end{fulllineitems}

\index{get\_spectral\_images() (SpitzerIRS method)@\spxentry{get\_spectral\_images()}\spxextra{SpitzerIRS method}}

\begin{fulllineitems}
\phantomsection\label{\detokenize{cascade.instruments:cascade.instruments.instruments.SpitzerIRS.get_spectral_images}}\pysiglinewithargsret{\sphinxbfcode{\sphinxupquote{get\_spectral\_images}}}{\emph{is\_background=False}}{}
This function combines all functionallity to read fits files
containing the (uncalibrated) spectral timeseries, including
orbital phase and wavelength information
\begin{quote}\begin{description}
\item[{Parameters}] \leavevmode
\sphinxstyleliteralstrong{\sphinxupquote{is\_background}} (\sphinxtitleref{bool}) \textendash{} if \sphinxtitleref{True} the data represents an observaton of the IR background
to be subtracted of the data of the transit spectroscopy target.

\item[{Returns}] \leavevmode
\sphinxstylestrong{SpectralTimeSeries} (\sphinxtitleref{cascade.data\_model.SpectralDataTimeSeries}) \textendash{} Instance of \sphinxtitleref{SpectralDataTimeSeries} containing all spectroscopic
data including uncertainties, time, wavelength and bad pixel mask.

\item[{Raises}] \leavevmode
\sphinxstyleemphasis{AssertionError, KeyError} \textendash{} Raises an error if no data is found or if certain expected
fits keywords are not present in the data files.

\end{description}\end{quote}

\begin{sphinxadmonition}{note}{Notes}
\begin{description}
\item[{Notes on FOV:}] \leavevmode
in the fits header the following relevant info is used:
\begin{itemize}
\item {} 
FOVID     26     IRS\_Short-Lo\_1st\_Order\_1st\_Position

\item {} 
FOVID     27     IRS\_Short-Lo\_1st\_Order\_2nd\_Position

\item {} 
FOVID     28     IRS\_Short-Lo\_1st\_Order\_Center\_Position

\item {} 
FOVID     29     IRS\_Short-Lo\_Module\_Center

\item {} 
FOVID     32     IRS\_Short-Lo\_2nd\_Order\_1st\_Position

\item {} 
FOVID     33     IRS\_Short-Lo\_2nd\_Order\_2nd\_Position

\item {} 
FOVID     34     IRS\_Short-Lo\_2nd\_Order\_Center\_Position

\item {} 
FOVID     40     IRS\_Long-Lo\_1st\_Order\_Center\_Position

\item {} 
FOVID     46     IRS\_Long-Lo\_2nd\_Order\_Center\_Position

\end{itemize}

\end{description}

Notes on timing:
\begin{itemize}
\item {} 
FRAMTIME the total effective exposure time (ramp length)
in seconds

\end{itemize}
\end{sphinxadmonition}

\end{fulllineitems}

\index{\_define\_convolution\_kernel() (SpitzerIRS method)@\spxentry{\_define\_convolution\_kernel()}\spxextra{SpitzerIRS method}}

\begin{fulllineitems}
\phantomsection\label{\detokenize{cascade.instruments:cascade.instruments.instruments.SpitzerIRS._define_convolution_kernel}}\pysiglinewithargsret{\sphinxbfcode{\sphinxupquote{\_define\_convolution\_kernel}}}{}{}
Define the instrument specific convolution kernel which will be used
in the correction procedure of bad pixels

\end{fulllineitems}

\index{\_define\_region\_of\_interest() (SpitzerIRS method)@\spxentry{\_define\_region\_of\_interest()}\spxextra{SpitzerIRS method}}

\begin{fulllineitems}
\phantomsection\label{\detokenize{cascade.instruments:cascade.instruments.instruments.SpitzerIRS._define_region_of_interest}}\pysiglinewithargsret{\sphinxbfcode{\sphinxupquote{\_define\_region\_of\_interest}}}{}{}
Defines region on detector which containes the intended target star.

\end{fulllineitems}

\index{\_get\_order\_mask() (SpitzerIRS method)@\spxentry{\_get\_order\_mask()}\spxextra{SpitzerIRS method}}

\begin{fulllineitems}
\phantomsection\label{\detokenize{cascade.instruments:cascade.instruments.instruments.SpitzerIRS._get_order_mask}}\pysiglinewithargsret{\sphinxbfcode{\sphinxupquote{\_get\_order\_mask}}}{}{}
Gets the mask which defines the pixels used with a given spectral order

\end{fulllineitems}

\index{\_get\_wavelength\_calibration() (SpitzerIRS method)@\spxentry{\_get\_wavelength\_calibration()}\spxextra{SpitzerIRS method}}

\begin{fulllineitems}
\phantomsection\label{\detokenize{cascade.instruments:cascade.instruments.instruments.SpitzerIRS._get_wavelength_calibration}}\pysiglinewithargsret{\sphinxbfcode{\sphinxupquote{\_get\_wavelength\_calibration}}}{}{}
Get wavelength calibration file

\end{fulllineitems}

\index{get\_detector\_cubes() (SpitzerIRS method)@\spxentry{get\_detector\_cubes()}\spxextra{SpitzerIRS method}}

\begin{fulllineitems}
\phantomsection\label{\detokenize{cascade.instruments:cascade.instruments.instruments.SpitzerIRS.get_detector_cubes}}\pysiglinewithargsret{\sphinxbfcode{\sphinxupquote{get\_detector\_cubes}}}{\emph{is\_background=False}}{}
This function combines all functionallity to read fits files
containing the (uncalibrated) detector cubes (detector data
on ramp level) timeseries, including
orbital phase and wavelength information
\begin{quote}\begin{description}
\item[{Parameters}] \leavevmode
\sphinxstyleliteralstrong{\sphinxupquote{is\_background}} (\sphinxtitleref{bool}) \textendash{} if \sphinxtitleref{True} the data represents an observaton of the IR background
to be subtracted of the data of the transit spectroscopy target.

\item[{Returns}] \leavevmode
\sphinxstylestrong{SpectralTimeSeries} (\sphinxtitleref{cascade.data\_model.SpectralDataTimeSeries}) \textendash{} Instance of \sphinxtitleref{SpectralDataTimeSeries} containing all spectroscopic
data including uncertainties, time, wavelength and bad pixel mask.

\item[{Raises}] \leavevmode
\sphinxstyleemphasis{AssertionError, KeyError} \textendash{} Raises an error if no data is found or if certain expected
fits keywords are not present in the data files.

\end{description}\end{quote}

\begin{sphinxadmonition}{note}{Notes}

Notes on timing in header:

There are several integration-time-related keywords.
Of greatest interest to the observer is the
“effective integration time”, which is the time on-chip between
the first and last non-destructive reads for each pixel. It is called:
\begin{quote}

RAMPTIME = Total integration time for the current DCE.
\end{quote}

The value of RAMPTIME gives the usable portion of the integration ramp,
occurring between the beginning of the first read and the end of the
last read. It excludes detector array pre-conditioning time.
It may also be of interest to know the exposure time at other points
along the ramp. The SUR sequence consists of the time taken at the
beginning of a SUR sequence to condition the array
(header keyword DEADTIME), the time taken to complete one read and
one spin through the array (GRPTIME), and the non-destructive reads
separated by uniform wait times. The wait consists of “clocking”
through the array without reading or resetting. The time it takes to
clock through the array once is given by the SAMPTIME keyword.
So, for an N-read ramp:
\begin{quote}

RAMPTIME = 2x(N-1)xSAMPTIME
\end{quote}

and
\begin{quote}

DCE duration = DEADTIME + GRPTIME + RAMPTIME
\end{quote}

Note that peak-up data is not obtained in SUR mode. It is obtained in
Double Correlated Sampling (DCS) mode. In that case, RAMPTIME gives the
time interval between the 2nd sample and the preceeding reset.
\end{sphinxadmonition}

\end{fulllineitems}

\index{get\_spectral\_trace() (SpitzerIRS method)@\spxentry{get\_spectral\_trace()}\spxextra{SpitzerIRS method}}

\begin{fulllineitems}
\phantomsection\label{\detokenize{cascade.instruments:cascade.instruments.instruments.SpitzerIRS.get_spectral_trace}}\pysiglinewithargsret{\sphinxbfcode{\sphinxupquote{get\_spectral\_trace}}}{}{}
Get spectral trace

\end{fulllineitems}


\end{fulllineitems}



\subsection{The cascade.utilities module}
\label{\detokenize{cascade.utilities:module-cascade.utilities.utilities}}\label{\detokenize{cascade.utilities:the-cascade-utilities-module}}\label{\detokenize{cascade.utilities::doc}}\index{cascade.utilities.utilities (module)@\spxentry{cascade.utilities.utilities}\spxextra{module}}
This Module defines some utility functions used in cascade
\index{write\_timeseries\_to\_fits() (in module cascade.utilities.utilities)@\spxentry{write\_timeseries\_to\_fits()}\spxextra{in module cascade.utilities.utilities}}

\begin{fulllineitems}
\phantomsection\label{\detokenize{cascade.utilities:cascade.utilities.utilities.write_timeseries_to_fits}}\pysiglinewithargsret{\sphinxbfcode{\sphinxupquote{write\_timeseries\_to\_fits}}}{\emph{data}, \emph{path}}{}
Write spectral timeseries data object to fits files
\begin{quote}\begin{description}
\item[{Parameters}] \leavevmode\begin{itemize}
\item {} 
\sphinxstyleliteralstrong{\sphinxupquote{data}} (\sphinxstyleliteralemphasis{\sphinxupquote{'ndarry'}}\sphinxstyleliteralemphasis{\sphinxupquote{ or }}\sphinxstyleliteralemphasis{\sphinxupquote{'cascade.data\_model.SpectralDataTimeSeries'}}) \textendash{} The data cube which will be save to fits file. For each time step
a fits file will be generated.

\item {} 
\sphinxstyleliteralstrong{\sphinxupquote{path}} (\sphinxstyleliteralemphasis{\sphinxupquote{'str'}}) \textendash{} Path to the directory where the fits files will be saved.

\end{itemize}

\end{description}\end{quote}

\end{fulllineitems}

\index{find() (in module cascade.utilities.utilities)@\spxentry{find()}\spxextra{in module cascade.utilities.utilities}}

\begin{fulllineitems}
\phantomsection\label{\detokenize{cascade.utilities:cascade.utilities.utilities.find}}\pysiglinewithargsret{\sphinxbfcode{\sphinxupquote{find}}}{\emph{pattern}, \emph{path}}{}
Return  a list of all data files
\begin{quote}\begin{description}
\item[{Parameters}] \leavevmode\begin{itemize}
\item {} 
\sphinxstyleliteralstrong{\sphinxupquote{pattern}} (\sphinxstyleliteralemphasis{\sphinxupquote{'str'}}) \textendash{} Pattern used to search for files.

\item {} 
\sphinxstyleliteralstrong{\sphinxupquote{" 'str'}} (\sphinxstyleliteralemphasis{\sphinxupquote{path}}) \textendash{} Path to directory to be searched.

\end{itemize}

\item[{Returns}] \leavevmode
\sphinxstylestrong{result} (\sphinxstyleemphasis{‘list’ of ‘str’}) \textendash{} Sorted list of filenames matching the ‘pattern’ search

\end{description}\end{quote}

\end{fulllineitems}

\index{spectres() (in module cascade.utilities.utilities)@\spxentry{spectres()}\spxextra{in module cascade.utilities.utilities}}

\begin{fulllineitems}
\phantomsection\label{\detokenize{cascade.utilities:cascade.utilities.utilities.spectres}}\pysiglinewithargsret{\sphinxbfcode{\sphinxupquote{spectres}}}{\emph{new\_spec\_wavs}, \emph{old\_spec\_wavs}, \emph{spec\_fluxes}, \emph{spec\_errs=None}}{}
SpectRes: A fast spectral resampling function.
Copyright (C) 2017  A. C. Carnall
Function for resampling spectra (and optionally associated uncertainties)
onto a new wavelength basis.
\begin{quote}\begin{description}
\item[{Parameters}] \leavevmode\begin{itemize}
\item {} 
\sphinxstyleliteralstrong{\sphinxupquote{new\_spec\_wavs}} (\sphinxstyleliteralemphasis{\sphinxupquote{numpy.ndarray}}) \textendash{} Array containing the new wavelength sampling desired for the spectrum
or spectra.

\item {} 
\sphinxstyleliteralstrong{\sphinxupquote{old\_spec\_wavs}} (\sphinxstyleliteralemphasis{\sphinxupquote{numpy.ndarray}}) \textendash{} 1D array containing the current wavelength sampling of the spectrum or
spectra.

\item {} 
\sphinxstyleliteralstrong{\sphinxupquote{spec\_fluxes}} (\sphinxstyleliteralemphasis{\sphinxupquote{numpy.ndarray}}) \textendash{} Array containing spectral fluxes at the wavelengths specified in
old\_spec\_wavs, last dimension must correspond to the shape of
old\_spec\_wavs.
Extra dimensions before this may be used to include multiple spectra.

\item {} 
\sphinxstyleliteralstrong{\sphinxupquote{spec\_errs}} (\sphinxstyleliteralemphasis{\sphinxupquote{numpy.ndarray}}\sphinxstyleliteralemphasis{\sphinxupquote{ (}}\sphinxstyleliteralemphasis{\sphinxupquote{optional}}\sphinxstyleliteralemphasis{\sphinxupquote{)}}) \textendash{} Array of the same shape as spec\_fluxes containing uncertainties
associated with each spectral flux value.

\end{itemize}

\item[{Returns}] \leavevmode
\begin{itemize}
\item {} 
\sphinxstylestrong{resampled\_fluxes} (\sphinxstyleemphasis{numpy.ndarray}) \textendash{} Array of resampled flux values, first dimension is the same length
as new\_spec\_wavs, other dimensions are the same as spec\_fluxes

\item {} 
\sphinxstylestrong{resampled\_errs} (\sphinxstyleemphasis{numpy.ndarray}) \textendash{} Array of uncertainties associated with fluxes in resampled\_fluxes.
Only returned if spec\_errs was specified.

\end{itemize}


\end{description}\end{quote}

\end{fulllineitems}



\chapter{Indices and tables}
\label{\detokenize{index:indices-and-tables}}\begin{itemize}
\item {} 
\DUrole{xref,std,std-ref}{genindex}

\item {} 
\DUrole{xref,std,std-ref}{search}

\end{itemize}


\renewcommand{\indexname}{Python Module Index}
\begin{sphinxtheindex}
\let\bigletter\sphinxstyleindexlettergroup
\bigletter{c}
\item\relax\sphinxstyleindexentry{cascade.cpm\_model.cpm\_model}\sphinxstyleindexpageref{cascade.cpm_model:\detokenize{module-cascade.cpm_model.cpm_model}}
\item\relax\sphinxstyleindexentry{cascade.data\_model.data\_model}\sphinxstyleindexpageref{cascade.data_model:\detokenize{module-cascade.data_model.data_model}}
\item\relax\sphinxstyleindexentry{cascade.exoplanet\_tools.exoplanet\_tools}\sphinxstyleindexpageref{cascade.exoplanet_tools:\detokenize{module-cascade.exoplanet_tools.exoplanet_tools}}
\item\relax\sphinxstyleindexentry{cascade.initialize.initialize}\sphinxstyleindexpageref{cascade.initialize:\detokenize{module-cascade.initialize.initialize}}
\item\relax\sphinxstyleindexentry{cascade.instruments.instruments}\sphinxstyleindexpageref{cascade.instruments:\detokenize{module-cascade.instruments.instruments}}
\item\relax\sphinxstyleindexentry{cascade.TSO.TSO}\sphinxstyleindexpageref{cascade.TSO:\detokenize{module-cascade.TSO.TSO}}
\item\relax\sphinxstyleindexentry{cascade.utilities.utilities}\sphinxstyleindexpageref{cascade.utilities:\detokenize{module-cascade.utilities.utilities}}
\end{sphinxtheindex}

\renewcommand{\indexname}{Index}
\printindex
\end{document}