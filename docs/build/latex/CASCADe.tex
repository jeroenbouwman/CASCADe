%% Generated by Sphinx.
\def\sphinxdocclass{report}
\documentclass[a4paper,11pt,english]{sphinxmanual}
\ifdefined\pdfpxdimen
   \let\sphinxpxdimen\pdfpxdimen\else\newdimen\sphinxpxdimen
\fi \sphinxpxdimen=.75bp\relax

\PassOptionsToPackage{warn}{textcomp}
\usepackage[utf8]{inputenc}
\ifdefined\DeclareUnicodeCharacter
% support both utf8 and utf8x syntaxes
\edef\sphinxdqmaybe{\ifdefined\DeclareUnicodeCharacterAsOptional\string"\fi}
  \DeclareUnicodeCharacter{\sphinxdqmaybe00A0}{\nobreakspace}
  \DeclareUnicodeCharacter{\sphinxdqmaybe2500}{\sphinxunichar{2500}}
  \DeclareUnicodeCharacter{\sphinxdqmaybe2502}{\sphinxunichar{2502}}
  \DeclareUnicodeCharacter{\sphinxdqmaybe2514}{\sphinxunichar{2514}}
  \DeclareUnicodeCharacter{\sphinxdqmaybe251C}{\sphinxunichar{251C}}
  \DeclareUnicodeCharacter{\sphinxdqmaybe2572}{\textbackslash}
\fi
\usepackage{cmap}
\usepackage[T1]{fontenc}
\usepackage{amsmath,amssymb,amstext}
\usepackage{babel}
\usepackage{times}
\usepackage[Bjarne]{fncychap}
\usepackage{sphinx}

\fvset{fontsize=\small}
\usepackage{geometry}

% Include hyperref last.
\usepackage{hyperref}
% Fix anchor placement for figures with captions.
\usepackage{hypcap}% it must be loaded after hyperref.
% Set up styles of URL: it should be placed after hyperref.
\urlstyle{same}

\addto\captionsenglish{\renewcommand{\figurename}{Fig.}}
\addto\captionsenglish{\renewcommand{\tablename}{Table}}
\addto\captionsenglish{\renewcommand{\literalblockname}{Listing}}

\addto\captionsenglish{\renewcommand{\literalblockcontinuedname}{continued from previous page}}
\addto\captionsenglish{\renewcommand{\literalblockcontinuesname}{continues on next page}}
\addto\captionsenglish{\renewcommand{\sphinxnonalphabeticalgroupname}{Non-alphabetical}}
\addto\captionsenglish{\renewcommand{\sphinxsymbolsname}{Symbols}}
\addto\captionsenglish{\renewcommand{\sphinxnumbersname}{Numbers}}

\addto\extrasenglish{\def\pageautorefname{page}}

\setcounter{tocdepth}{0}



\title{CASCADe Documentation}
\date{Feb 02, 2019}
\release{0.9 beta}
\author{Jeroen Bouwman}
\newcommand{\sphinxlogo}{\vbox{}}
\renewcommand{\releasename}{Release}
\makeindex
\begin{document}

\pagestyle{empty}
\maketitle
\pagestyle{plain}
\sphinxtableofcontents
\pagestyle{normal}
\phantomsection\label{\detokenize{index::doc}}


At present several thousand transiting exoplanet systems have been discovered.
For relatively few systems, however, a spectro-photometric characterization of
the planetary atmospheres could be performed due to the tiny photometric signatures
of the atmospheres and the large systematic noise introduced by the used instruments
or the earth atmosphere. Several methods have been developed to deal with instrument
and atmospheric noise. These methods include high precision calibration and modeling
of the instruments, modeling of the noise using methods like principle component
analysis or Gaussian processes and the simultaneous observations of many reference
stars. Though significant progress has been made, most of these methods have drawbacks
as they either have to make too many assumptions or do not fully utilize all
information available in the data to negate the noise terms.

The CASCADe project implements a novel “data driven” method, pioneered by
Schoelkopf et al (2015) utilizing the causal connections within a data set,
and uses this to calibrate the spectral timeseries data of single transiting
systems. The current code has been tested successfully to spectroscopic data
obtained with the Spitzer and HST observatories.


\chapter{CASCADe API documentation}
\label{\detokenize{index:cascade-api-documentation}}

\section{Howto Install CASCADe}
\label{\detokenize{install:howto-install-cascade}}\label{\detokenize{install::doc}}
\#\#\# Clone a repository

To start working locally on an existing remote repository,
clone it with the command \sphinxtitleref{git clone \textless{}repository path\textgreater{}}.
By cloning a repository, you’ll download a copy of its
files into your local computer, preserving the Git
connection with the remote repository.

You can either clone it via HTTPS or {[}SSH{]}(../ssh/README.md).
If you chose to clone it via HTTPS, you’ll have to enter your
credentials every time you pull and push. With SSH, you enter
your credentials once and can pull and push straightaway.

You can find both paths (HTTPS and SSH) by navigating to
your project’s landing page and clicking \sphinxstylestrong{Clone}. GitLab
will prompt you with both paths, from which you can copy
and paste in your command line.

As an example, consider a repository path:
\begin{itemize}
\item {} 
HTTPS: \sphinxtitleref{https://gitlab.com/gitlab-org/gitlab-ce.git}

\item {} 
SSH: {}`{}` \sphinxhref{mailto:git@gitlab.com}{git@gitlab.com}:gitlab-org/gitlab-ce.git {}`{}`

\end{itemize}

To get started, open a terminal window in the directory
you wish to clone the repository files into, and run one
of the following commands.

Clone via HTTPS:

\sphinxcode{\sphinxupquote{{}`bash
git clone https://gitlab.com/gitlab-org/gitlab-ce.git
{}`}}

Clone via SSH:

\sphinxcode{\sphinxupquote{{}`bash
git clone git@gitlab.com:gitlab-org/gitlab-ce.git
{}`}}

Both commands will download a copy of the files in a
folder named after the project’s name.

You can then navigate to the directory and start working
on it locally.

\#\#\# Go to the master branch to pull the latest changes from there

\sphinxcode{\sphinxupquote{{}`bash
git checkout master
{}`}}

\#\#\# Download the latest changes in the project

This is for you to work on an up-to-date copy (it is important to do this every time you start working on a project), while you set up tracking branches. You pull from remote repositories to get all the changes made by users since the last time you cloned or pulled the project. Later, you can push your local commits to the remote repositories.

\sphinxcode{\sphinxupquote{{}`bash
git pull REMOTE NAME-OF-BRANCH
{}`}}

When you first clone a repository, REMOTE is typically “origin”. This is where the repository came from, and it indicates the SSH or HTTPS URL of the repository on the remote server. NAME-OF-BRANCH is usually “master”, but it may be any existing branch.


\section{Using Cascade}
\label{\detokenize{howto:using-cascade}}\label{\detokenize{howto::doc}}
to run the code, first load all needed modules:

\fvset{hllines={, ,}}%
\begin{sphinxVerbatim}[commandchars=\\\{\}]
\PYG{k+kn}{import} \PYG{n+nn}{cascade}
\end{sphinxVerbatim}

Then, create transit spectroscopy object

\fvset{hllines={, ,}}%
\begin{sphinxVerbatim}[commandchars=\\\{\}]
\PYG{n}{tso} \PYG{o}{=} \PYG{n}{cascade}\PYG{o}{.}\PYG{n}{TSO}\PYG{o}{.}\PYG{n}{TSOSuite}\PYG{p}{(}\PYG{p}{)}
\end{sphinxVerbatim}

To reset all previous divined or initialized parameters

\fvset{hllines={, ,}}%
\begin{sphinxVerbatim}[commandchars=\\\{\}]
\PYG{n}{tso}\PYG{o}{.}\PYG{n}{execute}\PYG{p}{(}\PYG{l+s+s2}{\PYGZdq{}}\PYG{l+s+s2}{reset}\PYG{l+s+s2}{\PYGZdq{}}\PYG{p}{)}
\end{sphinxVerbatim}

Initialize the TSO object using ini files which define the data, model parameters and behavior of the causal pixel model implemented in CASCADe.

\fvset{hllines={, ,}}%
\begin{sphinxVerbatim}[commandchars=\\\{\}]
\PYG{n}{path} \PYG{o}{=} \PYG{n}{cascade}\PYG{o}{.}\PYG{n}{initialize}\PYG{o}{.}\PYG{n}{default\PYGZus{}initialization\PYGZus{}path}
\PYG{n}{tso} \PYG{o}{=} \PYG{n}{cascade}\PYG{o}{.}\PYG{n}{TSO}\PYG{o}{.}\PYG{n}{TSOSuite}\PYG{p}{(}\PYG{l+s+s2}{\PYGZdq{}}\PYG{l+s+s2}{initialize}\PYG{l+s+s2}{\PYGZdq{}}\PYG{p}{,} \PYG{l+s+s2}{\PYGZdq{}}\PYG{l+s+s2}{cascade\PYGZus{}cpm.ini}\PYG{l+s+s2}{\PYGZdq{}}\PYG{p}{,}
                           \PYG{l+s+s2}{\PYGZdq{}}\PYG{l+s+s2}{cascade\PYGZus{}object.ini}\PYG{l+s+s2}{\PYGZdq{}}\PYG{p}{,}
                           \PYG{l+s+s2}{\PYGZdq{}}\PYG{l+s+s2}{cascade\PYGZus{}data\PYGZus{}spectral\PYGZus{}images.ini}\PYG{l+s+s2}{\PYGZdq{}}\PYG{p}{,} \PYG{n}{path}\PYG{o}{=}\PYG{n}{path}\PYG{p}{)}
\end{sphinxVerbatim}

Load the observational data

\fvset{hllines={, ,}}%
\begin{sphinxVerbatim}[commandchars=\\\{\}]
\PYG{n}{tso}\PYG{o}{.}\PYG{n}{execute}\PYG{p}{(}\PYG{l+s+s2}{\PYGZdq{}}\PYG{l+s+s2}{load\PYGZus{}data}\PYG{l+s+s2}{\PYGZdq{}}\PYG{p}{)}
\end{sphinxVerbatim}

Subtract the background

\fvset{hllines={, ,}}%
\begin{sphinxVerbatim}[commandchars=\\\{\}]
\PYG{n}{tso}\PYG{o}{.}\PYG{n}{execute}\PYG{p}{(}\PYG{l+s+s2}{\PYGZdq{}}\PYG{l+s+s2}{subtract\PYGZus{}background}\PYG{l+s+s2}{\PYGZdq{}}\PYG{p}{)}
\end{sphinxVerbatim}

Sigma clip data

\fvset{hllines={, ,}}%
\begin{sphinxVerbatim}[commandchars=\\\{\}]
\PYG{n}{tso}\PYG{o}{.}\PYG{n}{execute}\PYG{p}{(}\PYG{l+s+s2}{\PYGZdq{}}\PYG{l+s+s2}{sigma\PYGZus{}clip\PYGZus{}data}\PYG{l+s+s2}{\PYGZdq{}}\PYG{p}{)}
\end{sphinxVerbatim}

Determine the position of source from the spectroscopic data set

\fvset{hllines={, ,}}%
\begin{sphinxVerbatim}[commandchars=\\\{\}]
\PYG{n}{tso}\PYG{o}{.}\PYG{n}{execute}\PYG{p}{(}\PYG{l+s+s2}{\PYGZdq{}}\PYG{l+s+s2}{determine\PYGZus{}source\PYGZus{}position}\PYG{l+s+s2}{\PYGZdq{}}\PYG{p}{)}
\end{sphinxVerbatim}

Set the extraction area within which the signal of the exoplanet will be determined

\fvset{hllines={, ,}}%
\begin{sphinxVerbatim}[commandchars=\\\{\}]
\PYG{n}{tso}\PYG{o}{.}\PYG{n}{execute}\PYG{p}{(}\PYG{l+s+s2}{\PYGZdq{}}\PYG{l+s+s2}{set\PYGZus{}extraction\PYGZus{}mask}\PYG{l+s+s2}{\PYGZdq{}}\PYG{p}{)}
\end{sphinxVerbatim}

Extract the spectrum of the Star + planet in an optimal way

\fvset{hllines={, ,}}%
\begin{sphinxVerbatim}[commandchars=\\\{\}]
\PYG{n}{tso}\PYG{o}{.}\PYG{n}{execute}\PYG{p}{(}\PYG{l+s+s2}{\PYGZdq{}}\PYG{l+s+s2}{optimal\PYGZus{}extraction}\PYG{l+s+s2}{\PYGZdq{}}\PYG{p}{)}
\end{sphinxVerbatim}

Setup the matrix of regressors used to model the noise

\fvset{hllines={, ,}}%
\begin{sphinxVerbatim}[commandchars=\\\{\}]
\PYG{n}{tso}\PYG{o}{.}\PYG{n}{execute}\PYG{p}{(}\PYG{l+s+s2}{\PYGZdq{}}\PYG{l+s+s2}{select\PYGZus{}regressors}\PYG{l+s+s2}{\PYGZdq{}}\PYG{p}{)}
\end{sphinxVerbatim}

Define the eclipse model

\fvset{hllines={, ,}}%
\begin{sphinxVerbatim}[commandchars=\\\{\}]
\PYG{n}{tso}\PYG{o}{.}\PYG{n}{execute}\PYG{p}{(}\PYG{l+s+s2}{\PYGZdq{}}\PYG{l+s+s2}{define\PYGZus{}eclipse\PYGZus{}model}\PYG{l+s+s2}{\PYGZdq{}}\PYG{p}{)}
\end{sphinxVerbatim}

Derive the calibrated time series and fit for the planetary signal

\fvset{hllines={, ,}}%
\begin{sphinxVerbatim}[commandchars=\\\{\}]
\PYG{n}{tso}\PYG{o}{.}\PYG{n}{execute}\PYG{p}{(}\PYG{l+s+s2}{\PYGZdq{}}\PYG{l+s+s2}{calibrate\PYGZus{}timeseries}\PYG{l+s+s2}{\PYGZdq{}}\PYG{p}{)}
\end{sphinxVerbatim}

Extract the planetary signal

\fvset{hllines={, ,}}%
\begin{sphinxVerbatim}[commandchars=\\\{\}]
\PYG{n}{tso}\PYG{o}{.}\PYG{n}{execute}\PYG{p}{(}\PYG{l+s+s2}{\PYGZdq{}}\PYG{l+s+s2}{extract\PYGZus{}spectrum}\PYG{l+s+s2}{\PYGZdq{}}\PYG{p}{)}
\end{sphinxVerbatim}

Correct the extracted planetary signal for non uniform subtraction of average eclipse/transit signal

\fvset{hllines={, ,}}%
\begin{sphinxVerbatim}[commandchars=\\\{\}]
\PYG{n}{tso}\PYG{o}{.}\PYG{n}{execute}\PYG{p}{(}\PYG{l+s+s2}{\PYGZdq{}}\PYG{l+s+s2}{correct\PYGZus{}extracted\PYGZus{}spectrum}\PYG{l+s+s2}{\PYGZdq{}}\PYG{p}{)}
\end{sphinxVerbatim}

Save the planetary signal

\fvset{hllines={, ,}}%
\begin{sphinxVerbatim}[commandchars=\\\{\}]
\PYG{n}{tso}\PYG{o}{.}\PYG{n}{execute}\PYG{p}{(}\PYG{l+s+s2}{\PYGZdq{}}\PYG{l+s+s2}{save\PYGZus{}results}\PYG{l+s+s2}{\PYGZdq{}}\PYG{p}{)}
\end{sphinxVerbatim}

Plot results (planetary spectrum, residual etc.)

\fvset{hllines={, ,}}%
\begin{sphinxVerbatim}[commandchars=\\\{\}]
\PYG{n}{tso}\PYG{o}{.}\PYG{n}{execute}\PYG{p}{(}\PYG{l+s+s2}{\PYGZdq{}}\PYG{l+s+s2}{plot\PYGZus{}results}\PYG{l+s+s2}{\PYGZdq{}}\PYG{p}{)}
\end{sphinxVerbatim}


\section{Modules of the CASCADe package}
\label{\detokenize{cascade:modules-of-the-cascade-package}}\label{\detokenize{cascade::doc}}

\subsection{The cascade.TSO module}
\label{\detokenize{cascade.TSO:module-cascade.TSO.TSO}}\label{\detokenize{cascade.TSO:the-cascade-tso-module}}\label{\detokenize{cascade.TSO::doc}}\index{cascade.TSO.TSO (module)@\spxentry{cascade.TSO.TSO}\spxextra{module}}
The TSO module is the main module of the CASCADe package.
The classes defined in this module define the time series object and
all routines acting upon the TSO instance to extract the spectrum of the
transiting exoplanet.
\index{TSOSuite (class in cascade.TSO.TSO)@\spxentry{TSOSuite}\spxextra{class in cascade.TSO.TSO}}

\begin{fulllineitems}
\phantomsection\label{\detokenize{cascade.TSO:cascade.TSO.TSO.TSOSuite}}\pysiglinewithargsret{\sphinxbfcode{\sphinxupquote{class }}\sphinxbfcode{\sphinxupquote{TSOSuite}}}{\emph{*init\_files}, \emph{path=None}}{}
Bases: \sphinxcode{\sphinxupquote{object}}

Transit Spectroscopy Object Suite class.

This is the main class containing the ligth curve data of and transiting
exoplanet and all functionality to calibrate and analyse the light curves
and to extractthe spectrum of the transiting exoplanet.
\begin{quote}\begin{description}
\item[{Parameters}] \leavevmode
\sphinxstyleliteralstrong{\sphinxupquote{init\_files}} (\sphinxstyleliteralemphasis{\sphinxupquote{'list' of 'str'}}) \textendash{} List containing all the initialization files needed to run the
CASCADe code.

\item[{Raises}] \leavevmode
\sphinxcode{\sphinxupquote{ValueError}} \textendash{} Raised when commands not recognized as valid

\end{description}\end{quote}

\begin{sphinxadmonition}{note}{Examples}

To make instance of TSOSuite class

\fvset{hllines={, ,}}%
\begin{sphinxVerbatim}[commandchars=\\\{\}]
\PYG{g+gp}{\PYGZgt{}\PYGZgt{}\PYGZgt{} }\PYG{n}{tso} \PYG{o}{=} \PYG{n}{cascade}\PYG{o}{.}\PYG{n}{TSO}\PYG{o}{.}\PYG{n}{TSOSuite}\PYG{p}{(}\PYG{p}{)}
\end{sphinxVerbatim}
\end{sphinxadmonition}
\index{execute() (TSOSuite method)@\spxentry{execute()}\spxextra{TSOSuite method}}

\begin{fulllineitems}
\phantomsection\label{\detokenize{cascade.TSO:cascade.TSO.TSO.TSOSuite.execute}}\pysiglinewithargsret{\sphinxbfcode{\sphinxupquote{execute}}}{\emph{command}, \emph{*init\_files}, \emph{path=None}}{}
Check if command is valid and excecute if True
\begin{quote}\begin{description}
\item[{Parameters}] \leavevmode\begin{itemize}
\item {} 
\sphinxstyleliteralstrong{\sphinxupquote{command}} \textendash{} Command to be excecuted. If valid the method corresponding
to the command will be excecuted

\item {} 
\sphinxstyleliteralstrong{\sphinxupquote{*init\_files}} \textendash{} Single or multiple file names of the .ini files containing the
parameters defining the observation and calibration settings.

\item {} 
\sphinxstyleliteralstrong{\sphinxupquote{path}} \textendash{} (optional) Filepath to the .ini files, standard value in None

\end{itemize}

\item[{Raises}] \leavevmode
\sphinxcode{\sphinxupquote{ValueError}} \textendash{} error is raised if command is not valid

\end{description}\end{quote}

\begin{sphinxadmonition}{note}{Examples}

\fvset{hllines={, ,}}%
\begin{sphinxVerbatim}[commandchars=\\\{\}]
\PYG{g+gp}{\PYGZgt{}\PYGZgt{}\PYGZgt{} }\PYG{n}{tso}\PYG{o}{.}\PYG{n}{execute}\PYG{p}{(}\PYG{l+s+s1}{\PYGZsq{}}\PYG{l+s+s1}{reset}\PYG{l+s+s1}{\PYGZsq{}}\PYG{p}{)}
\end{sphinxVerbatim}
\end{sphinxadmonition}

\end{fulllineitems}

\index{initialize\_TSO() (TSOSuite method)@\spxentry{initialize\_TSO()}\spxextra{TSOSuite method}}

\begin{fulllineitems}
\phantomsection\label{\detokenize{cascade.TSO:cascade.TSO.TSO.TSOSuite.initialize_TSO}}\pysiglinewithargsret{\sphinxbfcode{\sphinxupquote{initialize\_TSO}}}{\emph{*init\_files}, \emph{path=None}}{}
Initializes the TSO object by reading in a single or
multiple .ini files
\begin{quote}\begin{description}
\item[{Parameters}] \leavevmode\begin{itemize}
\item {} 
\sphinxstyleliteralstrong{\sphinxupquote{*init\_files}} \textendash{} Single or multiple file names of the .ini files containing the
parameters defining the observation and calibration settings.

\item {} 
\sphinxstyleliteralstrong{\sphinxupquote{path}} \textendash{} (optional) Filepath to the .ini files, standard value in None

\end{itemize}

\item[{Variables}] \leavevmode
\sphinxstyleliteralstrong{\sphinxupquote{cascade\_parameters}} \textendash{} cascade.initialize.initialize.configurator

\item[{Raises}] \leavevmode
\sphinxcode{\sphinxupquote{FileNotFoundError}} \textendash{} Raises error if .ini file is not found

\end{description}\end{quote}

\begin{sphinxadmonition}{note}{Examples}

\fvset{hllines={, ,}}%
\begin{sphinxVerbatim}[commandchars=\\\{\}]
\PYG{g+gp}{\PYGZgt{}\PYGZgt{}\PYGZgt{} }\PYG{n}{tso}\PYG{o}{.}\PYG{n}{execute}\PYG{p}{(}\PYG{l+s+s2}{\PYGZdq{}}\PYG{l+s+s2}{initialize}\PYG{l+s+s2}{\PYGZdq{}}\PYG{p}{,} \PYG{n}{init\PYGZus{}flle\PYGZus{}name}\PYG{p}{)}
\end{sphinxVerbatim}
\end{sphinxadmonition}

\end{fulllineitems}

\index{reset\_TSO() (TSOSuite method)@\spxentry{reset\_TSO()}\spxextra{TSOSuite method}}

\begin{fulllineitems}
\phantomsection\label{\detokenize{cascade.TSO:cascade.TSO.TSO.TSOSuite.reset_TSO}}\pysiglinewithargsret{\sphinxbfcode{\sphinxupquote{reset\_TSO}}}{}{}
Reset initialization of TSO object by removing all loaded parameters.

\begin{sphinxadmonition}{note}{Examples}

\fvset{hllines={, ,}}%
\begin{sphinxVerbatim}[commandchars=\\\{\}]
\PYG{g+gp}{\PYGZgt{}\PYGZgt{}\PYGZgt{} }\PYG{n}{tso}\PYG{o}{.}\PYG{n}{execute}\PYG{p}{(}\PYG{l+s+s2}{\PYGZdq{}}\PYG{l+s+s2}{reset}\PYG{l+s+s2}{\PYGZdq{}}\PYG{p}{)}
\end{sphinxVerbatim}
\end{sphinxadmonition}

\end{fulllineitems}

\index{load\_data() (TSOSuite method)@\spxentry{load\_data()}\spxextra{TSOSuite method}}

\begin{fulllineitems}
\phantomsection\label{\detokenize{cascade.TSO:cascade.TSO.TSO.TSOSuite.load_data}}\pysiglinewithargsret{\sphinxbfcode{\sphinxupquote{load\_data}}}{}{}
Load the transit time series observations from file, for the
object, observatory, instrument and file location specified in the
loaded initialization files
\begin{quote}\begin{description}
\item[{Variables}] \leavevmode
\sphinxstyleliteralstrong{\sphinxupquote{observation}} \textendash{} cascade.instruments.instruments.Observation

\end{description}\end{quote}

\begin{sphinxadmonition}{note}{Examples}

To load the observed data into the tso object:

\fvset{hllines={, ,}}%
\begin{sphinxVerbatim}[commandchars=\\\{\}]
\PYG{g+gp}{\PYGZgt{}\PYGZgt{}\PYGZgt{} }\PYG{n}{tso}\PYG{o}{.}\PYG{n}{execute}\PYG{p}{(}\PYG{l+s+s2}{\PYGZdq{}}\PYG{l+s+s2}{load\PYGZus{}data}\PYG{l+s+s2}{\PYGZdq{}}\PYG{p}{)}
\end{sphinxVerbatim}
\end{sphinxadmonition}

\end{fulllineitems}

\index{subtract\_background() (TSOSuite method)@\spxentry{subtract\_background()}\spxextra{TSOSuite method}}

\begin{fulllineitems}
\phantomsection\label{\detokenize{cascade.TSO:cascade.TSO.TSO.TSOSuite.subtract_background}}\pysiglinewithargsret{\sphinxbfcode{\sphinxupquote{subtract\_background}}}{}{}
Subtract median background determined from data or background model
from the science observations.
\begin{quote}\begin{description}
\item[{Variables}] \leavevmode
\sphinxstyleliteralstrong{\sphinxupquote{isBackgroundSubtracted}} \textendash{} True if background is subtracted

\item[{Raises}] \leavevmode
\sphinxcode{\sphinxupquote{AttributeError}} \textendash{} In case no background data is defined

\end{description}\end{quote}

\begin{sphinxadmonition}{note}{Examples}

To subtract the background from the spectral images:

\fvset{hllines={, ,}}%
\begin{sphinxVerbatim}[commandchars=\\\{\}]
\PYG{g+gp}{\PYGZgt{}\PYGZgt{}\PYGZgt{} }\PYG{n}{tso}\PYG{o}{.}\PYG{n}{execute}\PYG{p}{(}\PYG{l+s+s2}{\PYGZdq{}}\PYG{l+s+s2}{subtract\PYGZus{}background}\PYG{l+s+s2}{\PYGZdq{}}\PYG{p}{)}
\end{sphinxVerbatim}
\end{sphinxadmonition}

\end{fulllineitems}

\index{sigma\_clip\_data\_cosmic() (TSOSuite static method)@\spxentry{sigma\_clip\_data\_cosmic()}\spxextra{TSOSuite static method}}

\begin{fulllineitems}
\phantomsection\label{\detokenize{cascade.TSO:cascade.TSO.TSO.TSOSuite.sigma_clip_data_cosmic}}\pysiglinewithargsret{\sphinxbfcode{\sphinxupquote{static }}\sphinxbfcode{\sphinxupquote{sigma\_clip\_data\_cosmic}}}{\emph{data}, \emph{sigma}}{}
Sigma Clip in time.
\begin{quote}\begin{description}
\item[{Parameters}] \leavevmode\begin{itemize}
\item {} 
\sphinxstyleliteralstrong{\sphinxupquote{data}} \textendash{} input data to be cliped

\item {} 
\sphinxstyleliteralstrong{\sphinxupquote{sigma}} \textendash{} sigma value of sigmaclip

\end{itemize}

\item[{Returns}] \leavevmode
\sphinxstyleemphasis{sigma\_clip\_mask} \textendash{} updated mask

\end{description}\end{quote}

\end{fulllineitems}

\index{sigma\_clip\_data() (TSOSuite method)@\spxentry{sigma\_clip\_data()}\spxextra{TSOSuite method}}

\begin{fulllineitems}
\phantomsection\label{\detokenize{cascade.TSO:cascade.TSO.TSO.TSOSuite.sigma_clip_data}}\pysiglinewithargsret{\sphinxbfcode{\sphinxupquote{sigma\_clip\_data}}}{}{}
Perform sigma clip on science data to flag bad data.

\end{fulllineitems}

\index{create\_cleaned\_dataset() (TSOSuite method)@\spxentry{create\_cleaned\_dataset()}\spxextra{TSOSuite method}}

\begin{fulllineitems}
\phantomsection\label{\detokenize{cascade.TSO:cascade.TSO.TSO.TSOSuite.create_cleaned_dataset}}\pysiglinewithargsret{\sphinxbfcode{\sphinxupquote{create\_cleaned\_dataset}}}{}{}
Create a cleaned dataset to be used in regresion analysis.

\end{fulllineitems}

\index{define\_eclipse\_model() (TSOSuite method)@\spxentry{define\_eclipse\_model()}\spxextra{TSOSuite method}}

\begin{fulllineitems}
\phantomsection\label{\detokenize{cascade.TSO:cascade.TSO.TSO.TSOSuite.define_eclipse_model}}\pysiglinewithargsret{\sphinxbfcode{\sphinxupquote{define\_eclipse\_model}}}{}{}
This function defines the light curve model used to analize the
transit or eclipse. We define both the actual trasit/eclipse signal
as wel as an calibration signal.

\end{fulllineitems}

\index{determine\_source\_position() (TSOSuite method)@\spxentry{determine\_source\_position()}\spxextra{TSOSuite method}}

\begin{fulllineitems}
\phantomsection\label{\detokenize{cascade.TSO:cascade.TSO.TSO.TSOSuite.determine_source_position}}\pysiglinewithargsret{\sphinxbfcode{\sphinxupquote{determine\_source\_position}}}{}{}
This function determines the position of the source in the slit
over time and the spectral trace.

We check if trace and position are already set, if not, determine them
from the data by deriving the “center of light” of the dispersed
light.
\begin{quote}\begin{description}
\item[{Parameters}] \leavevmode\begin{itemize}
\item {} 
\sphinxstyleliteralstrong{\sphinxupquote{spectral\_trace}} \textendash{} The trace of the dispersed light on the detector normalized
to its median position. In case the data are extracted spectra,
the trace is zero.

\item {} 
\sphinxstyleliteralstrong{\sphinxupquote{position}} \textendash{} Postion of the source on the detector in the cross dispersion
directon as a function of time, normalized to the
median position.

\item {} 
\sphinxstyleliteralstrong{\sphinxupquote{median\_position}} \textendash{} median source position.

\end{itemize}

\item[{Returns}] \leavevmode
\begin{itemize}
\item {} 
\sphinxstyleemphasis{position} \textendash{} position of source in time

\item {} 
\sphinxstyleemphasis{med\_position} \textendash{} median (in time) position of source

\end{itemize}


\end{description}\end{quote}

\end{fulllineitems}

\index{set\_extraction\_mask() (TSOSuite method)@\spxentry{set\_extraction\_mask()}\spxextra{TSOSuite method}}

\begin{fulllineitems}
\phantomsection\label{\detokenize{cascade.TSO:cascade.TSO.TSO.TSOSuite.set_extraction_mask}}\pysiglinewithargsret{\sphinxbfcode{\sphinxupquote{set\_extraction\_mask}}}{}{}
Set mask which defines the area of interest within which
a transit signal will be determined. The mask is set along the
spectral trace with a pixel width of nExtractionWidth
\begin{quote}\begin{description}
\item[{Returns}] \leavevmode
\sphinxstyleemphasis{mask} \textendash{} In case data are Spectra : 1D mask
In case data are Spectral images or cubes: 2D mask

\end{description}\end{quote}

\end{fulllineitems}

\index{\_create\_edge\_mask() (TSOSuite method)@\spxentry{\_create\_edge\_mask()}\spxextra{TSOSuite method}}

\begin{fulllineitems}
\phantomsection\label{\detokenize{cascade.TSO:cascade.TSO.TSO.TSOSuite._create_edge_mask}}\pysiglinewithargsret{\sphinxbfcode{\sphinxupquote{\_create\_edge\_mask}}}{\emph{kernel}, \emph{roi\_mask\_cube}}{}
Create an edge mask to mask all pixels for which the convolution kernel
extends beyond the ROI
:param kernel:
:param roi\_mask:
\begin{quote}\begin{description}
\item[{Returns}] \leavevmode
\sphinxstyleemphasis{edge\_mask}

\end{description}\end{quote}

\end{fulllineitems}

\index{\_create\_extraction\_profile() (TSOSuite method)@\spxentry{\_create\_extraction\_profile()}\spxextra{TSOSuite method}}

\begin{fulllineitems}
\phantomsection\label{\detokenize{cascade.TSO:cascade.TSO.TSO.TSOSuite._create_extraction_profile}}\pysiglinewithargsret{\sphinxbfcode{\sphinxupquote{\_create\_extraction\_profile}}}{\emph{cleaned\_data\_with\_roi\_mask}, \emph{extracted\_spectra}, \emph{kernel}, \emph{mask\_for\_extraction}}{}
Create the normilzed source profile used for optimal extraction.
The cleaned data is convolved with an appropriate kernel to smooth the
profile and to increase the SNR. On the edges, where the kernel extends
over the boundary, non convolved data is used to prevent edge effects
\begin{quote}\begin{description}
\item[{Parameters}] \leavevmode\begin{itemize}
\item {} 
\sphinxstyleliteralstrong{\sphinxupquote{cleaned\_data\_with\_roi\_mask}} \textendash{} 

\item {} 
\sphinxstyleliteralstrong{\sphinxupquote{extracted\_spectra}} \textendash{} 

\item {} 
\sphinxstyleliteralstrong{\sphinxupquote{kernel}} \textendash{} 

\item {} 
\sphinxstyleliteralstrong{\sphinxupquote{mask\_for\_extraction}} \textendash{} 

\end{itemize}

\item[{Returns}] \leavevmode
\sphinxstyleemphasis{extraction\_profile}

\end{description}\end{quote}

\end{fulllineitems}

\index{\_create\_3dKernel() (TSOSuite method)@\spxentry{\_create\_3dKernel()}\spxextra{TSOSuite method}}

\begin{fulllineitems}
\phantomsection\label{\detokenize{cascade.TSO:cascade.TSO.TSO.TSOSuite._create_3dKernel}}\pysiglinewithargsret{\sphinxbfcode{\sphinxupquote{\_create\_3dKernel}}}{\emph{sigma\_time}}{}
Create a 3d Kernel from 2d Instrument specific Kernel
to include time dimention
\begin{quote}\begin{description}
\item[{Parameters}] \leavevmode
\sphinxstyleliteralstrong{\sphinxupquote{sigma\_time}} \textendash{} 

\item[{Returns}] \leavevmode
\sphinxstyleemphasis{3dKernel}

\end{description}\end{quote}

\end{fulllineitems}

\index{optimal\_extraction() (TSOSuite method)@\spxentry{optimal\_extraction()}\spxextra{TSOSuite method}}

\begin{fulllineitems}
\phantomsection\label{\detokenize{cascade.TSO:cascade.TSO.TSO.TSOSuite.optimal_extraction}}\pysiglinewithargsret{\sphinxbfcode{\sphinxupquote{optimal\_extraction}}}{}{}
Optimally extract spectrum using procedure of
Horne 1986, PASP 98, 609
\begin{quote}\begin{description}
\item[{Returns}] \leavevmode
\sphinxstyleemphasis{Optimally extracted 1d Spectra}

\end{description}\end{quote}

\end{fulllineitems}

\index{select\_regressors() (TSOSuite method)@\spxentry{select\_regressors()}\spxextra{TSOSuite method}}

\begin{fulllineitems}
\phantomsection\label{\detokenize{cascade.TSO:cascade.TSO.TSO.TSOSuite.select_regressors}}\pysiglinewithargsret{\sphinxbfcode{\sphinxupquote{select\_regressors}}}{}{}
Select pixels which will be used as regressors.
\begin{description}
\item[{List of regressors, using the following list index:}] \leavevmode
first index: {[}\# nod{]}
second index: {[}\# valid pixel in extraction mask{]}
third index: {[}0=pixel coord; 1=list of regressors{]}
forth index: {[}0=coordinate wave direction;
\begin{quote}

1=coordinate spatial direction{]}
\end{quote}

\end{description}

\end{fulllineitems}

\index{get\_design\_matrix() (TSOSuite static method)@\spxentry{get\_design\_matrix()}\spxextra{TSOSuite static method}}

\begin{fulllineitems}
\phantomsection\label{\detokenize{cascade.TSO:cascade.TSO.TSO.TSOSuite.get_design_matrix}}\pysiglinewithargsret{\sphinxbfcode{\sphinxupquote{static }}\sphinxbfcode{\sphinxupquote{get\_design\_matrix}}}{\emph{cleaned\_data\_in}, \emph{original\_mask\_in}, \emph{regressor\_selection}, \emph{nrebin}, \emph{clip=False}, \emph{clip\_pctl\_time=0.0}, \emph{clip\_pctl\_regressors=0.0}}{}
Return the design matrix based on the data set itself
\begin{quote}
\begin{description}
\item[{cleaned\_data\_in}] \leavevmode
time series data with bad pixels corrected

\item[{original\_mask\_in}] \leavevmode
data mask before cleaning

\end{description}

regressor\_selection
nrebin
clip
clip\_pctl\_time
clip\_pctl\_regressors
\end{quote}
\begin{quote}

Design Matirx
\end{quote}

\end{fulllineitems}

\index{reshape\_data() (TSOSuite method)@\spxentry{reshape\_data()}\spxextra{TSOSuite method}}

\begin{fulllineitems}
\phantomsection\label{\detokenize{cascade.TSO:cascade.TSO.TSO.TSOSuite.reshape_data}}\pysiglinewithargsret{\sphinxbfcode{\sphinxupquote{reshape\_data}}}{\emph{data\_in}}{}
Reshape the time series data to a uniform dimentional shape

\end{fulllineitems}

\index{return\_all\_design\_matrices() (TSOSuite method)@\spxentry{return\_all\_design\_matrices()}\spxextra{TSOSuite method}}

\begin{fulllineitems}
\phantomsection\label{\detokenize{cascade.TSO:cascade.TSO.TSO.TSOSuite.return_all_design_matrices}}\pysiglinewithargsret{\sphinxbfcode{\sphinxupquote{return\_all\_design\_matrices}}}{\emph{clip=False}, \emph{clip\_pctl\_time=0.0}, \emph{clip\_pctl\_regressors=0.0}}{}
Setup the regression matrix based on the sub set of the data slected
to be used as calibrators.
\begin{description}
\item[{list with design matrici with the following index convention:}] \leavevmode
first index: {[}\# nods{]}
second index : {[}\# of valid pixels within extraction mask{]}
third index : {[}0{]}

\end{description}

\end{fulllineitems}

\index{calibrate\_timeseries() (TSOSuite method)@\spxentry{calibrate\_timeseries()}\spxextra{TSOSuite method}}

\begin{fulllineitems}
\phantomsection\label{\detokenize{cascade.TSO:cascade.TSO.TSO.TSOSuite.calibrate_timeseries}}\pysiglinewithargsret{\sphinxbfcode{\sphinxupquote{calibrate\_timeseries}}}{}{}
Calibrate the ligth curve data

\end{fulllineitems}

\index{extract\_spectrum() (TSOSuite method)@\spxentry{extract\_spectrum()}\spxextra{TSOSuite method}}

\begin{fulllineitems}
\phantomsection\label{\detokenize{cascade.TSO:cascade.TSO.TSO.TSOSuite.extract_spectrum}}\pysiglinewithargsret{\sphinxbfcode{\sphinxupquote{extract\_spectrum}}}{}{}
Extract the planetary spectrum from the calibrated ligth curve data

\end{fulllineitems}

\index{correct\_extracted\_spectrum() (TSOSuite method)@\spxentry{correct\_extracted\_spectrum()}\spxextra{TSOSuite method}}

\begin{fulllineitems}
\phantomsection\label{\detokenize{cascade.TSO:cascade.TSO.TSO.TSOSuite.correct_extracted_spectrum}}\pysiglinewithargsret{\sphinxbfcode{\sphinxupquote{correct\_extracted\_spectrum}}}{}{}
Make correction for non-uniform subtraction of transit signal due to
differences in the relative weighting of the regressors

\end{fulllineitems}

\index{save\_results() (TSOSuite method)@\spxentry{save\_results()}\spxextra{TSOSuite method}}

\begin{fulllineitems}
\phantomsection\label{\detokenize{cascade.TSO:cascade.TSO.TSO.TSOSuite.save_results}}\pysiglinewithargsret{\sphinxbfcode{\sphinxupquote{save\_results}}}{}{}
Save results

\end{fulllineitems}

\index{plot\_results() (TSOSuite method)@\spxentry{plot\_results()}\spxextra{TSOSuite method}}

\begin{fulllineitems}
\phantomsection\label{\detokenize{cascade.TSO:cascade.TSO.TSO.TSOSuite.plot_results}}\pysiglinewithargsret{\sphinxbfcode{\sphinxupquote{plot\_results}}}{}{}
Plot the extracted planetary spectrum and scaled signal on the
detector.

\end{fulllineitems}


\end{fulllineitems}



\subsection{The cascade.cpm\_model module}
\label{\detokenize{cascade.cpm_model:module-cascade.cpm_model.cpm_model}}\label{\detokenize{cascade.cpm_model:the-cascade-cpm-model-module}}\label{\detokenize{cascade.cpm_model::doc}}\index{cascade.cpm\_model.cpm\_model (module)@\spxentry{cascade.cpm\_model.cpm\_model}\spxextra{module}}
Routines used in causal pixel model

Version 1.0
\begin{description}
\item[{@author: Jeroen Bouwman}] \leavevmode
MPIA Heidelberg

\end{description}
\index{solve\_linear\_equation() (in module cascade.cpm\_model.cpm\_model)@\spxentry{solve\_linear\_equation()}\spxextra{in module cascade.cpm\_model.cpm\_model}}

\begin{fulllineitems}
\phantomsection\label{\detokenize{cascade.cpm_model:cascade.cpm_model.cpm_model.solve_linear_equation}}\pysiglinewithargsret{\sphinxbfcode{\sphinxupquote{solve\_linear\_equation}}}{\emph{design\_matrix}, \emph{data}, \emph{weights=None}, \emph{cv\_method='gcv'}, \emph{reg\_par=\{'lam0': 1e-06}, \emph{'lam1': 100.0}, \emph{'nlam': 60\}}, \emph{feature\_scaling='norm'}, \emph{degrees\_of\_freedom=None}}{}~\begin{quote}

Solve linear system using SVD with TIKHONOV regularization
\end{quote}
\begin{description}
\item[{For details see:}] \leavevmode\begin{quote}
\begin{description}
\item[{PHD thesis by Diana Maria SIMA, “Regularization techniques in}] \leavevmode
Model Fitting and Parameter estimation”, KU Leuven 2006

\end{description}

Hogg et al 2010, “Data analysis recipies: Fitting a model to data”
Rust \& O’Leaary, “Residual periodograms for choosing regularization
\begin{quote}

parameters for ill-posed porblems”
\end{quote}
\begin{description}
\item[{Krakauer et al “Using generalized cross-validationto select parameters}] \leavevmode
in inversions for regional carbon fluxes”

\end{description}
\end{quote}

This routine solves the linear equation

A x = y

by finding optimal solution x\_hat by minimizing

\textbar{}\textbar{}y-A*x\_hat\textbar{}\textbar{}\textasciicircum{}2 + lambda * \textbar{}\textbar{}x\_hat\textbar{}\textbar{}\textasciicircum{}2

Input parameters:
\begin{quote}
\begin{description}
\item[{design\_matrx:}] \leavevmode
Design matrix

\item[{data:}] \leavevmode
Data

\item[{weights:}] \leavevmode
Weights used in the linear least square minimization

\item[{cv\_method:}] \leavevmode\begin{description}
\item[{Method used to find optimal regularization parameter which can be:}] \leavevmode
“gvc” :  Generizalize Cross Validation {[}RECOMMENDED!!!{]}
“b95” :  normalized cumulatative periodogram using 95\% limit
“B100”:  normalized cumulatative periodogram

\end{description}

\item[{reg\_par:}] \leavevmode
Parameter describing search grid to find optimal regularization
parameter lambda:
\begin{quote}

lam0 : minimum lambda
lam1 : maximum lambda
nlam : number of grid points
\end{quote}

\item[{feature\_scaling:}] \leavevmode
norm : normalise features using L2 norm
None : no featue scaling

\end{description}
\end{quote}

\end{description}

\end{fulllineitems}

\index{return\_PCR() (in module cascade.cpm\_model.cpm\_model)@\spxentry{return\_PCR()}\spxextra{in module cascade.cpm\_model.cpm\_model}}

\begin{fulllineitems}
\phantomsection\label{\detokenize{cascade.cpm_model:cascade.cpm_model.cpm_model.return_PCR}}\pysiglinewithargsret{\sphinxbfcode{\sphinxupquote{return\_PCR}}}{\emph{design\_matrix}, \emph{n\_components=None}, \emph{variance\_prior\_scaling=1.0}}{}
Perform principal component regression with marginalization.
To marginalize over the eigen-lightcurves we need to solve
x = (A.T V\textasciicircum{}(-1) A)\textasciicircum{}(-1) * (A.T V\textasciicircum{}(-1) y), where V = C + B.T Lambda B,
with B matrix containing the eigenlightcurves and lambda
the median squared amplitudes of the eigenlightcurves.

\end{fulllineitems}



\subsection{The cascade.data\_model module}
\label{\detokenize{cascade.data_model:module-cascade.data_model.data_model}}\label{\detokenize{cascade.data_model:the-cascade-data-model-module}}\label{\detokenize{cascade.data_model::doc}}\index{cascade.data\_model.data\_model (module)@\spxentry{cascade.data\_model.data\_model}\spxextra{module}}
Created on Sun Jul 10 16:14:19 2016

@author: bouwman
\index{SpectralData (class in cascade.data\_model.data\_model)@\spxentry{SpectralData}\spxextra{class in cascade.data\_model.data\_model}}

\begin{fulllineitems}
\phantomsection\label{\detokenize{cascade.data_model:cascade.data_model.data_model.SpectralData}}\pysiglinewithargsret{\sphinxbfcode{\sphinxupquote{class }}\sphinxbfcode{\sphinxupquote{SpectralData}}}{\emph{wavelength=nan}, \emph{wavelength\_unit=None}, \emph{data=nan}, \emph{data\_unit=None}, \emph{uncertainty=nan}, \emph{mask=False}, \emph{**kwargs}}{}
Bases: \sphinxcode{\sphinxupquote{cascade.data\_model.data\_model.InstanceDescriptorMixin}}

Class defining basic properties of spectral data
INPUT, required:
—————\textendash{}
wavelength
\begin{quote}

wavelength of data (can be frequencies)
\end{quote}
\begin{description}
\item[{wavelenth\_unit}] \leavevmode
The physical unit of the wavelength (uses astropy.units)

\item[{data}] \leavevmode
spectral data

\item[{data\_unit}] \leavevmode
the physical unit of the data (uses astropy.units)

\item[{uncertainty}] \leavevmode
uncertainty on spectral data

\item[{mask}] \leavevmode
mask defining masked data

\end{description}
\begin{description}
\item[{{\color{red}\bfseries{}**}kwargs}] \leavevmode
any auxilary data relevant to the spectral data
(like position, detector temperature etc.)
If unit is not explicitly given a unit atribute is added.
Input argument can be instance of astropy quantity.
Auxilary atributes are added to instance of the SpectralData class
and not to the class itself. Only the required input stated above
is always defined for all instances.

\end{description}

Instance of SpectralData class.
All data are stored internally as numppy arrays. Outputted data are
astropy Quantities unless no units (=None) are specified.
\index{wavelength (SpectralData attribute)@\spxentry{wavelength}\spxextra{SpectralData attribute}}

\begin{fulllineitems}
\phantomsection\label{\detokenize{cascade.data_model:cascade.data_model.data_model.SpectralData.wavelength}}\pysigline{\sphinxbfcode{\sphinxupquote{wavelength}}}
\end{fulllineitems}

\index{wavelength\_unit (SpectralData attribute)@\spxentry{wavelength\_unit}\spxextra{SpectralData attribute}}

\begin{fulllineitems}
\phantomsection\label{\detokenize{cascade.data_model:cascade.data_model.data_model.SpectralData.wavelength_unit}}\pysigline{\sphinxbfcode{\sphinxupquote{wavelength\_unit}}}
\end{fulllineitems}

\index{data (SpectralData attribute)@\spxentry{data}\spxextra{SpectralData attribute}}

\begin{fulllineitems}
\phantomsection\label{\detokenize{cascade.data_model:cascade.data_model.data_model.SpectralData.data}}\pysigline{\sphinxbfcode{\sphinxupquote{data}}}
\end{fulllineitems}

\index{uncertainty (SpectralData attribute)@\spxentry{uncertainty}\spxextra{SpectralData attribute}}

\begin{fulllineitems}
\phantomsection\label{\detokenize{cascade.data_model:cascade.data_model.data_model.SpectralData.uncertainty}}\pysigline{\sphinxbfcode{\sphinxupquote{uncertainty}}}
\end{fulllineitems}

\index{data\_unit (SpectralData attribute)@\spxentry{data\_unit}\spxextra{SpectralData attribute}}

\begin{fulllineitems}
\phantomsection\label{\detokenize{cascade.data_model:cascade.data_model.data_model.SpectralData.data_unit}}\pysigline{\sphinxbfcode{\sphinxupquote{data\_unit}}}
\end{fulllineitems}

\index{mask (SpectralData attribute)@\spxentry{mask}\spxextra{SpectralData attribute}}

\begin{fulllineitems}
\phantomsection\label{\detokenize{cascade.data_model:cascade.data_model.data_model.SpectralData.mask}}\pysigline{\sphinxbfcode{\sphinxupquote{mask}}}
\end{fulllineitems}


\end{fulllineitems}

\index{SpectralDataTimeSeries (class in cascade.data\_model.data\_model)@\spxentry{SpectralDataTimeSeries}\spxextra{class in cascade.data\_model.data\_model}}

\begin{fulllineitems}
\phantomsection\label{\detokenize{cascade.data_model:cascade.data_model.data_model.SpectralDataTimeSeries}}\pysiglinewithargsret{\sphinxbfcode{\sphinxupquote{class }}\sphinxbfcode{\sphinxupquote{SpectralDataTimeSeries}}}{\emph{wavelength=nan}, \emph{wavelength\_unit=None}, \emph{data=array({[}{[}nan{]}{]})}, \emph{data\_unit=None}, \emph{uncertainty=array({[}{[}nan{]}{]})}, \emph{mask=False}, \emph{time=nan}, \emph{time\_unit=None}, \emph{**kwargs}}{}
Bases: {\hyperref[\detokenize{cascade.data_model:cascade.data_model.data_model.SpectralData}]{\sphinxcrossref{\sphinxcode{\sphinxupquote{cascade.data\_model.data\_model.SpectralData}}}}}

Class defining timeseries of spectral data. This class inherits from
SpectralData. The data now has one additional dimension: time
Input:
—\textendash{}
time
\begin{quote}

time of observation
\end{quote}
\begin{description}
\item[{time\_unit}] \leavevmode
physical unit of time data

\end{description}
\index{time (SpectralDataTimeSeries attribute)@\spxentry{time}\spxextra{SpectralDataTimeSeries attribute}}

\begin{fulllineitems}
\phantomsection\label{\detokenize{cascade.data_model:cascade.data_model.data_model.SpectralDataTimeSeries.time}}\pysigline{\sphinxbfcode{\sphinxupquote{time}}}
\end{fulllineitems}

\index{time\_unit (SpectralDataTimeSeries attribute)@\spxentry{time\_unit}\spxextra{SpectralDataTimeSeries attribute}}

\begin{fulllineitems}
\phantomsection\label{\detokenize{cascade.data_model:cascade.data_model.data_model.SpectralDataTimeSeries.time_unit}}\pysigline{\sphinxbfcode{\sphinxupquote{time\_unit}}}
\end{fulllineitems}


\end{fulllineitems}



\subsection{The cascade.exoplanet\_tools module}
\label{\detokenize{cascade.exoplanet_tools:module-cascade.exoplanet_tools.exoplanet_tools}}\label{\detokenize{cascade.exoplanet_tools:the-cascade-exoplanet-tools-module}}\label{\detokenize{cascade.exoplanet_tools::doc}}\index{cascade.exoplanet\_tools.exoplanet\_tools (module)@\spxentry{cascade.exoplanet\_tools.exoplanet\_tools}\spxextra{module}}
This Module defines the functionality to get catalog data on the targeted
exoplanet and define the model ligth curve for the system.
It also difines some usefull functionality for exoplanet atmosphere analysis.
\index{Kmag (in module cascade.exoplanet\_tools.exoplanet\_tools)@\spxentry{Kmag}\spxextra{in module cascade.exoplanet\_tools.exoplanet\_tools}}

\begin{fulllineitems}
\phantomsection\label{\detokenize{cascade.exoplanet_tools:cascade.exoplanet_tools.exoplanet_tools.Kmag}}\pysigline{\sphinxbfcode{\sphinxupquote{Kmag}}\sphinxbfcode{\sphinxupquote{ = Unit("Kmag")}}}
Definition of generic K band magnitude

\end{fulllineitems}

\index{Vmag (in module cascade.exoplanet\_tools.exoplanet\_tools)@\spxentry{Vmag}\spxextra{in module cascade.exoplanet\_tools.exoplanet\_tools}}

\begin{fulllineitems}
\phantomsection\label{\detokenize{cascade.exoplanet_tools:cascade.exoplanet_tools.exoplanet_tools.Vmag}}\pysigline{\sphinxbfcode{\sphinxupquote{Vmag}}\sphinxbfcode{\sphinxupquote{ = Unit("Vmag")}}}
Definition of a generic Vband magnitude

\end{fulllineitems}

\index{masked\_array\_input() (in module cascade.exoplanet\_tools.exoplanet\_tools)@\spxentry{masked\_array\_input()}\spxextra{in module cascade.exoplanet\_tools.exoplanet\_tools}}

\begin{fulllineitems}
\phantomsection\label{\detokenize{cascade.exoplanet_tools:cascade.exoplanet_tools.exoplanet_tools.masked_array_input}}\pysiglinewithargsret{\sphinxbfcode{\sphinxupquote{masked\_array\_input}}}{\emph{func}}{}
Decorator function to check and handel masked Quantities

If one of the input arguments is wavelength or flux, the array can be
a masked Quantity, masking out only ‘bad’ data. This decorator checks for
masked arrays and upon finding the first masked array, passes the data
and stores the mask to be used to create a masked Quantity after the
function returns.
\begin{quote}\begin{description}
\item[{Parameters}] \leavevmode
\sphinxstyleliteralstrong{\sphinxupquote{func}} (\sphinxstyleliteralemphasis{\sphinxupquote{method}}) \textendash{} Function to be decorated

\end{description}\end{quote}

\end{fulllineitems}

\index{KmagToJy() (in module cascade.exoplanet\_tools.exoplanet\_tools)@\spxentry{KmagToJy()}\spxextra{in module cascade.exoplanet\_tools.exoplanet\_tools}}

\begin{fulllineitems}
\phantomsection\label{\detokenize{cascade.exoplanet_tools:cascade.exoplanet_tools.exoplanet_tools.KmagToJy}}\pysiglinewithargsret{\sphinxbfcode{\sphinxupquote{KmagToJy}}}{\emph{magnitude: Unit("Kmag")}, \emph{system='Johnson'}}{}
Convert Kband Magnitudes to Jy
\begin{quote}\begin{description}
\item[{Parameters}] \leavevmode\begin{itemize}
\item {} 
\sphinxstyleliteralstrong{\sphinxupquote{magnitude}} (\sphinxstyleliteralemphasis{\sphinxupquote{'Kmag'}}) \textendash{} Input K band magnitude to be converted to Jy.

\item {} 
\sphinxstyleliteralstrong{\sphinxupquote{system}} (\sphinxstyleliteralemphasis{\sphinxupquote{'str'}}) \textendash{} optional, either ‘Johnson’ or ‘2MASS’, default is ‘Johnson’

\end{itemize}

\item[{Returns}] \leavevmode
\sphinxstylestrong{flux} (\sphinxstyleemphasis{‘astropy.units.Quantity’, u.Jy}) \textendash{} Flux in Jy, converted from input Kband magnitude

\item[{Raises}] \leavevmode
\sphinxcode{\sphinxupquote{AssertionError}} \textendash{} raises error if Photometric system not recognized

\end{description}\end{quote}

\end{fulllineitems}

\index{JytoKmag() (in module cascade.exoplanet\_tools.exoplanet\_tools)@\spxentry{JytoKmag()}\spxextra{in module cascade.exoplanet\_tools.exoplanet\_tools}}

\begin{fulllineitems}
\phantomsection\label{\detokenize{cascade.exoplanet_tools:cascade.exoplanet_tools.exoplanet_tools.JytoKmag}}\pysiglinewithargsret{\sphinxbfcode{\sphinxupquote{JytoKmag}}}{\emph{flux: Unit("Jy")}, \emph{system='Johnson'}}{}
Convert flux in Jy to Kband Magnitudes
\begin{quote}\begin{description}
\item[{Parameters}] \leavevmode\begin{itemize}
\item {} 
\sphinxstyleliteralstrong{\sphinxupquote{flux}} (\sphinxstyleliteralemphasis{\sphinxupquote{'astropy.units.Quantity'}}\sphinxstyleliteralemphasis{\sphinxupquote{, }}\sphinxstyleliteralemphasis{\sphinxupquote{'u.Jy}}\sphinxstyleliteralemphasis{\sphinxupquote{ or }}\sphinxstyleliteralemphasis{\sphinxupquote{equivalent'}}) \textendash{} Input Flux to be converted K band magnitude.

\item {} 
\sphinxstyleliteralstrong{\sphinxupquote{system}} (\sphinxstyleliteralemphasis{\sphinxupquote{'str'}}) \textendash{} optional, either ‘Johnson’ or ‘2MASS’, default is ‘Johnson’

\end{itemize}

\item[{Returns}] \leavevmode
\sphinxstylestrong{magnitude} (\sphinxstyleemphasis{‘astropy.units.Quantity’, Kmag}) \textendash{} Magnitude  converted from input fkux value

\item[{Raises}] \leavevmode
\sphinxcode{\sphinxupquote{AssertionError}} \textendash{} raises error if Photometric system not recognized

\end{description}\end{quote}

\end{fulllineitems}

\index{Planck() (in module cascade.exoplanet\_tools.exoplanet\_tools)@\spxentry{Planck()}\spxextra{in module cascade.exoplanet\_tools.exoplanet\_tools}}

\begin{fulllineitems}
\phantomsection\label{\detokenize{cascade.exoplanet_tools:cascade.exoplanet_tools.exoplanet_tools.Planck}}\pysiglinewithargsret{\sphinxbfcode{\sphinxupquote{Planck}}}{\emph{wavelength: Unit("micron")}, \emph{temperature: Unit("K")}}{}
This function calculates the emisison from a Black Body.
\begin{quote}\begin{description}
\item[{Parameters}] \leavevmode\begin{itemize}
\item {} 
\sphinxstyleliteralstrong{\sphinxupquote{wavelength}} (\sphinxstyleliteralemphasis{\sphinxupquote{'astropy.units.Quantity'}}) \textendash{} Input wavelength in units of microns or equivalent

\item {} 
\sphinxstyleliteralstrong{\sphinxupquote{temperature}} (\sphinxstyleliteralemphasis{\sphinxupquote{'astropy.units.Quantity'}}) \textendash{} Input temperature in units of Kelvin or equivalent

\end{itemize}

\item[{Returns}] \leavevmode
\sphinxstylestrong{blackbody} (\sphinxstyleemphasis{‘astropy.units.Quantity’}) \textendash{} B\_nu in cgs units {[} erg/s/cm2/Hz/sr{]}

\end{description}\end{quote}

\end{fulllineitems}

\index{SurfaceGravity() (in module cascade.exoplanet\_tools.exoplanet\_tools)@\spxentry{SurfaceGravity()}\spxextra{in module cascade.exoplanet\_tools.exoplanet\_tools}}

\begin{fulllineitems}
\phantomsection\label{\detokenize{cascade.exoplanet_tools:cascade.exoplanet_tools.exoplanet_tools.SurfaceGravity}}\pysiglinewithargsret{\sphinxbfcode{\sphinxupquote{SurfaceGravity}}}{\emph{MassPlanet: Unit("jupiterMass")}, \emph{RadiusPlanet: Unit("jupiterRad")}}{}
Calculates surface gravity of planet
\begin{quote}\begin{description}
\item[{Parameters}] \leavevmode\begin{itemize}
\item {} 
\sphinxstyleliteralstrong{\sphinxupquote{MassPlanet}} \textendash{} Mass of planet in units of Jupiter mass or equivalent

\item {} 
\sphinxstyleliteralstrong{\sphinxupquote{RadiusPlanet}} \textendash{} Radius of planet in units of Jupiter radius or equivalent

\end{itemize}

\item[{Returns}] \leavevmode
\sphinxstyleemphasis{sgrav} \textendash{} Surface gravity in units of  m s-2

\end{description}\end{quote}

\end{fulllineitems}

\index{ScaleHeight() (in module cascade.exoplanet\_tools.exoplanet\_tools)@\spxentry{ScaleHeight()}\spxextra{in module cascade.exoplanet\_tools.exoplanet\_tools}}

\begin{fulllineitems}
\phantomsection\label{\detokenize{cascade.exoplanet_tools:cascade.exoplanet_tools.exoplanet_tools.ScaleHeight}}\pysiglinewithargsret{\sphinxbfcode{\sphinxupquote{ScaleHeight}}}{\emph{MeanMolecularMass: Unit("u")}, \emph{SurfaceGravity: Unit("m / s2")}, \emph{Temperature: Unit("K")}}{}
Calculate the scaleheigth of the planet
\begin{quote}\begin{description}
\item[{Parameters}] \leavevmode\begin{itemize}
\item {} 
\sphinxstyleliteralstrong{\sphinxupquote{MeanMolecularMass}} (\sphinxstyleliteralemphasis{\sphinxupquote{'astropy.units.Quantity'}}) \textendash{} in units of mass of the hydrogen atom or equivalent

\item {} 
\sphinxstyleliteralstrong{\sphinxupquote{SurfaceGravity}} (\sphinxstyleliteralemphasis{\sphinxupquote{'astropy.units.Quantity'}}) \textendash{} in units of m s-2 or equivalent

\item {} 
\sphinxstyleliteralstrong{\sphinxupquote{Temperature}} (\sphinxstyleliteralemphasis{\sphinxupquote{'astropy.units.Quantity'}}) \textendash{} in units of K or equivalent

\end{itemize}

\item[{Returns}] \leavevmode
\sphinxstylestrong{ScaleHeight} (\sphinxstyleemphasis{‘astropy.units.Quantity’}) \textendash{} scaleheigth in unit of km

\end{description}\end{quote}

\end{fulllineitems}

\index{TransitDepth() (in module cascade.exoplanet\_tools.exoplanet\_tools)@\spxentry{TransitDepth()}\spxextra{in module cascade.exoplanet\_tools.exoplanet\_tools}}

\begin{fulllineitems}
\phantomsection\label{\detokenize{cascade.exoplanet_tools:cascade.exoplanet_tools.exoplanet_tools.TransitDepth}}\pysiglinewithargsret{\sphinxbfcode{\sphinxupquote{TransitDepth}}}{\emph{RadiusPlanet: Unit("jupiterRad")}, \emph{RadiusStar: Unit("solRad")}}{}
Calculates the depth of the planetary transit assuming one can
neglect the emision from the night side of the planet.
\begin{quote}\begin{description}
\item[{Parameters}] \leavevmode\begin{itemize}
\item {} 
\sphinxstyleliteralstrong{\sphinxupquote{Planet}} (\sphinxstyleliteralemphasis{\sphinxupquote{Radius}}) \textendash{} Planetary radius in Jovian radii or equivalent

\item {} 
\sphinxstyleliteralstrong{\sphinxupquote{Star}} (\sphinxstyleliteralemphasis{\sphinxupquote{Radius}}) \textendash{} Stellar radius in Solar radii or equivalent

\end{itemize}

\item[{Returns}] \leavevmode
\sphinxstyleemphasis{depth} \textendash{} Relative transit depth (unit less)

\end{description}\end{quote}

\end{fulllineitems}

\index{EquilibriumTemperature() (in module cascade.exoplanet\_tools.exoplanet\_tools)@\spxentry{EquilibriumTemperature()}\spxextra{in module cascade.exoplanet\_tools.exoplanet\_tools}}

\begin{fulllineitems}
\phantomsection\label{\detokenize{cascade.exoplanet_tools:cascade.exoplanet_tools.exoplanet_tools.EquilibriumTemperature}}\pysiglinewithargsret{\sphinxbfcode{\sphinxupquote{EquilibriumTemperature}}}{\emph{StellarTemperature: Unit("K")}, \emph{StellarRadius: Unit("solRad")}, \emph{SemiMajorAxis: Unit("AU")}, \emph{Albedo=0.3}, \emph{epsilon=0.7}}{}
Calculate the Equlibrium Temperature of the Planet
\begin{quote}\begin{description}
\item[{Parameters}] \leavevmode\begin{itemize}
\item {} 
\sphinxstyleliteralstrong{\sphinxupquote{StellarTemperature}} (\sphinxstyleliteralemphasis{\sphinxupquote{'astropy.units.Quantity'}}) \textendash{} Temperature of the central star in units of K or equivalent

\item {} 
\sphinxstyleliteralstrong{\sphinxupquote{StellarRadius}} (\sphinxstyleliteralemphasis{\sphinxupquote{'astropy.units.Quantity'}}) \textendash{} Radius of the central star in units of Solar Radii or equivalent

\item {} 
\sphinxstyleliteralstrong{\sphinxupquote{Albedo}} (\sphinxstyleliteralemphasis{\sphinxupquote{'float'}}) \textendash{} Albedo of the planet.

\item {} 
\sphinxstyleliteralstrong{\sphinxupquote{SemiMajorAxis}} (\sphinxstyleliteralemphasis{\sphinxupquote{'astropy.units.Quantity'}}) \textendash{} The semi-major axis of platetary orbit in units of AU or equivalent

\item {} 
\sphinxstyleliteralstrong{\sphinxupquote{epsilon}} (\sphinxstyleliteralemphasis{\sphinxupquote{'float'}}) \textendash{} Green house effect parameter

\end{itemize}

\item[{Returns}] \leavevmode
\sphinxstylestrong{ET} (\sphinxstyleemphasis{‘astropy.units.Quantity’}) \textendash{} Equlibrium Temperature of the exoplanet

\end{description}\end{quote}

\end{fulllineitems}

\index{convert\_spectrum\_to\_brighness\_temperature() (in module cascade.exoplanet\_tools.exoplanet\_tools)@\spxentry{convert\_spectrum\_to\_brighness\_temperature()}\spxextra{in module cascade.exoplanet\_tools.exoplanet\_tools}}

\begin{fulllineitems}
\phantomsection\label{\detokenize{cascade.exoplanet_tools:cascade.exoplanet_tools.exoplanet_tools.convert_spectrum_to_brighness_temperature}}\pysiglinewithargsret{\sphinxbfcode{\sphinxupquote{convert\_spectrum\_to\_brighness\_temperature}}}{\emph{wavelength: Unit("micron")}, \emph{contrast: Unit("\%")}, \emph{StellarTemperature: Unit("K")}, \emph{StellarRadius: Unit("solRad")}, \emph{RadiusPlanet: Unit("jupiterRad")}, \emph{error: Unit("\%") = None}}{}
Function to convert the secondary eclipse spectrum to brightness
temperature.
\begin{quote}\begin{description}
\item[{Parameters}] \leavevmode\begin{itemize}
\item {} 
\sphinxstyleliteralstrong{\sphinxupquote{wavelength}} \textendash{} Wavelength in u.micron or equivalent unit.

\item {} 
\sphinxstyleliteralstrong{\sphinxupquote{contrast}} \textendash{} Contrast between planet and star in u.percent.

\item {} 
\sphinxstyleliteralstrong{\sphinxupquote{StellarTemperature}} \textendash{} Temperature if the star in u.K or equivalent unit.

\item {} 
\sphinxstyleliteralstrong{\sphinxupquote{StellarRadius}} \textendash{} Radius of the star in u.R\_sun or equivalent unit.

\item {} 
\sphinxstyleliteralstrong{\sphinxupquote{RadiusPlanet}} \textendash{} Radius of the planet in u.R\_jupiter or equivalent unit.

\item {} 
\sphinxstyleliteralstrong{\sphinxupquote{error}} \textendash{} (optional) Error on contrast in u.percent (standart value = None).

\end{itemize}

\item[{Returns}] \leavevmode
\begin{itemize}
\item {} 
\sphinxstyleemphasis{brighness\_temperature} \textendash{} Eclipse spectrum in units of brightness temperature.

\item {} 
\sphinxstyleemphasis{error\_brighness\_temperature} \textendash{} (optional) Error on the spectrum in units of brightness temperature.

\end{itemize}


\end{description}\end{quote}

\end{fulllineitems}

\index{eclipse\_to\_transit() (in module cascade.exoplanet\_tools.exoplanet\_tools)@\spxentry{eclipse\_to\_transit()}\spxextra{in module cascade.exoplanet\_tools.exoplanet\_tools}}

\begin{fulllineitems}
\phantomsection\label{\detokenize{cascade.exoplanet_tools:cascade.exoplanet_tools.exoplanet_tools.eclipse_to_transit}}\pysiglinewithargsret{\sphinxbfcode{\sphinxupquote{eclipse\_to\_transit}}}{\emph{eclipse}}{}
Converts eclipse spectrum to transit spectrum
\begin{quote}\begin{description}
\item[{Parameters}] \leavevmode
\sphinxstyleliteralstrong{\sphinxupquote{eclipse}} \textendash{} Transit depth values to be converted

\item[{Returns}] \leavevmode
\sphinxstyleemphasis{transit} \textendash{} transit depth values derived from input eclipse values

\end{description}\end{quote}

\end{fulllineitems}

\index{transit\_to\_eclipse() (in module cascade.exoplanet\_tools.exoplanet\_tools)@\spxentry{transit\_to\_eclipse()}\spxextra{in module cascade.exoplanet\_tools.exoplanet\_tools}}

\begin{fulllineitems}
\phantomsection\label{\detokenize{cascade.exoplanet_tools:cascade.exoplanet_tools.exoplanet_tools.transit_to_eclipse}}\pysiglinewithargsret{\sphinxbfcode{\sphinxupquote{transit\_to\_eclipse}}}{\emph{transit}}{}
Converts transit spectrum to eclipse spectrum
\begin{quote}\begin{description}
\item[{Parameters}] \leavevmode
\sphinxstyleliteralstrong{\sphinxupquote{transit}} \textendash{} Transit depth values to be converted

\item[{Returns}] \leavevmode
\sphinxstyleemphasis{eclipse} \textendash{} eclipse depth values derived from input transit values

\end{description}\end{quote}

\end{fulllineitems}

\index{combine\_spectra() (in module cascade.exoplanet\_tools.exoplanet\_tools)@\spxentry{combine\_spectra()}\spxextra{in module cascade.exoplanet\_tools.exoplanet\_tools}}

\begin{fulllineitems}
\phantomsection\label{\detokenize{cascade.exoplanet_tools:cascade.exoplanet_tools.exoplanet_tools.combine_spectra}}\pysiglinewithargsret{\sphinxbfcode{\sphinxupquote{combine\_spectra}}}{\emph{identifier\_list={[}{]}}, \emph{path=''}}{}
Convienience function to combine multiple extracted spectra
of the same source by calculating a weighted averige.
\begin{quote}\begin{description}
\item[{Parameters}] \leavevmode\begin{itemize}
\item {} 
\sphinxstyleliteralstrong{\sphinxupquote{identifier\_list}} (\sphinxstyleliteralemphasis{\sphinxupquote{'list' of 'str'}}) \textendash{} List of file identifiers of the individual spectra to be combined

\item {} 
\sphinxstyleliteralstrong{\sphinxupquote{path}} (\sphinxstyleliteralemphasis{\sphinxupquote{'str'}}) \textendash{} path to the fits files

\end{itemize}

\item[{Returns}] \leavevmode
\sphinxstylestrong{combined\_spectrum} (\sphinxstyleemphasis{‘array\_like’}) \textendash{} The combined spectrum based on the spectra specified in the input list

\end{description}\end{quote}

\end{fulllineitems}

\index{get\_calalog() (in module cascade.exoplanet\_tools.exoplanet\_tools)@\spxentry{get\_calalog()}\spxextra{in module cascade.exoplanet\_tools.exoplanet\_tools}}

\begin{fulllineitems}
\phantomsection\label{\detokenize{cascade.exoplanet_tools:cascade.exoplanet_tools.exoplanet_tools.get_calalog}}\pysiglinewithargsret{\sphinxbfcode{\sphinxupquote{get\_calalog}}}{\emph{catalog\_name}, \emph{update=True}}{}
Get exoplanet catalog data
\begin{quote}\begin{description}
\item[{Parameters}] \leavevmode\begin{itemize}
\item {} 
\sphinxstyleliteralstrong{\sphinxupquote{catalog\_name}} (\sphinxstyleliteralemphasis{\sphinxupquote{'str'}}) \textendash{} name of catalog to use

\item {} 
\sphinxstyleliteralstrong{\sphinxupquote{update}} (\sphinxstyleliteralemphasis{\sphinxupquote{'bool'}}) \textendash{} Boolian indicating if local calalog file will be updated

\end{itemize}

\item[{Returns}] \leavevmode
\sphinxstylestrong{files\_downloaded} (\sphinxstyleemphasis{‘list’ of ‘str’}) \textendash{} list of downloaded catalog files

\end{description}\end{quote}

\end{fulllineitems}

\index{parse\_database() (in module cascade.exoplanet\_tools.exoplanet\_tools)@\spxentry{parse\_database()}\spxextra{in module cascade.exoplanet\_tools.exoplanet\_tools}}

\begin{fulllineitems}
\phantomsection\label{\detokenize{cascade.exoplanet_tools:cascade.exoplanet_tools.exoplanet_tools.parse_database}}\pysiglinewithargsret{\sphinxbfcode{\sphinxupquote{parse\_database}}}{\emph{catalog\_name}, \emph{update=True}}{}
Read CSV files containing exoplanet catalog data
\begin{quote}\begin{description}
\item[{Parameters}] \leavevmode\begin{itemize}
\item {} 
\sphinxstyleliteralstrong{\sphinxupquote{catalog\_name}} (\sphinxstyleliteralemphasis{\sphinxupquote{'str'}}) \textendash{} name of catalog to use

\item {} 
\sphinxstyleliteralstrong{\sphinxupquote{update}} (\sphinxstyleliteralemphasis{\sphinxupquote{'bool'}}) \textendash{} Boolian indicating if local calalog file will be updated

\end{itemize}

\item[{Returns}] \leavevmode
\sphinxstylestrong{table\_list} (\sphinxstyleemphasis{‘list’ of ‘astropy.table.Table’}) \textendash{} List containing astropy Tables with the parameters of the exoplanet
systems in the database.

\end{description}\end{quote}

\begin{sphinxadmonition}{note}{Note:}\begin{description}
\item[{The following exoplanet databases can be used:}] \leavevmode
The Transing exoplanet catalog (TEPCAT)
The NASA exoplanet Archive
The Exoplanet Orbit Database

\end{description}
\end{sphinxadmonition}
\begin{quote}\begin{description}
\item[{Raises}] \leavevmode
\sphinxcode{\sphinxupquote{ValueError}} \textendash{} Raises error if the input catalog is nor recognized

\end{description}\end{quote}

\end{fulllineitems}

\index{extract\_exoplanet\_data() (in module cascade.exoplanet\_tools.exoplanet\_tools)@\spxentry{extract\_exoplanet\_data()}\spxextra{in module cascade.exoplanet\_tools.exoplanet\_tools}}

\begin{fulllineitems}
\phantomsection\label{\detokenize{cascade.exoplanet_tools:cascade.exoplanet_tools.exoplanet_tools.extract_exoplanet_data}}\pysiglinewithargsret{\sphinxbfcode{\sphinxupquote{extract\_exoplanet\_data}}}{\emph{data\_list}, \emph{target\_name\_or\_position}, \emph{coord\_unit=None}, \emph{coordinate\_frame='icrs'}, \emph{search\_radius=\textless{}Quantity 5. arcsec\textgreater{}}}{}
Extract the data record for a single target
\begin{quote}\begin{description}
\item[{Parameters}] \leavevmode\begin{itemize}
\item {} 
\sphinxstyleliteralstrong{\sphinxupquote{data\_list}} (\sphinxstyleliteralemphasis{\sphinxupquote{'list' of 'astropy.Table'}}) \textendash{} List containing table with exoplanet data

\item {} 
\sphinxstyleliteralstrong{\sphinxupquote{target\_name\_or\_position}} \textendash{} Name of the target or coordinates of the target for
which the record is extracted

\item {} 
\sphinxstyleliteralstrong{\sphinxupquote{coord\_unit}} \textendash{} Unit of coordinates e.g (u.hourangle, u.deg)

\item {} 
\sphinxstyleliteralstrong{\sphinxupquote{coordinate\_frame}} \textendash{} Frame of coordinate system e.g icrs

\end{itemize}

\item[{Returns}] \leavevmode
\sphinxstylestrong{table\_list} (\sphinxstyleemphasis{‘list’}) \textendash{} List containing data record of the specified planet

\end{description}\end{quote}

\end{fulllineitems}

\index{lightcuve (class in cascade.exoplanet\_tools.exoplanet\_tools)@\spxentry{lightcuve}\spxextra{class in cascade.exoplanet\_tools.exoplanet\_tools}}

\begin{fulllineitems}
\phantomsection\label{\detokenize{cascade.exoplanet_tools:cascade.exoplanet_tools.exoplanet_tools.lightcuve}}\pysigline{\sphinxbfcode{\sphinxupquote{class }}\sphinxbfcode{\sphinxupquote{lightcuve}}}
Bases: \sphinxcode{\sphinxupquote{object}}

Class defining lightcurve model used to model the observed
transit/eclipse observations.
Current valid lightcurve models: batman
\begin{quote}\begin{description}
\item[{Variables}] \leavevmode\begin{itemize}
\item {} 
\sphinxstyleliteralstrong{\sphinxupquote{lc}} (\sphinxstyleliteralemphasis{\sphinxupquote{'array\_like'}}) \textendash{} The lightcurve model

\item {} 
\sphinxstyleliteralstrong{\sphinxupquote{par}} (\sphinxstyleliteralemphasis{\sphinxupquote{'ordered\_dict'}}) \textendash{} The lightcurve parameters

\end{itemize}

\end{description}\end{quote}

\begin{sphinxadmonition}{note}{Notes}

Uses factory method to pick model/package used to calculate
the lightcurve model.
\end{sphinxadmonition}
\begin{quote}\begin{description}
\item[{Raises}] \leavevmode
\sphinxcode{\sphinxupquote{ValueError}} \textendash{} Error is raised if no valid lightcurve model is defined

\end{description}\end{quote}
\index{valid\_models (lightcuve attribute)@\spxentry{valid\_models}\spxextra{lightcuve attribute}}

\begin{fulllineitems}
\phantomsection\label{\detokenize{cascade.exoplanet_tools:cascade.exoplanet_tools.exoplanet_tools.lightcuve.valid_models}}\pysigline{\sphinxbfcode{\sphinxupquote{valid\_models}}\sphinxbfcode{\sphinxupquote{ = \{'batman'\}}}}
\end{fulllineitems}


\end{fulllineitems}

\index{batman\_model (class in cascade.exoplanet\_tools.exoplanet\_tools)@\spxentry{batman\_model}\spxextra{class in cascade.exoplanet\_tools.exoplanet\_tools}}

\begin{fulllineitems}
\phantomsection\label{\detokenize{cascade.exoplanet_tools:cascade.exoplanet_tools.exoplanet_tools.batman_model}}\pysigline{\sphinxbfcode{\sphinxupquote{class }}\sphinxbfcode{\sphinxupquote{batman\_model}}}
Bases: \sphinxcode{\sphinxupquote{object}}

This class defines the lightcurve model used to analyse the observed
transit/eclipse using the batman package by Laura Kreidberg %
\begin{footnote}[1]\sphinxAtStartFootnote
Kreidberg, L. 2015, PASP 127, 1161
%
\end{footnote}.
\begin{quote}\begin{description}
\item[{Variables}] \leavevmode\begin{itemize}
\item {} 
\sphinxstyleliteralstrong{\sphinxupquote{lc}} (\sphinxstyleliteralemphasis{\sphinxupquote{'array\_like'}}) \textendash{} The values of the lightcurve model

\item {} 
\sphinxstyleliteralstrong{\sphinxupquote{par}} (\sphinxstyleliteralemphasis{\sphinxupquote{'ordered\_dict'}}) \textendash{} The model parameters difining the lightcurve model

\end{itemize}

\end{description}\end{quote}

\begin{sphinxadmonition}{note}{References}
\end{sphinxadmonition}
\index{define\_batman\_model() (batman\_model static method)@\spxentry{define\_batman\_model()}\spxextra{batman\_model static method}}

\begin{fulllineitems}
\phantomsection\label{\detokenize{cascade.exoplanet_tools:cascade.exoplanet_tools.exoplanet_tools.batman_model.define_batman_model}}\pysiglinewithargsret{\sphinxbfcode{\sphinxupquote{static }}\sphinxbfcode{\sphinxupquote{define\_batman\_model}}}{\emph{InputParameter}}{}
This function defines the light curve model used to analize the
transit or eclipse. We use the batman package to calculate the
light curves.We assume here a symmetric transit signal, that the
secondary transit is at phase 0.5 and primary transit at 0.0.
\begin{quote}\begin{description}
\item[{Parameters}] \leavevmode
\sphinxstyleliteralstrong{\sphinxupquote{InputParameter}} (\sphinxstyleliteralemphasis{\sphinxupquote{'dict'}}) \textendash{} Ordered dict containing all needed inut parameter to define model

\item[{Returns}] \leavevmode
\begin{itemize}
\item {} 
\sphinxstylestrong{tmodel} (\sphinxstyleemphasis{‘array\_like’}) \textendash{} Orbital phase of planet used for lightcurve model

\item {} 
\sphinxstylestrong{lcmode} (\sphinxstyleemphasis{‘array\_like’}) \textendash{} Normalized values of the lightcurve model

\end{itemize}


\end{description}\end{quote}

\end{fulllineitems}

\index{ReturnParFromIni() (batman\_model method)@\spxentry{ReturnParFromIni()}\spxextra{batman\_model method}}

\begin{fulllineitems}
\phantomsection\label{\detokenize{cascade.exoplanet_tools:cascade.exoplanet_tools.exoplanet_tools.batman_model.ReturnParFromIni}}\pysiglinewithargsret{\sphinxbfcode{\sphinxupquote{ReturnParFromIni}}}{}{}
Get relevant parameters for lightcurve model from CASCADe
intitialization files
\begin{quote}\begin{description}
\item[{Returns}] \leavevmode
\sphinxstylestrong{par} (\sphinxstyleemphasis{‘ordered\_dict’}) \textendash{} input model parameters for batman lightcurve model

\end{description}\end{quote}

\end{fulllineitems}

\index{ReturnParFromDB() (batman\_model method)@\spxentry{ReturnParFromDB()}\spxextra{batman\_model method}}

\begin{fulllineitems}
\phantomsection\label{\detokenize{cascade.exoplanet_tools:cascade.exoplanet_tools.exoplanet_tools.batman_model.ReturnParFromDB}}\pysiglinewithargsret{\sphinxbfcode{\sphinxupquote{ReturnParFromDB}}}{}{}
Get relevant parameters for lightcurve model from exoplanet database
specified in CASCADe initialization file
\begin{quote}\begin{description}
\item[{Returns}] \leavevmode
\sphinxstylestrong{par} (\sphinxstyleemphasis{‘ordered\_dict’}) \textendash{} input model parameters for batman lightcurve model

\item[{Raises}] \leavevmode
\sphinxcode{\sphinxupquote{ValueError}} \textendash{} Raises error in case the observation type is not recognized.

\end{description}\end{quote}

\end{fulllineitems}


\end{fulllineitems}



\subsection{The cascade.initialize module}
\label{\detokenize{cascade.initialize:module-cascade.initialize.initialize}}\label{\detokenize{cascade.initialize:the-cascade-initialize-module}}\label{\detokenize{cascade.initialize::doc}}\index{cascade.initialize.initialize (module)@\spxentry{cascade.initialize.initialize}\spxextra{module}}
This Module defines the functionality to generate and read .ini files which
are used to initialize CASCADe.

Created on Thu Mar 16 21:02:31 2017

@author: bouwman
\index{generate\_default\_initialization() (in module cascade.initialize.initialize)@\spxentry{generate\_default\_initialization()}\spxextra{in module cascade.initialize.initialize}}

\begin{fulllineitems}
\phantomsection\label{\detokenize{cascade.initialize:cascade.initialize.initialize.generate_default_initialization}}\pysiglinewithargsret{\sphinxbfcode{\sphinxupquote{generate\_default\_initialization}}}{}{}
Convenience function to generate example .ini file for CASCADe

\end{fulllineitems}

\index{configurator (class in cascade.initialize.initialize)@\spxentry{configurator}\spxextra{class in cascade.initialize.initialize}}

\begin{fulllineitems}
\phantomsection\label{\detokenize{cascade.initialize:cascade.initialize.initialize.configurator}}\pysiglinewithargsret{\sphinxbfcode{\sphinxupquote{class }}\sphinxbfcode{\sphinxupquote{configurator}}}{\emph{*file\_names}}{}
Bases: \sphinxcode{\sphinxupquote{object}}
\index{isInitialized (configurator attribute)@\spxentry{isInitialized}\spxextra{configurator attribute}}

\begin{fulllineitems}
\phantomsection\label{\detokenize{cascade.initialize:cascade.initialize.initialize.configurator.isInitialized}}\pysigline{\sphinxbfcode{\sphinxupquote{isInitialized}}\sphinxbfcode{\sphinxupquote{ = False}}}
\end{fulllineitems}

\index{reset() (configurator method)@\spxentry{reset()}\spxextra{configurator method}}

\begin{fulllineitems}
\phantomsection\label{\detokenize{cascade.initialize:cascade.initialize.initialize.configurator.reset}}\pysiglinewithargsret{\sphinxbfcode{\sphinxupquote{reset}}}{}{}
\end{fulllineitems}


\end{fulllineitems}



\subsection{The cascade.instruments module}
\label{\detokenize{cascade.instruments:module-cascade.instruments.instruments}}\label{\detokenize{cascade.instruments:the-cascade-instruments-module}}\label{\detokenize{cascade.instruments::doc}}\index{cascade.instruments.instruments (module)@\spxentry{cascade.instruments.instruments}\spxextra{module}}
CASCADe

Observatory and Instruments specific Module

@author: bouwman
\index{Observation (class in cascade.instruments.instruments)@\spxentry{Observation}\spxextra{class in cascade.instruments.instruments}}

\begin{fulllineitems}
\phantomsection\label{\detokenize{cascade.instruments:cascade.instruments.instruments.Observation}}\pysigline{\sphinxbfcode{\sphinxupquote{class }}\sphinxbfcode{\sphinxupquote{Observation}}}
Bases: \sphinxcode{\sphinxupquote{object}}

This class handles the selection of the correct observatory and
instrument classes and loads the time series data to be analyzed

\end{fulllineitems}

\index{HST (class in cascade.instruments.instruments)@\spxentry{HST}\spxextra{class in cascade.instruments.instruments}}

\begin{fulllineitems}
\phantomsection\label{\detokenize{cascade.instruments:cascade.instruments.instruments.HST}}\pysigline{\sphinxbfcode{\sphinxupquote{class }}\sphinxbfcode{\sphinxupquote{HST}}}
Bases: \sphinxcode{\sphinxupquote{cascade.instruments.instruments.ObservatoryBase}}

This observatory class defines the instuments and data handling for the
spectropgraphs of the Spitzer Space telescope
\index{name (HST attribute)@\spxentry{name}\spxextra{HST attribute}}

\begin{fulllineitems}
\phantomsection\label{\detokenize{cascade.instruments:cascade.instruments.instruments.HST.name}}\pysigline{\sphinxbfcode{\sphinxupquote{name}}}
\end{fulllineitems}

\index{location (HST attribute)@\spxentry{location}\spxextra{HST attribute}}

\begin{fulllineitems}
\phantomsection\label{\detokenize{cascade.instruments:cascade.instruments.instruments.HST.location}}\pysigline{\sphinxbfcode{\sphinxupquote{location}}}
\end{fulllineitems}

\index{NAIF\_ID (HST attribute)@\spxentry{NAIF\_ID}\spxextra{HST attribute}}

\begin{fulllineitems}
\phantomsection\label{\detokenize{cascade.instruments:cascade.instruments.instruments.HST.NAIF_ID}}\pysigline{\sphinxbfcode{\sphinxupquote{NAIF\_ID}}}
\end{fulllineitems}

\index{observatory\_instruments (HST attribute)@\spxentry{observatory\_instruments}\spxextra{HST attribute}}

\begin{fulllineitems}
\phantomsection\label{\detokenize{cascade.instruments:cascade.instruments.instruments.HST.observatory_instruments}}\pysigline{\sphinxbfcode{\sphinxupquote{observatory\_instruments}}}
\end{fulllineitems}


\end{fulllineitems}

\index{HSTWFC3 (class in cascade.instruments.instruments)@\spxentry{HSTWFC3}\spxextra{class in cascade.instruments.instruments}}

\begin{fulllineitems}
\phantomsection\label{\detokenize{cascade.instruments:cascade.instruments.instruments.HSTWFC3}}\pysigline{\sphinxbfcode{\sphinxupquote{class }}\sphinxbfcode{\sphinxupquote{HSTWFC3}}}
Bases: \sphinxcode{\sphinxupquote{cascade.instruments.instruments.InstrumentBase}}

This instrument class defines the properties of the WFC3 instrument of
the Hubble Space Telescope
\index{name (HSTWFC3 attribute)@\spxentry{name}\spxextra{HSTWFC3 attribute}}

\begin{fulllineitems}
\phantomsection\label{\detokenize{cascade.instruments:cascade.instruments.instruments.HSTWFC3.name}}\pysigline{\sphinxbfcode{\sphinxupquote{name}}}
\end{fulllineitems}

\index{load\_data() (HSTWFC3 method)@\spxentry{load\_data()}\spxextra{HSTWFC3 method}}

\begin{fulllineitems}
\phantomsection\label{\detokenize{cascade.instruments:cascade.instruments.instruments.HSTWFC3.load_data}}\pysiglinewithargsret{\sphinxbfcode{\sphinxupquote{load\_data}}}{}{}
\end{fulllineitems}

\index{get\_instrument\_setup() (HSTWFC3 method)@\spxentry{get\_instrument\_setup()}\spxextra{HSTWFC3 method}}

\begin{fulllineitems}
\phantomsection\label{\detokenize{cascade.instruments:cascade.instruments.instruments.HSTWFC3.get_instrument_setup}}\pysiglinewithargsret{\sphinxbfcode{\sphinxupquote{get\_instrument\_setup}}}{}{}
Retrieve all relevant parameters defining the instrument and data setup

\end{fulllineitems}

\index{get\_spectra() (HSTWFC3 method)@\spxentry{get\_spectra()}\spxextra{HSTWFC3 method}}

\begin{fulllineitems}
\phantomsection\label{\detokenize{cascade.instruments:cascade.instruments.instruments.HSTWFC3.get_spectra}}\pysiglinewithargsret{\sphinxbfcode{\sphinxupquote{get\_spectra}}}{\emph{is\_background=False}}{}
read (uncalibrated) spectral timeseries, phase and wavelength

\end{fulllineitems}

\index{get\_spectral\_images() (HSTWFC3 method)@\spxentry{get\_spectral\_images()}\spxextra{HSTWFC3 method}}

\begin{fulllineitems}
\phantomsection\label{\detokenize{cascade.instruments:cascade.instruments.instruments.HSTWFC3.get_spectral_images}}\pysiglinewithargsret{\sphinxbfcode{\sphinxupquote{get\_spectral\_images}}}{\emph{is\_background=False}}{}
read uncalibrated spectral images (flt data product)

\end{fulllineitems}

\index{\_define\_convolution\_kernel() (HSTWFC3 method)@\spxentry{\_define\_convolution\_kernel()}\spxextra{HSTWFC3 method}}

\begin{fulllineitems}
\phantomsection\label{\detokenize{cascade.instruments:cascade.instruments.instruments.HSTWFC3._define_convolution_kernel}}\pysiglinewithargsret{\sphinxbfcode{\sphinxupquote{\_define\_convolution\_kernel}}}{}{}
Define the instrument specific convolution kernel which will be used
in the correction procedure of bad pixels

\end{fulllineitems}

\index{\_define\_region\_of\_interest() (HSTWFC3 method)@\spxentry{\_define\_region\_of\_interest()}\spxextra{HSTWFC3 method}}

\begin{fulllineitems}
\phantomsection\label{\detokenize{cascade.instruments:cascade.instruments.instruments.HSTWFC3._define_region_of_interest}}\pysiglinewithargsret{\sphinxbfcode{\sphinxupquote{\_define\_region\_of\_interest}}}{}{}
Defines region on detector which containes the intended target star.

\end{fulllineitems}

\index{\_get\_background\_cal\_data() (HSTWFC3 method)@\spxentry{\_get\_background\_cal\_data()}\spxextra{HSTWFC3 method}}

\begin{fulllineitems}
\phantomsection\label{\detokenize{cascade.instruments:cascade.instruments.instruments.HSTWFC3._get_background_cal_data}}\pysiglinewithargsret{\sphinxbfcode{\sphinxupquote{\_get\_background\_cal\_data}}}{}{}
Get the calibration data from which the background in the science
images can be determined.  For further details see:
\sphinxurl{http://www.stsci.edu/hst/wfc3/documents/ISRs/WFC3-2015-17.pdf}

\end{fulllineitems}

\index{\_fit\_background() (HSTWFC3 method)@\spxentry{\_fit\_background()}\spxextra{HSTWFC3 method}}

\begin{fulllineitems}
\phantomsection\label{\detokenize{cascade.instruments:cascade.instruments.instruments.HSTWFC3._fit_background}}\pysiglinewithargsret{\sphinxbfcode{\sphinxupquote{\_fit\_background}}}{\emph{science\_data\_in}}{}
Fits the background in the HST Grism data using the method described
in: \sphinxurl{http://www.stsci.edu/hst/wfc3/documents/ISRs/WFC3-2015-17.pdf}

\end{fulllineitems}

\index{\_determine\_relative\_source\_position() (HSTWFC3 method)@\spxentry{\_determine\_relative\_source\_position()}\spxextra{HSTWFC3 method}}

\begin{fulllineitems}
\phantomsection\label{\detokenize{cascade.instruments:cascade.instruments.instruments.HSTWFC3._determine_relative_source_position}}\pysiglinewithargsret{\sphinxbfcode{\sphinxupquote{\_determine\_relative\_source\_position}}}{\emph{spectral\_image\_cube}, \emph{mask}}{}
Determine the shift of the spectra (source) relative to the first
integration. Note that it is important for this to work properly
to have identified bad pixels and to correct the values using an edge
preserving correction, i.e. an correction which takes into account
the dispersion direction and psf size (relative to pixel size)
Input:
——
\begin{quote}

spectral\_image\_cube

mask
\end{quote}
\begin{quote}

relative x and y position as a function of time.
\end{quote}

\end{fulllineitems}

\index{\_determine\_source\_position\_from\_cal\_image() (HSTWFC3 method)@\spxentry{\_determine\_source\_position\_from\_cal\_image()}\spxextra{HSTWFC3 method}}

\begin{fulllineitems}
\phantomsection\label{\detokenize{cascade.instruments:cascade.instruments.instruments.HSTWFC3._determine_source_position_from_cal_image}}\pysiglinewithargsret{\sphinxbfcode{\sphinxupquote{\_determine\_source\_position\_from\_cal\_image}}}{\emph{calibration\_image\_cube}, \emph{calibration\_data\_files}}{}
Determines the source position on the detector of the target source in
the calibration image takes prior to the spectroscopic observations.

\end{fulllineitems}

\index{\_read\_grism\_configuration\_files() (HSTWFC3 method)@\spxentry{\_read\_grism\_configuration\_files()}\spxextra{HSTWFC3 method}}

\begin{fulllineitems}
\phantomsection\label{\detokenize{cascade.instruments:cascade.instruments.instruments.HSTWFC3._read_grism_configuration_files}}\pysiglinewithargsret{\sphinxbfcode{\sphinxupquote{\_read\_grism\_configuration\_files}}}{}{}
Gets the relevant data from WFC3 configuration files

\end{fulllineitems}

\index{\_read\_reference\_pixel\_file() (HSTWFC3 method)@\spxentry{\_read\_reference\_pixel\_file()}\spxextra{HSTWFC3 method}}

\begin{fulllineitems}
\phantomsection\label{\detokenize{cascade.instruments:cascade.instruments.instruments.HSTWFC3._read_reference_pixel_file}}\pysiglinewithargsret{\sphinxbfcode{\sphinxupquote{\_read\_reference\_pixel\_file}}}{}{}
Read the calibration file containig the definition
of the reference pixel appropriate for a given sub array and or filer

\end{fulllineitems}

\index{\_search\_ref\_pixel\_cal\_file() (HSTWFC3 static method)@\spxentry{\_search\_ref\_pixel\_cal\_file()}\spxextra{HSTWFC3 static method}}

\begin{fulllineitems}
\phantomsection\label{\detokenize{cascade.instruments:cascade.instruments.instruments.HSTWFC3._search_ref_pixel_cal_file}}\pysiglinewithargsret{\sphinxbfcode{\sphinxupquote{static }}\sphinxbfcode{\sphinxupquote{\_search\_ref\_pixel\_cal\_file}}}{\emph{ptable}, \emph{inst\_aperture}, \emph{inst\_filter}}{}
Search the reference pixel calibration file for the reference pixel
given the instrument aperture and filter.
See also \sphinxurl{http://www.stsci.edu/hst/observatory/apertures/wfc3.html}

\end{fulllineitems}

\index{\_get\_subarray\_size() (HSTWFC3 method)@\spxentry{\_get\_subarray\_size()}\spxextra{HSTWFC3 method}}

\begin{fulllineitems}
\phantomsection\label{\detokenize{cascade.instruments:cascade.instruments.instruments.HSTWFC3._get_subarray_size}}\pysiglinewithargsret{\sphinxbfcode{\sphinxupquote{\_get\_subarray\_size}}}{\emph{calibration\_data}, \emph{spectral\_data}}{}
\end{fulllineitems}

\index{\_get\_wavelength\_calibration() (HSTWFC3 method)@\spxentry{\_get\_wavelength\_calibration()}\spxextra{HSTWFC3 method}}

\begin{fulllineitems}
\phantomsection\label{\detokenize{cascade.instruments:cascade.instruments.instruments.HSTWFC3._get_wavelength_calibration}}\pysiglinewithargsret{\sphinxbfcode{\sphinxupquote{\_get\_wavelength\_calibration}}}{}{}
Return the wavelength calibration

\end{fulllineitems}

\index{get\_spectral\_trace() (HSTWFC3 method)@\spxentry{get\_spectral\_trace()}\spxextra{HSTWFC3 method}}

\begin{fulllineitems}
\phantomsection\label{\detokenize{cascade.instruments:cascade.instruments.instruments.HSTWFC3.get_spectral_trace}}\pysiglinewithargsret{\sphinxbfcode{\sphinxupquote{get\_spectral\_trace}}}{}{}
Get spectral trace

\end{fulllineitems}

\index{\_WFC3Trace() (HSTWFC3 static method)@\spxentry{\_WFC3Trace()}\spxextra{HSTWFC3 static method}}

\begin{fulllineitems}
\phantomsection\label{\detokenize{cascade.instruments:cascade.instruments.instruments.HSTWFC3._WFC3Trace}}\pysiglinewithargsret{\sphinxbfcode{\sphinxupquote{static }}\sphinxbfcode{\sphinxupquote{\_WFC3Trace}}}{\emph{xc}, \emph{yc}, \emph{DYDX}, \emph{xref=522}, \emph{yref=522}, \emph{xref\_grism=522}, \emph{yref\_grism=522}, \emph{subarray=256}, \emph{subarray\_grism=256}}{}
This function defines the spectral trace for the wfc3 grism modes.
Details can be found in:
\begin{quote}

\sphinxurl{http://www.stsci.edu/hst/wfc3/documents/ISRs/WFC3-2016-15.pdf}
\end{quote}
\begin{description}
\item[{and}] \leavevmode
\sphinxurl{http://www.stsci.edu/hst/observatory/apertures/wfc3.html}

\end{description}

\end{fulllineitems}

\index{\_WFC3Dispersion() (HSTWFC3 static method)@\spxentry{\_WFC3Dispersion()}\spxextra{HSTWFC3 static method}}

\begin{fulllineitems}
\phantomsection\label{\detokenize{cascade.instruments:cascade.instruments.instruments.HSTWFC3._WFC3Dispersion}}\pysiglinewithargsret{\sphinxbfcode{\sphinxupquote{static }}\sphinxbfcode{\sphinxupquote{\_WFC3Dispersion}}}{\emph{xc}, \emph{yc}, \emph{DYDX}, \emph{DLDP}, \emph{xref=522}, \emph{yref=522}, \emph{xref\_grism=522}, \emph{yref\_grism=522}, \emph{subarray=256}, \emph{subarray\_grism=256}}{}
Convert pixel coordinate to wavelength. Method and coefficient
adopted from Kuntschner et al. (2009), Wilkins et al. (2014). See also
\sphinxurl{http://www.stsci.edu/hst/wfc3/documents/ISRs/WFC3-2016-15.pdf}

In case the direct image and spectral image are not taken with the
same aperture, the centroid measurement is adjusted according to the
table in: \sphinxurl{http://www.stsci.edu/hst/observatory/apertures/wfc3.html}
\begin{quote}
\begin{description}
\item[{xc:}] \leavevmode
X coordinate of direct image centroid

\item[{yc:}] \leavevmode
Y coordinate of direct image centroid

\end{description}
\end{quote}

xref
yref
xref\_grism
yref\_grism
subarray
subarray\_grism
\begin{quote}
\begin{description}
\item[{wavelength:}] \leavevmode
return wavelength mapping of x coordinate in micron

\end{description}
\end{quote}

\end{fulllineitems}


\end{fulllineitems}

\index{Spitzer (class in cascade.instruments.instruments)@\spxentry{Spitzer}\spxextra{class in cascade.instruments.instruments}}

\begin{fulllineitems}
\phantomsection\label{\detokenize{cascade.instruments:cascade.instruments.instruments.Spitzer}}\pysigline{\sphinxbfcode{\sphinxupquote{class }}\sphinxbfcode{\sphinxupquote{Spitzer}}}
Bases: \sphinxcode{\sphinxupquote{cascade.instruments.instruments.ObservatoryBase}}

This observatory class defines the instuments and data handling for the
spectropgraphs of the Spitzer Space telescope
\index{name (Spitzer attribute)@\spxentry{name}\spxextra{Spitzer attribute}}

\begin{fulllineitems}
\phantomsection\label{\detokenize{cascade.instruments:cascade.instruments.instruments.Spitzer.name}}\pysigline{\sphinxbfcode{\sphinxupquote{name}}}
\end{fulllineitems}

\index{location (Spitzer attribute)@\spxentry{location}\spxextra{Spitzer attribute}}

\begin{fulllineitems}
\phantomsection\label{\detokenize{cascade.instruments:cascade.instruments.instruments.Spitzer.location}}\pysigline{\sphinxbfcode{\sphinxupquote{location}}}
\end{fulllineitems}

\index{NAIF\_ID (Spitzer attribute)@\spxentry{NAIF\_ID}\spxextra{Spitzer attribute}}

\begin{fulllineitems}
\phantomsection\label{\detokenize{cascade.instruments:cascade.instruments.instruments.Spitzer.NAIF_ID}}\pysigline{\sphinxbfcode{\sphinxupquote{NAIF\_ID}}}
\end{fulllineitems}

\index{observatory\_instruments (Spitzer attribute)@\spxentry{observatory\_instruments}\spxextra{Spitzer attribute}}

\begin{fulllineitems}
\phantomsection\label{\detokenize{cascade.instruments:cascade.instruments.instruments.Spitzer.observatory_instruments}}\pysigline{\sphinxbfcode{\sphinxupquote{observatory\_instruments}}}
\end{fulllineitems}


\end{fulllineitems}

\index{SpitzerIRS (class in cascade.instruments.instruments)@\spxentry{SpitzerIRS}\spxextra{class in cascade.instruments.instruments}}

\begin{fulllineitems}
\phantomsection\label{\detokenize{cascade.instruments:cascade.instruments.instruments.SpitzerIRS}}\pysigline{\sphinxbfcode{\sphinxupquote{class }}\sphinxbfcode{\sphinxupquote{SpitzerIRS}}}
Bases: \sphinxcode{\sphinxupquote{cascade.instruments.instruments.InstrumentBase}}

This instrument class defines the properties of the IRS instrument of
the Spitzer Space Telescope
\index{name (SpitzerIRS attribute)@\spxentry{name}\spxextra{SpitzerIRS attribute}}

\begin{fulllineitems}
\phantomsection\label{\detokenize{cascade.instruments:cascade.instruments.instruments.SpitzerIRS.name}}\pysigline{\sphinxbfcode{\sphinxupquote{name}}}
\end{fulllineitems}

\index{load\_data() (SpitzerIRS method)@\spxentry{load\_data()}\spxextra{SpitzerIRS method}}

\begin{fulllineitems}
\phantomsection\label{\detokenize{cascade.instruments:cascade.instruments.instruments.SpitzerIRS.load_data}}\pysiglinewithargsret{\sphinxbfcode{\sphinxupquote{load\_data}}}{}{}
\end{fulllineitems}

\index{get\_instrument\_setup() (SpitzerIRS method)@\spxentry{get\_instrument\_setup()}\spxextra{SpitzerIRS method}}

\begin{fulllineitems}
\phantomsection\label{\detokenize{cascade.instruments:cascade.instruments.instruments.SpitzerIRS.get_instrument_setup}}\pysiglinewithargsret{\sphinxbfcode{\sphinxupquote{get\_instrument\_setup}}}{}{}
Retrieve all relevant parameters defining the instrument and data setup

\end{fulllineitems}

\index{get\_spectra() (SpitzerIRS method)@\spxentry{get\_spectra()}\spxextra{SpitzerIRS method}}

\begin{fulllineitems}
\phantomsection\label{\detokenize{cascade.instruments:cascade.instruments.instruments.SpitzerIRS.get_spectra}}\pysiglinewithargsret{\sphinxbfcode{\sphinxupquote{get\_spectra}}}{\emph{is\_background=False}}{}
read uncalibrated spectral timeseries, phase and wavelength

\end{fulllineitems}

\index{get\_spectral\_images() (SpitzerIRS method)@\spxentry{get\_spectral\_images()}\spxextra{SpitzerIRS method}}

\begin{fulllineitems}
\phantomsection\label{\detokenize{cascade.instruments:cascade.instruments.instruments.SpitzerIRS.get_spectral_images}}\pysiglinewithargsret{\sphinxbfcode{\sphinxupquote{get\_spectral\_images}}}{\emph{is\_background=False}}{}
read uncalibrated spectral images

Notes on FOV:

\# in the fits header the following relevant info is used:
\# FOVID     26     IRS\_Short-Lo\_1st\_Order\_1st\_Position
\# FOVID     27     IRS\_Short-Lo\_1st\_Order\_2nd\_Position
\# FOVID     28     IRS\_Short-Lo\_1st\_Order\_Center\_Position
\# FOVID     29     IRS\_Short-Lo\_Module\_Center
\# FOVID     32     IRS\_Short-Lo\_2nd\_Order\_1st\_Position
\# FOVID     33     IRS\_Short-Lo\_2nd\_Order\_2nd\_Position
\# FOVID     34     IRS\_Short-Lo\_2nd\_Order\_Center\_Position
\# FOVID     40     IRS\_Long-Lo\_1st\_Order\_Center\_Position
\# FOVID     46     IRS\_Long-Lo\_2nd\_Order\_Center\_Position

Notes on timing:

\# FRAMTIME the total effective exposure time (ramp length) in seconds

\end{fulllineitems}

\index{\_define\_convolution\_kernel() (SpitzerIRS method)@\spxentry{\_define\_convolution\_kernel()}\spxextra{SpitzerIRS method}}

\begin{fulllineitems}
\phantomsection\label{\detokenize{cascade.instruments:cascade.instruments.instruments.SpitzerIRS._define_convolution_kernel}}\pysiglinewithargsret{\sphinxbfcode{\sphinxupquote{\_define\_convolution\_kernel}}}{}{}
Define the instrument specific convolution kernel which will be used
in the correction procedure of bad pixels

\end{fulllineitems}

\index{\_define\_region\_of\_interest() (SpitzerIRS method)@\spxentry{\_define\_region\_of\_interest()}\spxextra{SpitzerIRS method}}

\begin{fulllineitems}
\phantomsection\label{\detokenize{cascade.instruments:cascade.instruments.instruments.SpitzerIRS._define_region_of_interest}}\pysiglinewithargsret{\sphinxbfcode{\sphinxupquote{\_define\_region\_of\_interest}}}{}{}
Defines region on detector which containes the intended target star.

\end{fulllineitems}

\index{\_get\_order\_mask() (SpitzerIRS method)@\spxentry{\_get\_order\_mask()}\spxextra{SpitzerIRS method}}

\begin{fulllineitems}
\phantomsection\label{\detokenize{cascade.instruments:cascade.instruments.instruments.SpitzerIRS._get_order_mask}}\pysiglinewithargsret{\sphinxbfcode{\sphinxupquote{\_get\_order\_mask}}}{}{}
Gets the mask which defines the pixels used with a given spectral order

\end{fulllineitems}

\index{\_get\_wavelength\_calibration() (SpitzerIRS method)@\spxentry{\_get\_wavelength\_calibration()}\spxextra{SpitzerIRS method}}

\begin{fulllineitems}
\phantomsection\label{\detokenize{cascade.instruments:cascade.instruments.instruments.SpitzerIRS._get_wavelength_calibration}}\pysiglinewithargsret{\sphinxbfcode{\sphinxupquote{\_get\_wavelength\_calibration}}}{}{}
Get wavelength calibration file

\end{fulllineitems}

\index{get\_detector\_cubes() (SpitzerIRS method)@\spxentry{get\_detector\_cubes()}\spxextra{SpitzerIRS method}}

\begin{fulllineitems}
\phantomsection\label{\detokenize{cascade.instruments:cascade.instruments.instruments.SpitzerIRS.get_detector_cubes}}\pysiglinewithargsret{\sphinxbfcode{\sphinxupquote{get\_detector\_cubes}}}{\emph{is\_background=False}}{}
Get detector cube data

Notes on timing in header:

There are several integration-time-related keywords.
Of greatest interest to the observer is the
“effective integration time”, which is the time on-chip between
the first and last non-destructive reads for each pixel. It is called:
\begin{quote}

RAMPTIME = Total integration time for the current DCE.
\end{quote}

The value of RAMPTIME gives the usable portion of the integration ramp,
occurring between the beginning of the first read and the end of the
last read. It excludes detector array pre-conditioning time.
It may also be of interest to know the exposure time at other points
along the ramp. The SUR sequence consists of the time taken at the
beginning of a SUR sequence to condition the array
(header keyword DEADTIME), the time taken to complete one read and
one spin through the array (GRPTIME), and the non-destructive reads
separated by uniform wait times. The wait consists of “clocking”
through the array without reading or resetting. The time it takes to
clock through the array once is given by the SAMPTIME keyword.
So, for an N-read ramp:
\begin{quote}

RAMPTIME = 2x(N-1)xSAMPTIME
\end{quote}
\begin{description}
\item[{and}] \leavevmode
DCE duration = DEADTIME + GRPTIME + RAMPTIME

\end{description}

Note that peak-up data is not obtained in SUR mode. It is obtained in
Double Correlated Sampling (DCS) mode. In that case, RAMPTIME gives the
time interval between the 2nd sample and the preceeding reset.

\end{fulllineitems}

\index{get\_spectral\_trace() (SpitzerIRS method)@\spxentry{get\_spectral\_trace()}\spxextra{SpitzerIRS method}}

\begin{fulllineitems}
\phantomsection\label{\detokenize{cascade.instruments:cascade.instruments.instruments.SpitzerIRS.get_spectral_trace}}\pysiglinewithargsret{\sphinxbfcode{\sphinxupquote{get\_spectral\_trace}}}{}{}
Get spectral trace

\end{fulllineitems}


\end{fulllineitems}



\subsection{The cascade.utilities module}
\label{\detokenize{cascade.utilities:module-cascade.utilities.utilities}}\label{\detokenize{cascade.utilities:the-cascade-utilities-module}}\label{\detokenize{cascade.utilities::doc}}\index{cascade.utilities.utilities (module)@\spxentry{cascade.utilities.utilities}\spxextra{module}}
This Module defines some utility functions used in cascade

@author: bouwman
\index{write\_timeseries\_to\_fits() (in module cascade.utilities.utilities)@\spxentry{write\_timeseries\_to\_fits()}\spxextra{in module cascade.utilities.utilities}}

\begin{fulllineitems}
\phantomsection\label{\detokenize{cascade.utilities:cascade.utilities.utilities.write_timeseries_to_fits}}\pysiglinewithargsret{\sphinxbfcode{\sphinxupquote{write\_timeseries\_to\_fits}}}{\emph{data}, \emph{path}}{}
Write spectral timeseries data object to fits files

\end{fulllineitems}

\index{find() (in module cascade.utilities.utilities)@\spxentry{find()}\spxextra{in module cascade.utilities.utilities}}

\begin{fulllineitems}
\phantomsection\label{\detokenize{cascade.utilities:cascade.utilities.utilities.find}}\pysiglinewithargsret{\sphinxbfcode{\sphinxupquote{find}}}{\emph{pattern}, \emph{path}}{}
Return  a list of all data files

\end{fulllineitems}

\index{spectres() (in module cascade.utilities.utilities)@\spxentry{spectres()}\spxextra{in module cascade.utilities.utilities}}

\begin{fulllineitems}
\phantomsection\label{\detokenize{cascade.utilities:cascade.utilities.utilities.spectres}}\pysiglinewithargsret{\sphinxbfcode{\sphinxupquote{spectres}}}{\emph{new\_spec\_wavs}, \emph{old\_spec\_wavs}, \emph{spec\_fluxes}, \emph{spec\_errs=None}}{}
SpectRes: A fast spectral resampling function.
Copyright (C) 2017  A. C. Carnall
Function for resampling spectra (and optionally associated uncertainties)
onto a new wavelength basis.
\begin{quote}\begin{description}
\item[{Parameters}] \leavevmode\begin{itemize}
\item {} 
\sphinxstyleliteralstrong{\sphinxupquote{new\_spec\_wavs}} (\sphinxstyleliteralemphasis{\sphinxupquote{numpy.ndarray}}) \textendash{} Array containing the new wavelength sampling desired for the spectrum
or spectra.

\item {} 
\sphinxstyleliteralstrong{\sphinxupquote{old\_spec\_wavs}} (\sphinxstyleliteralemphasis{\sphinxupquote{numpy.ndarray}}) \textendash{} 1D array containing the current wavelength sampling of the spectrum or
spectra.

\item {} 
\sphinxstyleliteralstrong{\sphinxupquote{spec\_fluxes}} (\sphinxstyleliteralemphasis{\sphinxupquote{numpy.ndarray}}) \textendash{} Array containing spectral fluxes at the wavelengths specified in
old\_spec\_wavs, last dimension must correspond to the shape of
old\_spec\_wavs.
Extra dimensions before this may be used to include multiple spectra.

\item {} 
\sphinxstyleliteralstrong{\sphinxupquote{spec\_errs}} (\sphinxstyleliteralemphasis{\sphinxupquote{numpy.ndarray}}\sphinxstyleliteralemphasis{\sphinxupquote{ (}}\sphinxstyleliteralemphasis{\sphinxupquote{optional}}\sphinxstyleliteralemphasis{\sphinxupquote{)}}) \textendash{} Array of the same shape as spec\_fluxes containing uncertainties
associated with each spectral flux value.

\end{itemize}

\item[{Returns}] \leavevmode
\begin{itemize}
\item {} 
\sphinxstylestrong{resampled\_fluxes} (\sphinxstyleemphasis{numpy.ndarray}) \textendash{} Array of resampled flux values, first dimension is the same length
as new\_spec\_wavs, other dimensions are the same as spec\_fluxes

\item {} 
\sphinxstylestrong{resampled\_errs} (\sphinxstyleemphasis{numpy.ndarray}) \textendash{} Array of uncertainties associated with fluxes in resampled\_fluxes.
Only returned if spec\_errs was specified.

\end{itemize}


\end{description}\end{quote}

\end{fulllineitems}



\chapter{Indices and tables}
\label{\detokenize{index:indices-and-tables}}\begin{itemize}
\item {} 
\DUrole{xref,std,std-ref}{genindex}

\item {} 
\DUrole{xref,std,std-ref}{search}

\end{itemize}


\renewcommand{\indexname}{Python Module Index}
\begin{sphinxtheindex}
\let\bigletter\sphinxstyleindexlettergroup
\bigletter{c}
\item\relax\sphinxstyleindexentry{cascade.cpm\_model.cpm\_model}\sphinxstyleindexpageref{cascade.cpm_model:\detokenize{module-cascade.cpm_model.cpm_model}}
\item\relax\sphinxstyleindexentry{cascade.data\_model.data\_model}\sphinxstyleindexpageref{cascade.data_model:\detokenize{module-cascade.data_model.data_model}}
\item\relax\sphinxstyleindexentry{cascade.exoplanet\_tools.exoplanet\_tools}\sphinxstyleindexpageref{cascade.exoplanet_tools:\detokenize{module-cascade.exoplanet_tools.exoplanet_tools}}
\item\relax\sphinxstyleindexentry{cascade.initialize.initialize}\sphinxstyleindexpageref{cascade.initialize:\detokenize{module-cascade.initialize.initialize}}
\item\relax\sphinxstyleindexentry{cascade.instruments.instruments}\sphinxstyleindexpageref{cascade.instruments:\detokenize{module-cascade.instruments.instruments}}
\item\relax\sphinxstyleindexentry{cascade.TSO.TSO}\sphinxstyleindexpageref{cascade.TSO:\detokenize{module-cascade.TSO.TSO}}
\item\relax\sphinxstyleindexentry{cascade.utilities.utilities}\sphinxstyleindexpageref{cascade.utilities:\detokenize{module-cascade.utilities.utilities}}
\end{sphinxtheindex}

\renewcommand{\indexname}{Index}
\printindex
\end{document}